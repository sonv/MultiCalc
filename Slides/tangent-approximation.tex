\documentclass[aspectratio=169]{beamer}
\usetheme{Copenhagen}
%% Remove draft for real article, put twocolumn for two columns
\usetheme{metropolis}
\usepackage{multicol}

\usepackage[utf8]{inputenc}
\newtheorem*{question}{Question}

\newcommand{\vectorproj}[2][]{\mathrm{proj}_{\vect{#1}}\vect{#2}}
\newcommand{\vectorcomp}[2][]{\mathrm{comp}_{\vect{#1}}\vect{#2}}
\newcommand{\vect}{\mathbf}
\newcommand{\R}{\mathbb{R}}
%% commentary bubble
\newcommand{\SV}[2][]{\sidenote[colback=green!10]{\textbf{SV\xspace #1:} #2}}

%% Title 
\title{ Multivariable Calculus } 
\institute{Fulbright University Vietnam}
%\author[1]{Co-author}
\author{Truong-Son Van}
\date{03/04/2024}

\begin{document}

\maketitle

\begin{frame}
    \frametitle{Level sets}
    We will restrict our attention to functions with one output for the next few weeks.
    $$f : \R^n \to \R.$$
    A level $c$-level set of a function is defined to be
    $$ f^{-1}(c) = \{ x: f(x) = c  \}.$$
    If $n = 2$, the level set is called the level curve.

    If $n = 3$, the level set is called the level surface.
\end{frame}

\begin{frame}
    \frametitle{ $c$-level curve }
    Suppose one can parametrize the $c$-level curve by $\vect{r}(t)$.

    That means 
    \begin{equation*}
        \frac{d}{dt} F(\vect{r}(t)) = \nabla F (\vect{r}(t)) \cdot \vect{r}'(t)= 0 \,.
    \end{equation*}
    Suppose at $t_0$, $\vect{r}(t_0) = ( a,b )$.
    We then have that, the tangent line of the $c$-level curve of $F$ at $(a,b)$
    must satisfy the relation
    \begin{equation*}
        \nabla F(a,b) \cdot \langle x_1 - a, x_2 - b \rangle = 0\,.
    \end{equation*}
    Another way to write this:
    \begin{equation*}
        \partial_{x_1} F (a,b) (x_1 -a ) + \partial_{x_2} F(a,b) (x_2 - b) = 0\,.
    \end{equation*}

    \url{https://www.youtube.com/watch?v=ZTbTYEMvo10}

\end{frame}

\begin{frame}
    \frametitle{$c$-level surface}
    Similar to $c$-level curves, a $c$-level surface is a surface that satisfies
    \begin{equation*}
        F(x,y,z) = c \,.
    \end{equation*}
    Reasoning similarly to the case of the $c$-level curve, we have that
    for ANY curve $\vect{r}(t)$ on the $c$-level surface,
    \begin{equation*}
        \frac{d}{dt} F(\vect{r}(t)) = \nabla F (\vect{r}(t)) \cdot \vect{r}'(t)= 0  \,.
    \end{equation*}
    That means for any curve that goes through the point $(a,b,c)$ at time $t_0$, it must be the case that
    \begin{equation*}
        \nabla F(a,b,c) \cdot \vect{r}'(t_0) = 0 \,.
    \end{equation*}
\end{frame}

\begin{frame}
    $\implies \nabla F(a,b,c)$ is perpendicular to ALL curves on the $c$-level surface
    that goes through $(a,b,c)$. \pause

    $\implies$ There is ONE vector that perpendicular to ANY 2 curves on the $c$-level surface at the point $(a,b,c)$. \pause

    $\implies$ The plane made by this very one vector will be tangent to ALL the curves. \pause

    $\implies$ The tangent plane is unique  and satisfies the formula (analogous to the
    curve case)

    \begin{equation*}
        \nabla F(a,b,c) \cdot \langle x-a, y-b, z-c \rangle = 0 \,.
    \end{equation*}

\end{frame}

\section{Steepest ascent}

\begin{frame}
    \frametitle{ Steepest ascent}
    Directional derivative tells you the change of the function $F$ in certain direction $\vect{u}$.

    Question: what is the direction that gives me the largest change PER UNIT? \pause

    We know that for a unit vector $\vect{u}$,
    \begin{equation*}
        D_{\vect{u}} f (\vect{x}_0) = \nabla f(\vect{x_0}) \cdot \vect{u} = | \nabla f(\vect{x_0})| \cos\theta \,.
    \end{equation*}
    \pause
    This is maximized at $\theta = 0$\,.

    \url{https://www.youtube.com/watch?v=TEB2z7ZlRAw}
\end{frame}

\begin{frame}
    Let $f: \R \to \R$
    and $T(x)$ be a linear function such that $T(a) = f(a)$.
    So, 
    \begin{equation*}
        T(x) = f(a) + m(x-a) \,.
    \end{equation*}
    This function is any arbitrary linear function that may give us an ``approximation''
    for $f$ near $a$. \pause

    To determine the ``best'' approximation, we want to see what happens to
    $f(x) - T(x)$ as $x\to a$. \pause

    Naturally, $\lim_{x\to a} f(x) - T(x) = 0$, by the first requirement. \pause

    \textbf{Can we do better?}
\end{frame}

\begin{frame}
    From Taylor's theorem,
    if 
    \begin{equation}
        T(x) = f(a) + f'(a) (x-a) \label{eq:linearapprox}
    \end{equation}
    then \pause
    \begin{equation}
        \lim_{x\to a} \frac{f(x) - T(x)}{|x-a|} = 0 \,, \label{eq:differentiable}
    \end{equation}
    which is a more significant statement than just
    \begin{equation*}
        \lim_{x\to a} f(x) - T(x) = 0 \,.
    \end{equation*}
    \pause
    From analysis class (take it when you have a chance!), 
   we will know that~\eqref{eq:differentiable} is the
    necessary and sufficient condition to 
    determine the linear approximation you learned before (e.g. \eqref{eq:linearapprox}).

    \pause
    Furthermore, condition~\eqref{eq:differentiable} is equivalent to the definition
    of differentiability.
    This justifies the complicated definition we learned in higher dimensions.
\end{frame}




\end{document}

