\documentclass[12pt]{amsart}
\usepackage{amsaddr}
\usepackage{marktext} 
%% Remove draft for real article, put twocolumn for two columns
\usepackage{svmacro}
\usepackage[utf8]{inputenc}
\usepackage{lineno}
\usepackage[style=alphabetic, backend=biber]{biblatex}
\addbibresource{bibliography.bib}

%% commentary bubble
\newcommand{\SV}[2][]{\sidenote[colback=green!10]{\textbf{SV\xspace #1:} #2}}

%% Title 
\title{ MATH 104: Worksheet 1}
\author{}

%\author{Co-author}
%\address{  }
%\email {  }
%
\date{\today}

\begin{document}

\maketitle

\section{Concepts}

\begin{enumerate}
    \item Lines \& planes in 2D \& 3D
    \item Curves \& surfaces in 2D \& 3D
    \item Implicit and parametric representations
\end{enumerate}

\section{Discussions}

\begin{question}
    What happens if you take the equation of a line in 2D, say $2x - 3y =7$
    and interpret it in 3D?
\end{question}

\begin{question}
    \begin{enumerate}
        \item What does each of
    \begin{equation*}
        3x + y - z = 4 
    \end{equation*}
    and
    \begin{equation*}
         x - 2y + z = 1 
    \end{equation*}
    represent? 

    \item If taking both of the above equation together, what do they represent?
    Is there another way to represent this object?
    \end{enumerate}
\end{question}

\begin{question}
    Where does the line 
    \begin{equation*}
        x(t) = 2t-1 \,; y(t) = 3t + 2 \,; z(t) = 4t
    \end{equation*}
    intersect the plane given by $4x + 3y - z = 3$?

    What happens if it's not a plane but a more general surface?
\end{question}

\begin{question}
    \begin{enumerate}
        \item Given two lines in 2D, what is there intersection? 
        \item What could it be?
        \item What about intersection two lines in 3D?
        \item What about intersection of a line and a plane in 3D?
        \item What about intersection of two planes in 3D?
    \end{enumerate}
\end{question}

\printbibliography 
%\bibliography{refs}
%\bibliographystyle{halpha-abbrv}


\end{document}
