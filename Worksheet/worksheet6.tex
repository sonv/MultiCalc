\documentclass[12pt]{amsart}
\usepackage{amsaddr}
\usepackage{marktext} 
%% Remove draft for real article, put twocolumn for two columns
\usepackage{svmacro}
\usepackage[utf8]{inputenc}
\usepackage{lineno}
\usepackage[style=alphabetic, backend=biber]{biblatex}
\addbibresource{bibliography.bib}

%% commentary bubble
\newcommand{\SV}[2][]{\sidenote[colback=green!10]{\textbf{SV\xspace #1:} #2}}

%% Title 
\title{ MATH 104: Worksheet 6}
\author{}

%\author{Co-author}
%\address{  }
%\email {  }
%
\date{01/28/2024}

\begin{document}

\maketitle

\section{Concepts}

\begin{enumerate}
    \item Multivariable functions
    \item Partial derivatives
    \item Derivative as matrix of partial derivatives
\end{enumerate}

\section{Discussions}

\begin{question}
    Let 
$$ F(x,y,z) = \frac{x^2\sqrt{y^3}}{z^4}.$$
Compute
$$ \frac{\partial F}{\partial x}\,, \quad \frac{\partial F}{\partial y} \,, \quad \frac{\partial F}{\partial z}$$
\end{question}

\begin{question}
    Recall that the equation for a paraboloid is
    $$ z = x^2 + y^2. $$
    This can be parametrized by the following equation $G: \R^2 \to \R^3$,
    $$ G(s,t) = \begin{pmatrix}
        s \\ t \\ s^2 + t^2 
    \end{pmatrix} \,.$$
    Compute the derivative, as a matrix, of this parametrization.

    Does it matter how we write the matrix?
\end{question}

\begin{question}
    The Cobb-Douglas model is a classic model in economics.
    It says production, $P$, is a function of materials, $M$, and labor $L$ 
    via the following relationship
    \begin{equation*}
        P = \kappa M^\alpha L^\beta
    \end{equation*}
    where $\kappa >0$, $0<\alpha, \beta <1$ and $\alpha + \beta = 1$.
    If the investment in labor is increased and the investment in materials
    is decreased at an equal rate, what is the impact on production?
\end{question}

\begin{question}
    Consider a function $f$ such that, at a particular point $a$,
    \begin{equation*}
        [Df]_a\begin{pmatrix}
            1 \\ -1
        \end{pmatrix}
        = \begin{pmatrix}
            3 \\ -2
        \end{pmatrix}.
    \end{equation*}
    \begin{enumerate}
        \item How many inputs does $f$ have?
        \item What happens if inputs change at rates $\vec{h} = \langle -2, 2 \rangle$?
        \item What if $\vec{h} = \langle 3, 3 \rangle$? Can you do this?
        \item Can you do the previous problem if you know 
            $$[Df]_a \begin{pmatrix}
                1 \\2 
            \end{pmatrix}
            = 
            \begin{pmatrix}
                4 \\ -4
            \end{pmatrix}?
            $$
    \end{enumerate}
\end{question}

\begin{question}
    Explain the velocity vector of a parametrized curve $\gamma(t)$ in terms of the definition 
    of a derivative.
\end{question}

\end{document}
