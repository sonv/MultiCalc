\documentclass[12pt]{amsart}
\usepackage{amsaddr}
\usepackage{marktext} 
%% Remove draft for real article, put twocolumn for two columns
\usepackage{svmacro}
\usepackage[utf8]{inputenc}
\usepackage{lineno}
\usepackage[style=alphabetic, backend=biber]{biblatex}
\addbibresource{bibliography.bib}

%% commentary bubble
\newcommand{\SV}[2][]{\sidenote[colback=green!10]{\textbf{SV\xspace #1:} #2}}

%% Title 
\title{ MATH 104: Worksheet 7}
\author{}

%\author{Co-author}
%\address{  }
%\email {  }
%
\date{01/31/2024}

\begin{document}


\maketitle

\section{Concepts}

\begin{enumerate}
    \item Multivariable functions
    \item Partial derivatives
    \item Derivative as matrix of partial derivatives
    \item Derivative as transformation
\end{enumerate}

\section{Discussions}

\begin{question}
    Consider a function $f$ such that, at a particular point $a$,
    \begin{equation*}
        [Df]_a\begin{pmatrix}
            1 \\ -1
        \end{pmatrix}
        = \begin{pmatrix}
            3 \\ -2
        \end{pmatrix}.
    \end{equation*}
    \begin{enumerate}
        \item How many inputs does $f$ have?
        \item What happens if inputs change at rates $\vec{h} = \langle -2, 2 \rangle$?
        \item What if $\vec{h} = \langle 3, 3 \rangle$? Can you do this?
        \item Can you do the previous problem if you know 
            $$[Df]_a \begin{pmatrix}
                1 \\2 
            \end{pmatrix}
            = 
            \begin{pmatrix}
                4 \\ -4
            \end{pmatrix}?
            $$
    \end{enumerate}
\end{question}

\begin{question}
    Explain the velocity vector of a parametrized curve $\gamma(t)$ in terms of the definition 
    of a derivative.
\end{question}

\begin{question}
    Consider the following 
    \begin{equation*}
        f
        \begin{pmatrix}
            u \\ v \\ w
        \end{pmatrix}
        =
        \begin{pmatrix}
            u^2 v^{-3} w \\ 2u - 5w \\ uv - vw
        \end{pmatrix} \,.
    \end{equation*}
    \begin{enumerate}
        \item Compute the derivative $[Df]$.
        \item Evaluate this derivative at the point where $u = 1, v = -1, w = 2$.
        \item If, at this point,all the inputs are decreasing at the same rate, which output is increasing the most?
    \end{enumerate}
\end{question}

\begin{question}
    From Calculus, we know that if $f:\R \to \R$ and $g: \R \to \R$, then
    \begin{equation*}
        (g\circ f)'(x) = g'(f(a)) f'(a) \,. 
    \end{equation*}
    Now, if we have $f: \R^n \to \R^m$ and $g: \R^m \to \R^l$, what is your guess of
    $[D (g\circ f)]_a$?
\end{question}

\end{document}
