\documentclass[12pt]{amsart}
\usepackage{amsaddr}
\usepackage{marktext} 
%% Remove draft for real article, put twocolumn for two columns
\usepackage{svmacro}
\usepackage[utf8]{inputenc}
\usepackage{lineno}
\usepackage[style=alphabetic, backend=biber]{biblatex}
\addbibresource{bibliography.bib}

%% commentary bubble
\newcommand{\SV}[2][]{\sidenote[colback=green!10]{\textbf{SV\xspace #1:} #2}}

%% Title 
\title{ MATH 104: Worksheet 1}
\author{}

%\author{Co-author}
%\address{  }
%\email {  }
%
\date{\today}

\begin{document}

\maketitle

\section{Concepts}

\begin{enumerate}
	\item Surfaces
	\item Vectors
\end{enumerate}

\section{Discussions}

\begin{definition}[Line in 2D]
	And equation with two variables $f(x,y) = 0$ represents a line in $\R^2$.
\end{definition}

\begin{definition}[Surface in 3D]
	And equation with three variables $f(x,y,z) = 0$ represents a surface in $\R^3$.
\end{definition}

\begin{definition}[Hypersurface]
	And equation with $n$ variables $f(x_1, \dots, x_n) = 0$ represents a hypersurface in $\R^n$.
\end{definition}

\begin{question}
	What are the surfaces that the following equations represent in $\R^3$?
	\begin{enumerate}
		\item $(x-1)^2 + (y-1)^2 + (z-2)^2 = 9$
		      \vspace{5cm}
		\item $(x-1)^2 + \frac{1}{4} (y -3)^2 = 4$, $1 \leq z \leq 2$
		      \vspace{5cm}
		\item $3x + 3y - 5z = 0$
		      \vspace{5cm}
	\end{enumerate}
\end{question}

\begin{definition}
	A point $x \in \R^3$ is a tuple
	\begin{equation*}
		x = ( x_1, x_2, x_3 ) \,.
	\end{equation*}
	A vector $\vec{x} \in \R^3$ is a tuple
	\begin{equation*}
		\vec{v} = \langle v_1, v_2, v_3 \rangle \,.
	\end{equation*}
	A vector represents a quantity that has a length and a direction (starting and ending points are not important).
\end{definition}

\begin{question}
	What is the length of
	\begin{enumerate}
		\item $\langle 1,2,3 \rangle$?
		      \vspace{5cm}
		\item $\vec{v} + \vec{w}$ where $\vec{v} = \langle  1,2,3 \rangle$ and $\vec{w} = \langle 2,3,1  \rangle$ ?
		      \vspace{5cm}
		\item $\vec{w} = 10 \vec{v}$ where $\vec{v} = \langle  1,2,3 \rangle$?
		      \vspace{5cm}
	\end{enumerate}
\end{question}



\printbibliography
%\bibliography{refs}
%\bibliographystyle{halpha-abbrv}


\end{document}
