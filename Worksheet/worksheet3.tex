\documentclass[12pt]{amsart}
\usepackage{amsaddr}
\usepackage{marktext} 
%% Remove draft for real article, put twocolumn for two columns
\usepackage{svmacro}
\usepackage[utf8]{inputenc}
\usepackage{lineno}
\usepackage[style=alphabetic, backend=biber]{biblatex}
\addbibresource{bibliography.bib}

%% commentary bubble
\newcommand{\SV}[2][]{\sidenote[colback=green!10]{\textbf{SV\xspace #1:} #2}}

%% Title 
\title{ MATH 104: Worksheet 3}
\author{}

%\author{Co-author}
%\address{  }
%\email {  }
%
\date{01/17/2024}

\begin{document}

\maketitle

\section{Concepts}

\begin{enumerate}
    \item Velocity and accelerations
    \item Unit tangent and unit normal vectors to curves
    \item Curvature and geometry of curves
\end{enumerate}

\section{Discussions}

\begin{question}
    If you move along a curve in 3D at a constant speed, what can you say about acceleration? Is it zero?
\end{question}

\begin{question}
    Compute the velocity, acceleration, and arclength element of the curve with components $(\sin t, \cos 2t, t)$.
\end{question}

\begin{question}
    Compute the length of a general helix in 3D with radius R and height C.
    What are the asymptotics for small $R$ and $C$?
\end{question}

\begin{question}
    Compute the arclength of the parametrized curve in 4D:
    \begin{equation*}
        \gamma(t) = \begin{pmatrix}
            A \cos t\\ A\sin t \\ B \cos t  \\ B \sin t
        \end{pmatrix}
        \,, 0 \leq t \leq 2\pi
    \end{equation*}
\end{question}

\begin{question}
    It was stated that the unit tangent $\hat T$ and the unit normal $\hat N$ to a curve
    are always orthogonal. Why?
\end{question}

\printbibliography 
%\bibliography{refs}
%\bibliographystyle{halpha-abbrv}


\end{document}
