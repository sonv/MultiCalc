\documentclass[12pt]{amsart}
\usepackage{amsaddr}
\usepackage{marktext} 
%% Remove draft for real article, put twocolumn for two columns
\usepackage{svmacro}
\usepackage[utf8]{inputenc}
\usepackage{lineno}
\usepackage[style=alphabetic, backend=biber]{biblatex}
\addbibresource{bibliography.bib}

%% commentary bubble
\newcommand{\SV}[2][]{\sidenote[colback=green!10]{\textbf{SV\xspace #1:} #2}}

%% Title 
\title{ MATH 104: Worksheet 14}
\author{}

%\author{Co-author}
%\address{  }
%\email {  }
%
\date{\today}

\begin{document}

\maketitle

\section{Concepts}

\begin{enumerate}
	\item Maximum and minimum values
\end{enumerate}

A function of two variables has a \textbf{local maximum} at $(a, b)$ if
\[
	f(x, y) \leq f(a, b)
\]
when $(x, y)$ is near $(a, b)$. [This means that $f(x, y) \leq f(a, b)$ for all points $(x, y)$ in some disk with center $(a, b)$.] The number $f(a, b)$ is called a \textbf{local maximum value}. If
\[
	f(x, y) \geq f(a, b)
\]
when $(x, y)$ is near $(a, b)$, then $f$ has a \textbf{local minimum} at $(a, b)$ and $f(a, b)$ is a \textbf{local minimum value}.


If the inequalities in Definition 1 hold for \textit{all} points $(x, y)$ in the domain of $f$, then $f$ has an \textbf{absolute maximum} (or \textbf{absolute minimum}) at $(a, b)$.

\begin{theorem}
	If $f$ has a local maximum or minimum at $(a,b)$ and the first-order partial derivatives of $f$
	exist here, then $f_(a,b) = f_y(a,b) = 0$.
\end{theorem}

\begin{enumerate}
	\item Critical points: $\grad f (x,y) = 0$ or if one of the partial
	      derivatives doesn't exist.
\end{enumerate}

\begin{theorem}[Second derivative test]
	Suppose the second partial derivatives of \( f \) are continuous on a disk with center \( (a, b) \), and suppose that
	\( f_x(a, b) = 0 \) and \( f_y(a, b) = 0 \) [that is, \( (a, b) \) is a critical point of \( f \)]. Let
	\[
		D = D(a, b) = f_{xx}(a, b) f_{yy}(a, b) - [f_{xy}(a, b)]^2
	\]

	\begin{itemize}
		\item[(a)] If \( D > 0 \) and \( f_{xx}(a, b) > 0 \), then \( f(a, b) \) is a local minimum.
		\item[(b)] If \( D > 0 \) and \( f_{xx}(a, b) < 0 \), then \( f(a, b) \) is a local maximum.
		\item[(c)] If \( D < 0 \), then \( f(a, b) \) is not a local maximum or minimum.
	\end{itemize}
\end{theorem}
\section{Discussions}

\begin{problem}
Find the local maximum/minimum values and saddle points.

\begin{enumerate}
	\item $ f(x,y) = x^2 + xy + y^2 + y$
	      \vspace{7cm}
	\item $f(x,y) = e^x \cos y$
	      \vspace{7cm}
\end{enumerate}
\end{problem}


\begin{problem}
Show that $f(x,y) = x^2 + 4y^2 - 4xy +2$ has infinite number of critical points and that $D = 0$ at each one.
Then show that $f$ has a local (and absolute) minimum at each critical point.
\end{problem}

\end{document}
