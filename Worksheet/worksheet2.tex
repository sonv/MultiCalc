\documentclass[12pt]{amsart}
\usepackage{amsaddr}
\usepackage{marktext} 
%% Remove draft for real article, put twocolumn for two columns
\usepackage{svmacro}
\usepackage[utf8]{inputenc}
\usepackage{lineno}
\usepackage[style=alphabetic, backend=biber]{biblatex}
\addbibresource{bibliography.bib}

%% commentary bubble
\newcommand{\SV}[2][]{\sidenote[colback=green!10]{\textbf{SV\xspace #1:} #2}}

%% Title 
\title{ MATH 104: Worksheet 2}
\author{}

%\author{Co-author}
%\address{  }
%\email {  }
%
\date{\today}

\begin{document}

\maketitle

\section{Concepts}

\begin{enumerate}
	\item Length
	\item Dot products
\end{enumerate}


\begin{definition}
	The length of a vector $\vec{v}$ in $\R^n$ is
	\begin{equation*}
		\abs{ \vec{v}} = \sqrt{ v_1^2 + \dots + v_n^2 } \,.
	\end{equation*}
\end{definition}

\begin{definition}
	Scalar multiplication of a number $c$ and a vector $\vec{v}$ in $\R^n$ is
	\begin{equation*}
		c\vec{v}
		= \langle c v_1 , \dots,  c v_n  \rangle\,.
	\end{equation*}
\end{definition}

\begin{definition}
	Addition between two vectors $\vec{v}$ and $\vec{w}$ in $\R^n$ is
	\begin{equation*}
		\vec{v} + \vec{w} = \langle v_1 + w_1 , \dots,  v_n + w_n \rangle\,.
	\end{equation*}
\end{definition}

\begin{definition}
	The dot product between two vectors $\vec{v}$ and $\vec{w}$ in $\R^n$ is
	\begin{equation*}
		\vec{v} \cdot \vec{w} = v_1 w_1 + \dots + v_n w_n \,.
	\end{equation*}
\end{definition}


\begin{theorem}[Law of cosine]
	\begin{equation*}
		\vec{v} \cdot \vec{w} = \abs{\vec{v}} \abs{ \vec{w}} \cos\theta
	\end{equation*}
	where $\theta$ is the angle between the two vectors.
\end{theorem}

\begin{definition}[Projections]
	Given 2 vectors $\vec{a}$ and $\vec{b}$ in $\R^n$.

	Scalar projection of $\vec{b}$ to $\vec{a}$ is:
	\begin{equation*}
		\mathrm{comp}_{\vec{a}} \vec{b} = \frac{ \vec{a} \cdot \vec{b}}{\abs{\vec{a}}} \,.
	\end{equation*}

	Vector projection of $\vec{b}$ to $\vec{a}$ is:
	\begin{equation*}
		\mathrm{proj}_{\vec{a}} \vec{b} = \left(\frac{ \vec{a} \cdot \vec{b}}{\abs{\vec{a}}}\right) \frac{\vec{a}}{\abs{\vec{a}}} \,.
	\end{equation*}
\end{definition}

\section{Discussions}

\begin{question}
	Given two points $A(x_1,y_1, z_1)$ and $B(x_2, y_2, z_2)$.
	How do you construct a vector $\vec{AB}$ with the beginning at $A$ and the end at $B$?
\end{question}
\vspace{5cm}

\begin{question}
	What are the properties of vectors (Section 12.2)?

\end{question}
\vspace{5cm}

\begin{question}
	Prove the law of cosine.
\end{question}
\vspace{5cm}

\begin{question}
	What are the properties of dot product (Section 12.3)?

\end{question}
\vspace{5cm}

\begin{question}
	Intepret the equation of a plane going through the origin in $R^3$:
	ax + by + cz = 0 \,.
\end{question}
\vspace{5cm}

\begin{question}
	What is the value of $c$ so that the planes
	$2cx - y + c^2 = 15$ and $x + 5cy - 3z = 4$ are orthogonal?
\end{question}
\vspace{5cm}

\begin{question}
	What is the angle between the planes $x - 2y + 3z = 6$ and $2x + 3y -z = 11$?
\end{question}
\vspace{5cm}
%
%\begin{question}
%	What is the angle between the grand diagonal of a cube in $\R^n$
%	and an incident edge?
%
%	(Hint: try $\R^2$ and $\R^3$ to have intuition first.)
%\end{question}
%
%
%\begin{question}
%	What is the area of the triangle in the plane with vertices at
%	$(1,3), (-2,0), (5,2)$?
%\end{question}
%
%
%\begin{question}
%	What is the volume of the parallelopiped spanned by the vectors $\hat{i}, \hat{j}$ and $\vec{v}$.
%\end{question}
%
%\begin{question}
%	What is the projected length (component) of vector $\vec{w}$ onto the vector $\vec{v}$:
%	\begin{equation*}
%		\vec{w} = \begin{pmatrix}
%			5 \\ - 6 \\ 2 \\ -7
%		\end{pmatrix}
%		\,, \quad
%		\vec{v} =
%		\begin{pmatrix}
%			0 \\ 3 \\ 4 \\0
%		\end{pmatrix}
%		\,.
%	\end{equation*}
%\end{question}
%
\printbibliography
%\bibliography{refs}
%\bibliographystyle{halpha-abbrv}


\end{document}
