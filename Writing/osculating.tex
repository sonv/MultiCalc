\documentclass[12pt]{amsart}
\usepackage[draft,noshift]{marktext} 
%% Remove draft for real article, put twocolumn for two columns
\usepackage[draft]{svmacro}
\usepackage[utf8]{inputenc}
\usepackage[style=alphabetic, backend=biber]{biblatex}
\addbibresource{bibliography.bib}

%% commentary bubble
\newcommand{\SV}[2][]{\sidenote[colback=green!10]{\textbf{SV\xspace #1:} #2}}

%% Title 
\title{ On osculating circle and curvature in 2D }
%\author[1]{Co-author}
\author{Truong-Son Van}
%\affil[1]{Institute}
\address{Fulbright University Vietnam}
\email{son.van@fulbright.edu.vn}

\date{\today}

\begin{document}


\maketitle


\section{Introduction}
In this expository note, we derive the relationship between curvature and the radius of the osculating circle.
In particular, let $p$ be a point on a twice differentiable curve, $R$ be
the radius of the osculating circle at that point and $\kappa$ is the curvature of the curve 
at that point. Then,
\begin{equation*}
    R = \frac{1}{\kappa} \,.
\end{equation*}


From Wikipedia, the definition of an osculating circle is the following.

\begin{definition}
The osculating circle of a sufficiently smooth plane curve at a given point $p$
on the curve has been traditionally defined as the circle passing through $p$
and a pair of additional points on the curve infinitesimally close to $p$.
\end{definition}

\printbibliography 
%\bibliography{refs}
%\bibliographystyle{halpha-abbrv}


\end{document}
