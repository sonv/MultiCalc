\documentclass[12pt]{amsart}
\usepackage{amsaddr}
\usepackage{marktext} 
%% Remove draft for real article, put twocolumn for two columns
\usepackage{svmacro}
\usepackage[utf8]{inputenc}
\usepackage{lineno}
\usepackage[style=alphabetic, backend=biber]{biblatex}
\addbibresource{bibliography.bib}

%% commentary bubble
\newcommand{\SV}[2][]{\sidenote[colback=green!10]{\textbf{SV\xspace #1:} #2}}

%% Title 
\title{ MATH 104: Homework 2}
\author{Due date: In class -- Friday, Feb 2, 2024}
\address{Fulbright University, Ho Chi Minh City, Vietnam}

%\author{Co-author}
%\address{  }
%\email {  }
%
\date{\today}

\begin{document}

\maketitle

\begin{problem}
 Consider the following planes in $\mathbb{R}^3$, where $C$ is constant:
$$
\begin{gathered}
2 C x-3 y+(C+4) z=5 \\
(C+1) x+C y-z=1
\end{gathered}
$$
\begin{enumerate}
    \item  Find the value(s) of $C$ that makes these planes orthogonal.
    \item  Explain why if the planes are orthogonal at one intersection point they are orthogonal at all intersection points.
\end{enumerate}
\end{problem}

\begin{problem}
Consider the following hyperplanes in $\mathbb{R}^4$ :
$$
\begin{aligned}
& x_1-2 x_2+5 x_3+x_4=8 \\
& 3 x_1+6 x_2+x_3+4 x_4=17
\end{aligned}
$$

Identify vectors orthogonal to each hyperplane and use these to show that the hyperplanes are orthogonal.
\end{problem}

\begin{problem}
Consider the following planes in $\mathbb{R}^3$ :
$$
\begin{aligned}
& 2 x-3 y+z=5 \\
& 3 x+y-2 z=-1
\end{aligned}
$$

Identify vectors orthogonal to each plane and use these to compute a vector that is tangent to both planes.
\end{problem}


\begin{problem}
Consider the following vector in $\mathbb{R}^5$ :
$$
\vec{v}=\left(\begin{array}{c}
2 \\
-1 \\
3 \\
-1 \\
1
\end{array}\right)
$$
\begin{enumerate}
    \item Give an example of a nonzero vector that is orthogonal to $\vec{v}$.
    \item  Compute the angle between the vector $\vec{v}$ and the basis vector $\hat{e}_1$.
\end{enumerate}
\end{problem}

\begin{problem}
Consider the following vectors in $\mathbb{R}^4$ :
$$
\vec{u}=\left(\begin{array}{c}
2 \\
1 \\
-2 \\
0
\end{array}\right), 
\vec{v}=\left(\begin{array}{c}
2 \\
1 \\
2 \\
-4
\end{array}\right)
$$
\begin{enumerate}
    \item  Compute the angle between these two vectors..
    \item Find a vector that is orthogonal to both $u$ and $v$.
\end{enumerate}
\end{problem}

\begin{problem}
 Consider the following vectors in $\mathbb{R}^4$ :
$$
\vec{u}=\left(\begin{array}{c}
4 \\
-5 \\
2 \\
-2
\end{array}\right), 
\vec{v}=\left(\begin{array}{l}
3 \\
0 \\
4 \\
0
\end{array}\right)
$$
\begin{enumerate}
    \item Compute the angle between these two vectors.
    \item Compute the projected length of $\vec{u}$ onto the " $v$-axis".
\end{enumerate}
\end{problem}

\begin{problem}
Consider the following four vectors in $\mathbb{R}^3$ :
$$
\vec{a}=\left(\begin{array}{c}
1 \\
0 \\
-3
\end{array}\right), 
\vec{b}=\left(\begin{array}{l}
0 \\
2 \\
5
\end{array}\right), 
\vec{c}=\left(\begin{array}{l}
1 \\
3 \\
0
\end{array}\right), 
\vec{d}=\left(\begin{array}{l}
4 \\
1 \\
0
\end{array}\right)
$$
\begin{enumerate}
    \item Is there a pair of orthogonal vectors among the above? Explain.
    \item Which three of the vectors above span a parallelopiped with the largest volume?
\end{enumerate}
\end{problem}


\begin{problem}
Consider the following two curves in $\mathbb{R}^3$ :
$$
\gamma_1(s)=\left(\begin{array}{c}
s^2-3 s \\
e^s-1 \\
1-\cos 2 s
\end{array}\right), 
\quad \gamma_2(t)=\left(\begin{array}{c}
\sin (t-1) \\
t^2+t-2 \\
1-\sqrt{t}
\end{array}\right)
$$
\begin{enumerate}
    \item  Verify that these curves intersect at the origin for some values of $s$ and $t$.
    \item  At what angle do these curves intersect at the origin?
    \item  Find a vector that is orthogonal to both curves at the origin.
\end{enumerate}
\end{problem}


\begin{problem}
Consider the parametrized curve in 3-D given by
$$
\vec{\gamma}(t)=\begin{pmatrix}
t^2-t+4 \\
t^3-3 t^2+2 t-1 \\
2 t
\end{pmatrix} \begin{gathered}
\leftarrow x \\
\leftarrow y \\
\leftarrow z
\end{gathered}
$$
\begin{enumerate}
    \item  Compute the velocity vector of this curve.
    \item Write down the equation of a plane orthogonal to this curve at $\vec{\gamma}(0)$.
    \item At what angle does this curve cross the $(x, y)$ plane $z=0$ ? Explain your reasoning and give your answer as best you can without a calculator.
\end{enumerate}
\end{problem}

\begin{problem}
Consider the following matrix and vectors:
$$
A=\left[\begin{array}{cccc}
1 & -1 & 2 & 2 \\
-1 & 3 & 0 & 1 \\
2 & 1 & -3 & 1 \\
4 & 0 & 1 & 0
\end{array}\right], 
\vec{u}=\left(\begin{array}{c}
1 \\
0 \\
-2 \\
1
\end{array}\right), 
\vec{v}=\left(\begin{array}{l}
1 \\
1 \\
1 \\
2
\end{array}\right), 
\vec{w}=\left(\begin{array}{c}
-2 \\
-1 \\
1 \\
-3
\end{array}\right)
$$
\begin{enumerate}
    \item Evaluate the dot product $\vec{v} \cdot A \vec{u}$ if possible. If not, explain why not.
    \item  Compute $A \vec{u} + A \vec{v}+A \vec{w}$.
    \item Compute the quantity $\left(\vec{v} \vec{w}^T\right) \vec{u}$ if it exists; if not, explain why not.
\end{enumerate}
\end{problem}


\begin{problem}
Consider the following matrices / vectors:
$$
A=\left[\begin{array}{cc}
3 & 6 \\
-2 & 5 \\
7 & -1
\end{array}\right], B=\left[\begin{array}{ccc}
4 & 1 & 3 \\
2 & -6 & 0
\end{array}\right], \vec{x}=\left(\begin{array}{c}
2 \\
-1
\end{array}\right), \vec{y}=\left(\begin{array}{c}
3 \\
0 \\
-2
\end{array}\right)
$$

Compute the following products, if possible: if not, explain why not.
\begin{enumerate}
    \item  $A B$
    \item $B \vec{y}$
    \item  $\vec{x} B$
    \item  $\vec{x}^T B \vec{y}$
    \item  $A^T$
\end{enumerate}
\end{problem}

\begin{problem}
Consider the following row-reduction:
$$
A=\left[\begin{array}{cccc}
1 & -2 & 1 & 0 \\
3 & -4 & 8 & 3 \\
1 & 0 & 7 & 4 \\
0 & 2 & 5 & 6
\end{array}\right] \Rightarrow\left[\begin{array}{cccc}
1 & -2 & 1 & 0 \\
0 & 2 & 5 & 3 \\
0 & 0 & 1 & 1 \\
0 & 0 & 0 & 3
\end{array}\right]=B
$$

\begin{enumerate}
    \item  Write out the steps of the row-reduction, identifying each row operation.

    \item  Solve the system of equations given by
$$
B x=\left[\begin{array}{cccc}
1 & -2 & 1 & 0 \\
0 & 2 & 5 & 3 \\
0 & 0 & 1 & 1 \\
0 & 0 & 0 & 3
\end{array}\right]\left(\begin{array}{c}
x \\
y \\
z \\
w
\end{array}\right)=\left(\begin{array}{c}
-2 \\
1 \\
-1 \\
6
\end{array}\right)
$$
\end{enumerate}
\end{problem}


\begin{problem}
Consider the following system of linear equations
$$
\begin{aligned}
& x-2 y+3 u-v=-10 \\
& 2 x-7 y-4 u+v=16 \\
& 3 u-2 v=-13 \\
& 6 u+3 v=-12
\end{aligned}
$$
\begin{enumerate}
    \item  Rewrite this as a linear system of the form $A \vec{x}=\vec{b}$, specifying $A, \vec{x}$, and $\vec{b}$ carefully.
    \item  Row-reduce the augmented matrix of this system to lower-triangular form.
    \item  Solve the original equations for the unknowns using your answer to part (2) and back-substitution.
\end{enumerate}
\end{problem}


\end{document}
