\documentclass[aspectratio=169]{beamer}
\usetheme{Copenhagen}
%% Remove draft for real article, put twocolumn for two columns
\usetheme{metropolis}
\usepackage{multicol}
\usepackage[style=british]{csquotes}

\def\signed #1{{\leavevmode\unskip\nobreak\hfil\penalty50\hskip1em
  \hbox{}\nobreak\hfill #1%
  \parfillskip=0pt \finalhyphendemerits=0 \endgraf}}

\newsavebox\mybox
\newenvironment{aquote}[1]
  {\savebox\mybox{#1}\begin{quote}\openautoquote\hspace*{-.7ex}}
  {\unskip\closeautoquote\vspace*{1mm}\signed{\usebox\mybox}\end{quote}}

\usepackage[utf8]{inputenc}
\newtheorem*{question}{Question}

\newcommand{\vectorproj}[2][]{\mathrm{proj}_{\vect{#1}}\vect{#2}}
\newcommand{\vectorcomp}[2][]{\mathrm{comp}_{\vect{#1}}\vect{#2}}
\newcommand{\vect}{\mathbf}
\newcommand{\R}{\mathbb{R}}
%% commentary bubble
\newcommand{\SV}[2][]{\sidenote[colback=green!10]{\textbf{SV\xspace #1:} #2}}

\title{ Multivariable Calculus \\ Day  20 \\ Vector Calculus: Line integrals}
\date{Spring 2023}

\begin{document}

\maketitle

\begin{frame}
    \frametitle{ Warning}
    Don't be confused with arc length!!!
\end{frame}

\begin{frame}
    \frametitle{Line integrals}
    \begin{definition}
        Let $D$ be a domain on $\R^n$.
        A vector field on $\R^n$ is a function $\vect{F}: D \to \R^n$
        that assign each point $\vect{x}\in D$ to a vector $\vect{F}(\vect{x}) \in \R^n$.
    \end{definition}
    We only think about vector fields on $\R^2$ and $\R^3$.
\end{frame}


\end{document}

