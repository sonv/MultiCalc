\documentclass[aspectratio=169]{beamer}
\usetheme{Copenhagen}
%% Remove draft for real article, put twocolumn for two columns
\usetheme{metropolis}

\usepackage[utf8]{inputenc}
\newtheorem*{question}{Question}

\newcommand{\vectorproj}[2][]{\mathrm{proj}_{\vect{#1}}\vect{#2}}
\newcommand{\vectorcomp}[2][]{\mathrm{comp}_{\vect{#1}}\vect{#2}}
\newcommand{\vect}{\mathbf}
%% commentary bubble
\newcommand{\SV}[2][]{\sidenote[colback=green!10]{\textbf{SV\xspace #1:} #2}}

%% Title 
\title{ Multivariable Calculus \\ Day 2 \\ Linear Algebra (cont.)}
\institute{Fulbright University Vietnam}
%\author[1]{Co-author}
\author{Truong-Son Van}
\date{Spring 2023}

\begin{document}

\maketitle

\section{Dot product}
\begin{frame}
    \frametitle{\secname}
    \begin{definition}
The dot product of vectors $\textbf{u} = \langle u_1, \dots, u_n \rangle$
and $\textbf{v} = \langle v_1, \dots, v_n \rangle$ in $\mathbb{R}^n$ is the
scalar
\begin{equation*}
    \textbf{u} \cdot \textbf{v} = u_1 v_1 +\dots + u_n v_n \,.
\end{equation*}
    \end{definition}
\end{frame}

\begin{frame}
    \frametitle{\secname: Geometric meaning}
    \begin{theorem}
If $\theta$ is the angle between the vectors $\textbf{u}$ and $\textbf{v}$, then
   \begin{equation*}
        \textbf{u}\cdot \textbf{v} = |\textbf{u}|| \textbf{v}| \cos \theta \,.
   \end{equation*}
    \end{theorem}
    \pause

    \begin{proof}[Proof in $\mathbb{R}^2$]
        Homework
    \end{proof}

    \pause
    \begin{corollary}
Two vectors $\textbf{u}$ and $\textbf{v}$ are orthogonal to each other
if $\textbf{u} \cdot \textbf{v} = 0$.
    \end{corollary}
\end{frame}

\begin{frame}
    \frametitle{\secname: Projection}
Let $\textbf{u}, \textbf{v}\in \mathbb{R}^n$. The component of $\textbf{u}$ 
in the direction of $\textbf{v}$ is the scalar
\begin{equation*}
\vectorcomp[v]{u} =  
\end{equation*}
and the projection of $\vect{u}$ onto $\vect{v}$ is the vector
\begin{equation*}
    \vectorproj[v]{u}  =
\end{equation*}
\end{frame}

\begin{frame}
    \frametitle{Example}

\end{frame}

\begin{frame}
    \frametitle{Worksheet}
    \begin{enumerate}
        \item Let $\vect{a} = \langle 3,0,-1\rangle$. Find a vector
            $\vect{b}$ such that $\vectorcomp[a]{b} = 2$.
        \item
         Suppose $\vect{a}$ and $\vect{b}$ are nonzero vectors.
            When would it be true that $\vectorcomp[a]{b} = \vectorcomp[b]{a}$?
        \item When would it be true that 
                 $\vectorproj[a]{b} = \vectorcomp[b]{a}$?
    \end{enumerate}
\end{frame}

\section{3D special: Cross product}

\begin{frame}
    \frametitle{\secname}
    \begin{block}{WARNING}
This concept is very specific to \(\mathbb{R}^3\).
It will not make sense in other dimensions.
    \end{block}
\end{frame}

\begin{frame}
    \frametitle{\secname}
\begin{definition}
Let \(\mathbf{u}, \mathbf{v} \in \mathbb{R}^3\).
The cross product of \(\mathbf{a}\) and \(\mathbf{b}\) is defined to be
\begin{equation*}
    \mathbf{u} \times \mathbf{v} = \langle u_2 v_3 - u_3 v_2, u_3v_1 - u_1 v_3, u_1v_2 - u_2v_1 \rangle \,.
\end{equation*}
\end{definition}
\end{frame}
  
\begin{frame}
    \frametitle{Worksheet}
    Given a  $3\times 3$ matrix
    \begin{equation*}
        A =
        \begin{pmatrix}
            a_{11} & a_{12} & a_{13} \\
            a_{21} & a_{22} & a_{23} \\
            a_{31} & a_{32} & a_{33} 
        \end{pmatrix}\,,
    \end{equation*}
    the determinant of the matrix $A$ is
    \begin{equation*}
        \det(A) = 
        \begin{vmatrix}
            a_{11} & a_{12} & a_{13} \\
            a_{21} & a_{22} & a_{23} \\
            a_{31} & a_{32} & a_{33} 
        \end{vmatrix}
        = 
    \end{equation*}
    \begin{question}
        Relate the cross product of vectors $\vect{u}$ and $\vect{v}$ with the 
        determinant of a $3\times 3$ matrix.
    \end{question}
\end{frame}

\begin{frame}
    \frametitle{Worksheet}
    Suppose $\vect{u} = \langle 0,1,3\rangle, \vect{v} = \langle 2,-1,0\rangle$.
    \begin{enumerate}
        \item Find the cross products $\vect{u} \times \vect{v}$
        \item Find the cross products $\vect{v} \times \vect{u}$
        \item Compute  $\vect{u} \cdot (\vect{u} \times \vect{v})$
            and $\vect{v} \cdot \vect{u} \times \vect{v}$.
        \item Compute $\vect{u} \times \vect{u}$.
    \end{enumerate}
\end{frame}

\begin{frame}
    \frametitle{\secname}
Let \(\mathbf{u}, \mathbf{v}, \mathbf{w}\) be vectors in \(\mathbb{R}^3\) and
\(c\in \mathbb{R}\). Then

\begin{enumerate}
\def\labelenumi{\arabic{enumi}.}
\item
  (anti-symmetry) \(\mathbf{u}\times \mathbf{v} = - \mathbf{v}\times \mathbf{u}\)
\item
  \((\mathbf{u} + \mathbf{v})\times \mathbf{w} = (\mathbf{u}\times \mathbf{w}) + (\mathbf{v}\times \mathbf{w})\)
\item
  \((c \mathbf{u})\times \mathbf{w} = c(\mathbf{u}\times \mathbf{w}) = \mathbf{u}\times (c\mathbf{w})\)
\item
  \(\mathbf{u}\times \mathbf{v} = \mathbf{0}\) if \(\mathbf{u}\) and \(\mathbf{v}\) are
  parallel
\item
  \textbf{WARNING:} in general
  \((\mathbf{u}\times \mathbf{v})\times \mathbf{w} \not = \mathbf{u}\times (\mathbf{v}\times \mathbf{w})\)
\end{enumerate}

\end{frame}


\begin{frame}
    \frametitle{\secname}
\begin{theorem}
Let \(\theta\) be the angle between \(\mathbf{a}\) and \(\mathbf{b}\). Then,
\begin{equation*}
    | \mathbf{a} \times \mathbf{b} | = |\mathbf{a}||\mathbf{b}| \sin\theta \,.
\end{equation*}
\end{theorem}


\begin{theorem}
The vector \(\mathbf{a}\times \mathbf{b}\) is orthogonal to both \(\mathbf{a}\) and \(\mathbf{b}\).
\end{theorem}

\end{frame}


\begin{frame}
    \frametitle{\secname}

\begin{theorem}
The length, \(|\mathbf{u}\times\mathbf{v}|\), of the cross product of
vectors \(\mathbf{u}\) and \(\mathbf{v}\) is the area of the parallelogram
determined by \(\mathbf{u}\) and \(\mathbf{v}\).
\end{theorem}

\begin{example}
    Find the area of the parallelogram created by two vectors
    $ \langle 1,3,-2 \rangle$ and $\langle 3,0,1 \rangle$.
\end{example}

\end{frame}

\begin{frame}
    \frametitle{Worksheet}
    \begin{enumerate}
        \item Prove the above theorem.
        \item Find the area of the parallelogram in $\mathbb{R}^3$ whose
            vertices are $(1,0,1), (0,0,1), (2,1,0), (1,1,0)$.
            Hint: maybe draw it out.
    \end{enumerate}
\end{frame}


\section{Equations for lines and planes}

\begin{frame}
    \frametitle{Equation for a line}
    High school version:

    \vspace{2cm}
    \pause
    Grown-up version:
\end{frame}



\begin{frame}
    \frametitle{Equation for plane}
\end{frame}






\end{document}
