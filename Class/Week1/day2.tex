\documentclass[aspectratio=169]{beamer}
\usetheme{Copenhagen}
%% Remove draft for real article, put twocolumn for two columns
\usepackage[utf8]{inputenc}

\newcommand{\vectorproj}[2][]{\mathrm{proj}_{\vect{#1}}\vect{#2}}
\newcommand{\vectorcomp}[2][]{\mathrm{comp}_{\vect{#1}}\vect{#2}}
\newcommand{\vect}{\mathbf}
%% commentary bubble
\newcommand{\SV}[2][]{\sidenote[colback=green!10]{\textbf{SV\xspace #1:} #2}}

%% Title 
\title{ Multivariable Calculus \\ Day 2}
\institute{Fulbright University Vietnam}
%\author[1]{Co-author}
\author{Truong-Son Van}
\date{Spring 2023}

\begin{document}

\maketitle
\begin{frame}
    \frametitle{Dot product}
    \begin{definition}
The dot product of vectors $\textbf{u} = \langle u_1, \dots, u_n \rangle$
and $\textbf{v} = \langle v_1, \dots, v_n \rangle$ in $\mathbb{R}^n$ is the
scalar
\begin{equation*}
    \textbf{u} \cdot \textbf{v} = u_1 v_1 +\dots + u_n v_n \,.
\end{equation*}
    \end{definition}
\end{frame}

\begin{frame}
    \frametitle{Geometric meaning}
    \begin{theorem}
If $\theta$ is the angle between the vectors $\textbf{u}$ and $\textbf{v}$, then
   \begin{equation*}
        \textbf{u}\cdot \textbf{v} = |\textbf{u}|| \textbf{v}| \cos \theta \,.
   \end{equation*}
    \end{theorem}
    \pause

    \begin{proof}[Proof in $\mathbb{R}^2$]
        Homework
    \end{proof}

    \pause
    \begin{corollary}
Two vectors $\textbf{u}$ and $\textbf{v}$ are orthogonal to each other
if $\textbf{u} \cdot \textbf{v} = 0$.
    \end{corollary}
\end{frame}

\begin{frame}
    \frametitle{Projection}
Let $\textbf{u}, \textbf{v}\in \mathbb{R}^n$. The component of $\textbf{u}$ 
in the direction of $\textbf{v}$ is the scalar
\begin{equation*}
\vectorcomp[v]{u} =  
\end{equation*}
and the projection of $\vect{u}$ onto $\vect{v}$ is the vector
\begin{equation*}
    \vectorproj[v]{u}  =
\end{equation*}
\end{frame}



\end{document}
