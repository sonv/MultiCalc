\documentclass[aspectratio=169]{beamer}
\usetheme{Copenhagen}
%% Remove draft for real article, put twocolumn for two columns
\usetheme{metropolis}
\usepackage{multicol}
\usepackage[style=british]{csquotes}

\def\signed #1{{\leavevmode\unskip\nobreak\hfil\penalty50\hskip1em
  \hbox{}\nobreak\hfill #1%
  \parfillskip=0pt \finalhyphendemerits=0 \endgraf}}

\newsavebox\mybox
\newenvironment{aquote}[1]
  {\savebox\mybox{#1}\begin{quote}\openautoquote\hspace*{-.7ex}}
  {\unskip\closeautoquote\vspace*{1mm}\signed{\usebox\mybox}\end{quote}}

\usepackage[utf8]{inputenc}
\newtheorem*{question}{Question}

\newcommand{\vectorproj}[2][]{\mathrm{proj}_{\vect{#1}}\vect{#2}}
\newcommand{\vectorcomp}[2][]{\mathrm{comp}_{\vect{#1}}\vect{#2}}
\newcommand{\vect}{\mathbf}
\newcommand{\R}{\mathbb{R}}
%% commentary bubble
\newcommand{\SV}[2][]{\sidenote[colback=green!10]{\textbf{SV\xspace #1:} #2}}

\title{ Multivariable Calculus \\ Day  14\\ Optimization (cont.) }
\date{Spring 2023}

\begin{document}

\maketitle


\begin{frame}
    \frametitle{Basics}
\begin{definition}
A function \(f:D \to \mathbb{R}\) has a \textbf{local maximum} at \(\mathbf{x_0}\) if
\(f(\mathbf{x_0}) \geq f(\mathbf{x})\) for \(\mathbf{x} \in B_\delta(\mathbf{x_0})\) for small enough \(\delta\).
\(f\) has a \textbf{global maximum} at \(\mathbf{x_0}\) if
\(f(\mathbf{x_0}) \geq f(\mathbf{x})\) for \(\mathbf{x} \in D\).
\(f\) has a \textbf{local (global) minimum} at \(\mathbf{x_0}\) if
\(-f\) has a local (global) maximum at \(\mathbf{x_0}\)
\end{definition}

\end{frame}


\begin{frame}
    \frametitle{A necessary condition}
\begin{theorem}[First derivative test]
Let \(f:D \to \mathbb{R}\) be a function.
If \(\mathbf{x_0}\) is a local minimum and \(f\) has partial derivatives at \(\mathbf{x_0}\).
Then
\begin{equation*}
    \partial_{x_i} f(\mathbf{x}_0) = 0 \,.
\end{equation*}
\end{theorem}
\end{frame}

\begin{frame}
    \begin{definition}
\(\mathbf{x}_0\) is said to be a \textbf{critical point} of \(f:D\to \mathbb{R}\) if
\begin{equation*}
    \nabla f(\mathbf{x}_0) = 0
\end{equation*}
or one of the partial derivatives \(\partial_{x_i} f(\mathbf{x}_0)\) fails to exist.
    \end{definition}
\pause

\textbf{Worksheet:}
    Find critical points of the following functions
    \begin{enumerate}
        \item $f(x,y) = |x| + |y|$
        \item $f(x,y) = 2xy -4x + 2y - 3$
        \item $f(x,y) = x^2 + y^2 - qxy$ where $q\in \R$ is a given constant
    \end{enumerate}
\end{frame}

\begin{frame}
    \frametitle{Second derivative test}
Suppose the second partial derivatives of \(f\) are continuous near \((a,b)\)
and suppose that \((a,b)\) is a critical point of \(f\).
Let
\begin{equation*}
    D = f_{xx}(a,b) f_{yy}(a,b) - f_{xy}(a,b)^2\,.
\end{equation*}

\begin{enumerate}
\def\labelenumi{\arabic{enumi}.}
\item
  If \(D>0\) and \(f_{xx}(a,b) >0\), then \(f(a,b)\) is a local minimum.
\item
  If \(D>0\) and \(f_{xx}(a,b) <0\), then \(f(a,b)\) is a local maximum.
\item
  If \(D<0\), then \(f(a,b)\) is neither a local maximum nor local minimum.
\item
  If \(D=0\), then we cannot conclude.
\end{enumerate}
$D$ is called the discriminant of the function $f$ at $(a,b)$.
\end{frame}

\begin{frame}
    \frametitle{Examples}
    Find the critical points of the following functions and use 
    the second derivative test to classify them
    \begin{enumerate}
        \item $f(x,y) = 3x^2 + y^2 - 9x +4y$
        \item $f(x,y) = xy + \frac{2}{x} + \frac{4}{y}$
    \end{enumerate}
\end{frame}


\begin{frame}
\begin{theorem}[Extreme value theorem]
If \(f\) is continuous on a \emph{closed} and \emph{bounded} set \(D\). Then,
\(f\) attains an absolute minimum and an absolute maximum in \(D\).
\end{theorem}
\pause

\textbf{Algorithm to find absolute maxima and minima on closed bounded regions}

\begin{enumerate}
\def\labelenumi{\arabic{enumi}.}
\item
  Find the values of \(f\) at the critical points of \(f\) in \(D\).
\item
  Find the extreme values of \(f\) on the boundary of \(D\).
\item
  The largest of the values from steps 1 and 2 is the absolute maximum value;
  the smallest of these values is the absolute minimum value.
\end{enumerate}

\end{frame}

\begin{frame}
    \frametitle{Worksheet}
    Suppose the temperature $T$ at each point on a disk of radius 1 is given by
    \begin{equation*}
        T(x,y) = 2x^2 + y^2 - y \,.
    \end{equation*}
    What would be the hottest and coldest point on this disk?
\end{frame}


\begin{frame}
    \frametitle{Constrained optimization}
Constrained optimization takes various forms, depending on the assumptions.
We will deal with the most straight forward form.
The problem we will study is the following:

Maximize/minimize a function \(f:D\to \mathbb{R}\), subject to a constraint (side condition)
of the form
\(g(\mathbf{x}) = k\), for some constant \(k\in \mathbb{R}\).


Typically, people will write as follows
\begin{equation*}
\begin{aligned}
& \underset{x}{\text{minimize}}
& & f(x) \\
& \text{subject to}
& & 
g(x) = k \,.
\end{aligned}
\end{equation*}
\end{frame}

\begin{frame}
    \frametitle{Constrained optimization}
    
\begin{theorem}[Method of Lagrange Multiplier]

Suppose the maximum/minimum values of \(f\) exist and \(\nabla g(\mathbf{x}) \not=0\) where \(g(\mathbf{x}) = k\).
To find the maximum and minimum values of \(f\) subject to constraint
\(g(\mathbf{x}) = k\), we do the following:

\begin{enumerate}
\def\labelenumi{\arabic{enumi}.}
\item
  Find all values of \(\mathbf{x}\) and \(\lambda \in \mathbb{R}\) such that
  \begin{equation*}
   \nabla f(\mathbf{x}) =\lambda \nabla g(\mathbf{x})\,,
  \end{equation*}
  and
  \begin{equation*}
   g(\mathbf{x}) = k \,.
  \end{equation*}
\item
  Evaluate \(f\) at all the points \(\mathbf{x}\) that result from step 1. The largest of
  these values is the maximum of \(f\); the smallest is the minimum value of \(f\).
\end{enumerate}

\end{theorem}
\end{frame}

\begin{frame}
    \frametitle{Example}
    \url{https://youtu.be/hQ4UNu1P2kw}
\end{frame}

\end{document}

