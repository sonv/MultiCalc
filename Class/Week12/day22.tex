\documentclass[aspectratio=169]{beamer}
\usetheme{Copenhagen}
%% Remove draft for real article, put twocolumn for two columns
\usetheme{metropolis}
\usepackage{multicol}
\usepackage[style=british]{csquotes}

\def\signed #1{{\leavevmode\unskip\nobreak\hfil\penalty50\hskip1em
  \hbox{}\nobreak\hfill #1%
  \parfillskip=0pt \finalhyphendemerits=0 \endgraf}}

\newsavebox\mybox
\newenvironment{aquote}[1]
  {\savebox\mybox{#1}\begin{quote}\openautoquote\hspace*{-.7ex}}
  {\unskip\closeautoquote\vspace*{1mm}\signed{\usebox\mybox}\end{quote}}

\usepackage[utf8]{inputenc}
\newtheorem*{question}{Question}
\newtheorem*{proposition}{Proposition}

\newcommand{\vectorproj}[2][]{\mathrm{proj}_{\vect{#1}}\vect{#2}}
\newcommand{\vectorcomp}[2][]{\mathrm{comp}_{\vect{#1}}\vect{#2}}
\newcommand{\vect}{\mathbf}
\newcommand{\R}{\mathbb{R}}
%% commentary bubble
\newcommand{\SV}[2][]{\sidenote[colback=green!10]{\textbf{SV\xspace #1:} #2}}

\title{ Multivariable Calculus \\ Day  22 \\ Vector Calculus: Line integrals (cont.)}
\date{Spring 2023}

\begin{document}

\maketitle


\begin{frame}
    \frametitle{Recap: Worksheet}
    Work done by a force field along a path is 
    \begin{equation*}
        W = \int_C  \vect{F} \cdot d\vect{r} \,.
    \end{equation*}
    A force field is given $\vect{F}(x,y) = x^2 \vect{i} - xy \vect{j}$.
    Suppose we want to move a particle along the quarter circle
    $\vect{r}(t) = \cos t \vect{i} + \sin t \vect{j}, 0 \leq t \leq \pi/2$.
    Compute the work done.
\end{frame}



\begin{frame}
    \frametitle{Fundamental Theorem for line integrals}
    \begin{theorem}[Fundamental Theorem for line integrals]
    Let \(C\) be a smooth curve given by the parametrization \(\mathbf{r}(t)\),
    \(a \leq t \leq b\).
    Let \(f\) be a differentiable function of two or three
    variables whose gradient vector \(\nabla f\) is continuous on \(C\).
    Then,
    \begin{equation*}
        \int_C \nabla f \cdot d\mathbf{r} = f(\mathbf{r}(b)) - f(\mathbf{r}(a)) \,.
    \end{equation*}
    \end{theorem}

\end{frame}

\begin{frame}
    \frametitle{Worksheet}
    Prove the Fundamental Theorem of line integrals.
\end{frame}

\begin{frame}
    \frametitle{Worksheet}
    \begin{enumerate}
        \item Find the work done by the gravitational field in $\R^3$
            \begin{equation*}
                \vect{F}(\vect{x}) = - \frac{-mMG}{|\vect{x}|^3} \vect{x}
            \end{equation*}
            in moving a particle with mass $m$ from
            $(3,4,12) \to (2,2,0)$ along a straight line.
    \end{enumerate}
\end{frame}

\begin{frame}
    \frametitle{Closed curves}
    \begin{definition}
    A \textbf{closed curve} is a curve that starts and ends at the same point.

    A \textbf{simple closed curve} is a closed curve that never crosses itself.
    \end{definition}
\end{frame}

\begin{frame}
\begin{corollary}
If \(C\) is a closed curve and \(f:\mathbb{R}^n \to \mathbb{R}\) is a smooth function,
then
\begin{equation*}
    \oint_C \nabla f \cdot d\mathbf{r}  = 0 \,.
\end{equation*}
\end{corollary}
\end{frame}

\begin{frame}
    \frametitle{Conservative vector fields}
\begin{definition}
A vector field \(\mathbf{F}\) is called a \textbf{conservative vector field}
if it is the gradient of some scalar function, that is there exists
a function \(f\) such that
\begin{equation*}
    \nabla f = \mathbf{F} \,.
\end{equation*}
\end{definition}
\end{frame}

\begin{frame}
    \frametitle{Worksheet}
         True or False? 
If \(\mathbf{F}\) is a conservative vector field, then
\begin{equation*}
    \oint_C \mathbf{F} \cdot d\mathbf{r} = 0 \,.
\end{equation*}

\end{frame}

\section{Independence of path}

\begin{frame}
    \frametitle{Worksheet}
    Evaluate
    \begin{equation*}
        \int_{C} y^2 \, dx + x\, dy \,, \qquad i = 1,2
    \end{equation*}
    \begin{enumerate}
        \item where \(C\) is the line segment from \((-5,-3) \to (0,2)\) 
        \item where \(C\) is the arc of the parabola \(x = 4-y^2\) from \((-5,-3) \to (0,2)\).
        \item repeat the above two steps with 
            \begin{equation*}
                P = x\,, \qquad Q = y \,.
            \end{equation*}
    \end{enumerate}
\end{frame}


\begin{frame}
\begin{definition}
Let \(\mathbf{F}\) be a continuous vector field with domain \(D\), we say that the
line integral
\begin{equation*}
    \int_C \mathbf{F} \cdot d\mathbf{r} 
\end{equation*}
is \textbf{independent of path} if
\begin{equation*}
    \int_{C_1} \mathbf{F}\cdot d\mathbf{r} 
    =
    \int_{C_2} \mathbf{F} \cdot d\mathbf{r} 
\end{equation*}
\end{definition}
\end{frame}


\begin{frame}
    \frametitle{Independence of path and conservative vector fields}
    \begin{theorem}
    \(\int_C \mathbf{F}\cdot d\mathbf{r}\) is independent of path in \(D\) if and only if
    \(\int_C \mathbf{F} \cdot d\mathbf{r} = 0\) for every closed path \(C\) in \(D\).
    \end{theorem}
\end{frame}

\begin{frame}
    \frametitle{Worksheet}
    Prove the above theorem.
\end{frame}


\end{document}

