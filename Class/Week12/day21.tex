\documentclass[aspectratio=169]{beamer}
\usetheme{Copenhagen}
%% Remove draft for real article, put twocolumn for two columns
\usetheme{metropolis}
\usepackage{multicol}
\usepackage[style=british]{csquotes}

\def\signed #1{{\leavevmode\unskip\nobreak\hfil\penalty50\hskip1em
  \hbox{}\nobreak\hfill #1%
  \parfillskip=0pt \finalhyphendemerits=0 \endgraf}}

\newsavebox\mybox
\newenvironment{aquote}[1]
  {\savebox\mybox{#1}\begin{quote}\openautoquote\hspace*{-.7ex}}
  {\unskip\closeautoquote\vspace*{1mm}\signed{\usebox\mybox}\end{quote}}

\usepackage[utf8]{inputenc}
\newtheorem*{question}{Question}
\newtheorem*{proposition}{Proposition}

\newcommand{\vectorproj}[2][]{\mathrm{proj}_{\vect{#1}}\vect{#2}}
\newcommand{\vectorcomp}[2][]{\mathrm{comp}_{\vect{#1}}\vect{#2}}
\newcommand{\vect}{\mathbf}
\newcommand{\R}{\mathbb{R}}
%% commentary bubble
\newcommand{\SV}[2][]{\sidenote[colback=green!10]{\textbf{SV\xspace #1:} #2}}

\title{ Multivariable Calculus \\ Day  21 \\ Vector Calculus: Line integrals (cont.)}
\date{Spring 2023}

\begin{document}

\maketitle


\begin{frame}
    \frametitle{Line integrals}
    We now perform a Riemann-sum-like action.
    \begin{definition}
    Let $C$ be a smooth curve.
    The \textbf{line integral of $f$ along $C$} is defined as
    \begin{equation*}
        \int_C f(x,y) \, ds = \lim_{n\to \infty} \sum_{i=1}^n f(x_i^*, y_i^*) \Delta s_i \,,
    \end{equation*}
    where $\Delta s_i$ is the length of a subarc of $C$.
    \end{definition}
\end{frame}

\begin{frame}
    \frametitle{Formula}
    \begin{equation*}
        \int_C f(x,y) \, ds = \int_a^b f(\mathbf{r}(t)) |\mathbf{r}'(t)| \, dt \,.
    \end{equation*}

    Note: everything you do for 2 dimension will generalize directly to $n$ dimension. 
    So, we will not cover the details for 3 or higher dimensions. However, questions
    about those will be fair game. 
\end{frame}

\begin{frame}
    \frametitle{Worksheet}
    Let's relate a few things with each other. 
    \begin{enumerate}
        \item Look up Wikipedia or recall from your memory the way to compute the
            expectation of a random variable.
        \item Watch the videos about center of mass
            \begin{enumerate}
                \item \url{https://youtu.be/ol1COj0LACs},
                \item \url{https://youtu.be/e548hRYcXlg}
            \end{enumerate}
        \item What is the relationship between the center of mass and the expectation?
        \item What is the analog of the density function in the calculation of the center of mass?
    \end{enumerate}
\end{frame}

\begin{frame}
    \frametitle{Worksheet}
    A wire takes the shape of the semicircle $x^2 + y^2 = 1, y>1$ and is thicker near its base than near the top. Find the center of mass of the wire if the linear density at any point is proportional to its distance from the line $y = 1$.
\end{frame}




\begin{frame}
    \frametitle{Line integrals for vector fields}
    \begin{definition}
    Let \(\mathbf{F}\) be a continuous vector field defined
    on a curve \(C\).
    Then the \textbf{line integral of \(\mathbf{F}\) along \(C\)} is defined as
    \begin{equation*}
        \int_C \mathbf{F} \cdot d \mathbf{r} 
        = \int_C \mathbf{F}\cdot \mathbf{T} \, ds \,,
    \end{equation*}
    where \(\mathbf{T}\) is the unit tangent vector.
    \end{definition}
\end{frame}


\begin{frame}
\begin{proposition}
Suppose \(C\) is smooth and
parametrized by \(\mathbf{r}(t), a \leq t \leq b\).
Then
\begin{equation*}
    \int_C \mathbf{F} \cdot d \mathbf{r} 
    = \int_a^b \mathbf{F}(\mathbf{r}(t)) \cdot \mathbf{r}'(t) \, dt\,.
\end{equation*}
\end{proposition}
\end{frame}

\begin{frame}
    \frametitle{Notations}
    \begin{align*}
    \int_C f(x,y) dx := \int_a^b f(x(t), y(t) ) \, x'(t) \, dt \,, \\
    \int_C f(x,y) dy := \int_a^b f(x(t), y(t) ) \, y'(t) \, dt \,, \\
    \end{align*}

    We can abbreviate the above by
    \begin{equation*}
    \int_C P(x,y) dx +
    \int_C Q(x,y) dy 
    = \int_C P(x,y) \, dx + Q(x,y) \, dy \,.
    \end{equation*}
\end{frame}

\begin{frame}
Note: when integrating with respect to arc length like this, reverse the direction of traversing the curve \(C\) will NOT result
in a change of sign of the final solution.
\begin{equation*}
    \int_{-C} f(x,y) \, ds =  \int_C f(x,y) \, ds \,.
\end{equation*}
On the other hand,
\begin{equation*}
\int_{-C} P(x,y) \, dx + Q(x,y) \, dy 
= -\int_C P(x,y) \, dx + Q(x,y) \, dy \,.
\end{equation*}
\end{frame}
\begin{frame}
    \frametitle{Worksheet}
    Work is: \url{https://www.youtube.com/watch?v=oQqskrRWGco}

    A force field is given $\vect{F}(x,y) = x^2 \vect{i} - xy \vect{j}$.
    Suppose we want to move a particle along the quarter circle
    $\vect{r}(t) = \cos t \vect{i} + \sin t \vect{j}, 0 \leq t \leq \pi/2$.
    Compute the work done.
\end{frame}


\begin{frame}
\begin{theorem}[Fundamental Theorem for line integrals]
Let \(C\) be a smooth curve given by the parametrization \(\mathbf{r}(t)\),
\(a \leq t \leq b\).
Let \(f\) be a differentiable function of two or three
variables whose gradient vector \(\nabla f\) is continuous on \(C\).
Then,
\begin{equation*}
    \int_C \nabla f \cdot d\mathbf{r} = f(\mathbf{r}(b)) - f(\mathbf{r}(a)) \,.
\end{equation*}
\end{theorem}
\end{frame}


\end{document}

