\documentclass[aspectratio=169]{beamer}
\usetheme{Copenhagen}
%% Remove draft for real article, put twocolumn for two columns
\usetheme{metropolis}

\usepackage[utf8]{inputenc}
\newtheorem*{question}{Question}

\newcommand{\vectorproj}[2][]{\mathrm{proj}_{\vect{#1}}\vect{#2}}
\newcommand{\vectorcomp}[2][]{\mathrm{comp}_{\vect{#1}}\vect{#2}}
\newcommand{\vect}{\mathbf}
%% commentary bubble
\newcommand{\SV}[2][]{\sidenote[colback=green!10]{\textbf{SV\xspace #1:} #2}}

%% Title 
\title{ Multivariable Calculus \\ Day 3 \\ Some toy examples in $\R^3$} 
\institute{Fulbright University Vietnam}
%\author[1]{Co-author}
\author{Truong-Son Van}
\date{Spring 2023}

\begin{document}

\maketitle

\section{Equations for lines and planes}

\begin{frame}
    \frametitle{Equation for a line}
A line is a collection of points that is parallel to a vector and goes through a 
\begin{equation*}
    L = \{\vect{r}(t) \,|  \vect{r}(t) = \vect{r}_0 + t \vect{v}, t\in \R \}  \,,
\end{equation*}
where ${r}_0$ is the initial position and $\vect{v}$ is the direction.
The equation for $\vect{r}(t)$ is called a \textbf{vector equation for a line $L$}.
\end{frame}

\begin{frame}
    \frametitle{Equation for a line}
Let $\vect{v} = \langle v_1, v_2, v_3 \rangle$ and $\vect{r}_0 = ( x_0, y_0, z_0 )$.
The \textbf{parametric equations} of $L$ is the following system of equations

\begin{gather*}
    x = x_0 + v_1 t\,, \\
    y = y_0 + v_2 t\,, \\
    z = z_0 + v_3 t \,. 
\end{gather*}

This leads to the \textbf{symmetric equations} of $L$

\begin{equation*}
    \frac{x - x_0}{v_1} = \frac{y - y_0}{v_2} = \frac{z - z_0}{v_3} \,.
\end{equation*}
\end{frame}

\begin{frame}
    \frametitle{Worksheet}
    Find parametric equations and symmetric equations of the line that passes 
    through the points $A(2,4,-4)$ and $B(3,-1,1)$.
\end{frame}


\begin{frame}
    \frametitle{Equation for plane}
A plane is a collection of points that is perpendicular to one specific direction 
\begin{equation*}
    P = \{ \vect{r} \, | \, \vect{n} \cdot (\vect{r}- \vect{r}_0 ) = 0 \} \,.
\end{equation*}
$\vect{n}$ is the perpendicular vector to the plane called the normal vector.
In higher dimension, planes are called hyperplanes.
\end{frame}





\end{document}

