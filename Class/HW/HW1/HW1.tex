\documentclass[12pt]{article}
\usepackage{amsmath,amssymb,amsthm}
%% Remove draft for real article, put twocolumn for two columns

\newcommand{\vectorproj}[2][]{\mathrm{proj}_{\vect{#1}}\vect{#2}}
\newcommand{\vectorcomp}[2][]{\mathrm{comp}_{\vect{#1}}\vect{#2}}
\newcommand{\vect}{\mathbf}
\theoremstyle{definition}
\newtheorem{problem}{Problem}
\newcommand{\R}{\mathbb{R}}
%% commentary bubble

%% Title 
\title{Homework 1}
\author{ Multivariable Calculus}
%\author[1]{Co-author}
\date{Due: Feb 16, 2023, 23:59}

\begin{document}

\maketitle

Computational problems are graded for completion, each problem is worth 1 points.

Conceptual problems are graded for correctness, each problem is worth 5 points.

Show all your work to get full credits for each problem.
\section{Computational}
Do the following problems in Stewart's calculus textbook, 8th edition.

Section 12.2: 7, 20, 22, 24

Section 12.3: 2-10

Section 12.4: 1-7

\section{Conceptual}

\begin{problem}
    A street vendor sells $a$ banh mi, $b$ banh bao, $c$ coconuts
    on a given day.
    He charges \$ 25,000 VND for a banh mi, \$20,000 VND for a banh bao, 
    \$ 15,000 VND for a cononut.
    If $\vect{A} = \langle a,b,c \rangle$ and $\vect{P} = \langle 25, 20,15 \rangle$, what is the meaning of the dot product $\vect{A}\cdot \vect{P}$?
\end{problem}

\begin{problem}
    Let $\vect{r} = \langle x,y \rangle$ and 
    $\vect{r_1} = \langle x_1, y_1 \rangle, 
    \vect{r_2} = \langle x_2, y_2 \rangle$.
    Describe the set of all the points $(x,y)$
    such that
    $| \vect{r}- \vect{r_1} | + | \vect{r} - \vect{r_2} | = k$
    where 
    $k > | \vect{r_1} | + \vect{r_2}|$.
\end{problem}

\begin{problem}
     Prove the law of cosine in $\mathbb{R}^2$. In other words,
        prove that, 
        If $\theta$ is the angle between the vectors $\textbf{u}$ and $\textbf{v}$ in $\mathbb{R}^2$, then
   \begin{equation*}
        \textbf{u}\cdot \textbf{v} = |\textbf{u}|| \textbf{v}| \cos \theta \,.
   \end{equation*}
 Hint: 

 Step 1: prove the high school trigonometry identity $c^2 = a^2 + b^2 - 2ab\cos\theta$. (Look it up (and cite) if you don't know it)

 Step 2: use the identity in step 1.

\end{problem}


\begin{problem}
    Suppose $\vect{a}, \vect{b}, \vect{c}$ are
    nonzero vectors in $\R^3$.
    A formula to find the volume of the parallelepiped
    created by these vectors is
    \begin{equation*}
        V = | \vect{a} \cdot (\vect{b}\times \vect{c})| \,.
    \end{equation*}
    \begin{enumerate}
        \item Prove this formula.
        \item Test this formula with an example of a rectangular box constructed by yourself.
    \end{enumerate}
\end{problem}

%\bibliography{refs}


\end{document}
