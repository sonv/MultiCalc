\documentclass[12pt]{article}
\usepackage{amsmath,amssymb,amsthm}
%% Remove draft for real article, put twocolumn for two columns

\newcommand{\vectorproj}[2][]{\mathrm{proj}_{\vect{#1}}\vect{#2}}
\newcommand{\vectorcomp}[2][]{\mathrm{comp}_{\vect{#1}}\vect{#2}}
\newcommand{\vect}{\mathbf}
\theoremstyle{definition}
\newtheorem{problem}{Problem}
\newcommand{\R}{\mathbb{R}}
%% commentary bubble

%% Title 
\title{Homework 3}
\author{ Multivariable Calculus}
%\author[1]{Co-author}
\date{Due: Mar 10, 2023, 23:59}

\begin{document}

\maketitle

Computational problems are graded for completion, each problem is worth 1 points.

Conceptual problems are NOT graded but good for your own development if you like to prove things.

Show all your work to get full credits for each problem.
\section{Computational}
Do the following problems in Stewart's calculus textbook, 8th edition.

Section 14.2: 5-22, 25,26, 29-38

\section{Conceptual}
For this homework, conceptual problems are present for those who like proofs.
There will be no grading for these but you're encouraged to meet with Dr. Son
to discuss the problems.

\begin{problem}
    Graph and discuss the continuity of the function
    \begin{equation*}
        f(x,y) = 
        \begin{cases}
            \frac{\sin xy}{xy} & xy\not=0 \\
            1 & xy= 0 \,.
        \end{cases}
    \end{equation*}
\end{problem}

\begin{problem}
    Prove the following theorem.



Let \(L,M\) and \(k\) be real numbers and that
\begin{equation*}
    \lim_{(x,y) \to (x_0,y_0)} f(x,y) = L \,, \qquad 
    \lim_{(x,y) \to (x_0,y_0)} g(x,y) = M \,.
\end{equation*}
We then have

\begin{enumerate}
\def\labelenumi{\arabic{enumi}.}
\item
  \(\displaystyle \lim_{(x,y) \to (x_0,y_0)} (f(x,y) + g(x,y)) = L + M\),
\item
  \(\displaystyle \lim_{(x,y) \to (x_0,y_0)} (k f(x,y)) = kL\),
\item
  \(\displaystyle \lim_{(x,y) \to (x_0,y_0)} (f(x,y) g(x,y)) = LM\),
\item
  \(\displaystyle \lim_{(x,y) \to (x_0,y_0)} \frac{f(x,y)}{g(x,y)} = \frac{L}{M}\) if \(M \not= 0\),
\item
  \(\displaystyle \lim_{(x,y) \to (x_0,y_0)} {f(x,y)^p} = L^p\) for \(p=1,2,3 \dots\).
\end{enumerate}

\end{problem}


%\bibliography{refs}


\end{document}
