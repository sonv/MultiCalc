\documentclass[aspectratio=169]{beamer}
\usetheme{Copenhagen}
%% Remove draft for real article, put twocolumn for two columns
\usetheme{metropolis}

\usepackage[utf8]{inputenc}
\newtheorem*{question}{Question}

\newcommand{\vectorproj}[2][]{\mathrm{proj}_{\vect{#1}}\vect{#2}}
\newcommand{\vectorcomp}[2][]{\mathrm{comp}_{\vect{#1}}\vect{#2}}
\newcommand{\vect}{\mathbf}
\newcommand{\R}{\mathbb{R}}
%% commentary bubble
\newcommand{\SV}[2][]{\sidenote[colback=green!10]{\textbf{SV\xspace #1:} #2}}

%% Title 
\title{ Multivariable Calculus \\ Day 5 \\ Integrals of vector functions} 
\institute{Fulbright University Vietnam}
%\author[1]{Co-author}
\author{Truong-Son Van}
\date{Spring 2023}

\begin{document}

\maketitle

\begin{frame}
    \frametitle{Arc Length}
    Terminology:
    Arc length = curve length

    Line segment length = the length of a line

    Idea:
    Arc length = limit of sum of lengths of small line segments
\end{frame}

\begin{frame}
    \frametitle{Worksheet}
    \begin{itemize}
        \item Compute the length of the curve
            $$\vect{r}(t) =  \langle \cos(t) ,\sin(t), t \rangle$$
            where $t\in [0,2\pi]$.
        \item Compute the length of the curve made by the graph of the function
    $$ f(x) = x^3 $$
    where $x \in [1,4]$.
    \end{itemize}
\end{frame}


\end{document}

