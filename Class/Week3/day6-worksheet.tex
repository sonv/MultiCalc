\documentclass{amsart}
%% Remove draft for real article, put twocolumn for two columns
\calclayout
\usepackage[utf8]{inputenc}
\newtheorem*{question}{Question}

\newcommand{\vectorproj}[2][]{\mathrm{proj}_{\vect{#1}}\vect{#2}}
\newcommand{\vectorcomp}[2][]{\mathrm{comp}_{\vect{#1}}\vect{#2}}
\newcommand{\vect}{\mathbf}
\newcommand{\R}{\mathbb{R}}
%% commentary bubble
\newcommand{\SV}[2][]{\sidenote[colback=green!10]{\textbf{SV\xspace #1:} #2}}

%% Title 
\title{ Multivariable Calculus \\ Worksheet \\ Integrals of vector functions} 
\author{}
\date{Spring 2023}

\begin{document}

\maketitle
\begin{enumerate}
    \item Compute the length of the curve
        $$\vect{r}(t) =  \langle \cos(t) ,\sin(t), t \rangle$$
        where $t\in [0,2\pi]$.
    \item Compute the length of the curve made by the graph of the function
$$ f(x) = x^3 $$
where $x \in [1,4]$.
    \item Derive a formula to compute the length of a curve made by 
        the graph of a function
        $f:[a,b] \to \R$.

        (Hint: use the previous question as inspiration.)

    \item 
        According to Wikipedia,
        ``In differential geometry of curves, the osculating circle of a sufficiently smooth plane curve at a given point p on the curve has been traditionally defined as the circle passing through p and a pair of additional points on the curve infinitesimally close to p. Its center lies on the inner normal line, and its curvature defines the curvature of the given curve at that point. This circle, which is the one among all tangent circles at the given point that approaches the curve most tightly, was named circulus osculans (Latin for "kissing circle") by Leibniz.''

        Given a space curve $\vect{r}(t)$ that is infinitely differentiable and 
        $\vect{r}'(t)\not=0$ and $\vect{r}''(t)\not=0$,
        do the following
        \begin{enumerate}
            \item Without formulas, describe an idea to compute the radius of 
                the osculating circle at any given point on the curve.
            \item Execute your plan.
            \item Relate your result with the derivative of the unit tangent vector.
        \end{enumerate}
\end{enumerate}

\end{document}
