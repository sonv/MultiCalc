\documentclass[aspectratio=169]{beamer}
\usetheme{Copenhagen}
%% Remove draft for real article, put twocolumn for two columns
\usetheme{metropolis}

\usepackage[utf8]{inputenc}
\newtheorem*{question}{Question}

\newcommand{\vectorproj}[2][]{\mathrm{proj}_{\vect{#1}}\vect{#2}}
\newcommand{\vectorcomp}[2][]{\mathrm{comp}_{\vect{#1}}\vect{#2}}
\newcommand{\vect}{\mathbf}
\newcommand{\R}{\mathbb{R}}
%% commentary bubble
\newcommand{\SV}[2][]{\sidenote[colback=green!10]{\textbf{SV\xspace #1:} #2}}

%% Title 
\title{ Multivariable Calculus \\ Day 5 \\ Integrals of vector functions} 
\institute{Fulbright University Vietnam}
%\author[1]{Co-author}
\author{Truong-Son Van}
\date{Spring 2023}

\begin{document}

\maketitle

\begin{frame}
    \frametitle{Refresher}
    Let 
    $$\vect{r}(t) = \langle \cos(t), t^2, \sin(t) \rangle$$
    \begin{itemize}
        \item Compute $\vect{s}(t)=\vect{r}'(t)/|\vect{r}'(t)|$
        \item Compute $\vect{s}'(t)\cdot \vect{r}'(t)$
    \end{itemize}

\end{frame}


\section{Integrals}

\begin{frame}
    \frametitle{Coordinate-wise integrals}
    Indefinite integral
\begin{equation*}
    \int \vect{r}(t) \, dt = \left\langle \int r_1(t) \, dt, \int r_2(t) \, dt, \int r_3(t) \, dt \right\rangle = \vect{R}(t) + \vect{C} \,. 
\end{equation*}

Definite integral
\begin{equation*}
    \int_a^b \vect{r}(t) \, dt = \left\langle \int_a^b r_1(t) \, dt, \int_a^b r_2(t) \, dt, \int_a^b r_3(t) \, dt \right\rangle \,. 
\end{equation*}

Fundamental Theorem of Calculus
\begin{equation*}
    \int_a^b \vect{r}(t) \, dt =  ?
\end{equation*}
\end{frame}

\begin{frame}
    \frametitle{Worksheet}
    Compute
         $$\vect{R}(t) = \int (\cos(t)\vect{i} + \vect{j} - 2t\vect{k}) \, dt \,.$$
    Compute 
        $$\int_0^\pi \langle \cos(t), 1, -2t  \rangle \, dt \,. $$
\end{frame}



\begin{frame}
    \frametitle{ Arc length}
\end{frame}

\begin{frame}
    \frametitle{Worksheet}
    Compute the length of the line segment
    \begin{equation*}
        \vect{r}(t) = \langle 1,1,1 \rangle + t \langle 2,2,2 \rangle \,, t\in [0, 4] \,.
    \end{equation*}
\end{frame}

\begin{frame}
    \frametitle{Space curve}
    A vector function that is continuous.

    \begin{example}
        \begin{equation*}
            \vect{r}(t) = \langle \cos(t), \sin(t), t \rangle
        \end{equation*}
    \end{example}
\end{frame}

\begin{frame}
    \frametitle{Worksheet}
    Compute the length of the following curve
    \begin{equation*}
        \vect{r}(t) = 
        \begin{cases}
            \langle 1,1,1 \rangle + t \vect{i} \,, t \in [0,1] \\
            \vect{r}(1) + (t-1) \vect{j} \,, t \in [1,2] \,.
        \end{cases}
    \end{equation*}
\end{frame}

\begin{frame}
    \frametitle{Question}
    How can you compute the length of any curve?
\end{frame}


\begin{frame}
    \frametitle{Worksheet}
    \begin{itemize}
        \item Compute the length of the curve
            $$\vect{r}(t) =  \langle \cos(t) ,\sin(t), t \rangle$$
            where $t\in [0,2\pi]$.
        \item Compute the length of the curve made by the graph of the function
    $$ f(x) = x^3 $$
    where $x \in [1,4]$.
    \end{itemize}
\end{frame}


\end{document}

