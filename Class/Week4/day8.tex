\documentclass[aspectratio=169,handout]{beamer}
\usetheme{Copenhagen}
%% Remove draft for real article, put twocolumn for two columns
\usetheme{metropolis}
\usepackage{multicol}

\usepackage[utf8]{inputenc}
\newtheorem*{question}{Question}

\newcommand{\vectorproj}[2][]{\mathrm{proj}_{\vect{#1}}\vect{#2}}
\newcommand{\vectorcomp}[2][]{\mathrm{comp}_{\vect{#1}}\vect{#2}}
\newcommand{\vect}{\mathbf}
\newcommand{\R}{\mathbb{R}}
%% commentary bubble
\newcommand{\SV}[2][]{\sidenote[colback=green!10]{\textbf{SV\xspace #1:} #2}}

%% Title 
\title{ Multivariable Calculus \\ Day 8 \\ Limits and Continuity} 
\institute{Fulbright University Vietnam}
%\author[1]{Co-author}
\author{Truong-Son Van}
\date{Spring 2023}

\begin{document}

\maketitle


\section{Limits}


\begin{frame}
    \begin{definition}
Let $f$ be a function of two variables whose domain $D$ includes points arbitrarily close to $(a,b)$. Then we say that the limit of $f(x,y)$ as $(x,y)$ approaches $(a,b)$ is $L$ and we write

$$\lim_{(x,y)\to(a,b)} f(x,y) = L$$

if for every number $\epsilon > 0$ there is a corresponding number $\delta > 0$ such that
$|f(x,y) - L| < \epsilon$
if $(x,y) \in D$ and $0 < \sqrt{(x-a)^2 + (y-b)^2} < \delta$.
    \end{definition}
\end{frame}


\begin{frame}
    Sadly, we will not go deeply about this concept because it requires real analysis and
    we have bigger fishes to fry.

    What we will learn:
    \begin{enumerate}
        \item Simple cases when limits exist
        \item Typical cases when limits don't exist
    \end{enumerate}
\end{frame}

\begin{frame}
    \frametitle{Simple cases when limits exist}
    \begin{theorem}
Let $L,M$ and $k$ be real numbers and that 
\begin{equation*}
    \lim_{(x,y) \to (x_0,y_0)} f(x,y) = L \,, \qquad 
    \lim_{(x,y) \to (x_0,y_0)} g(x,y) = M \,.
\end{equation*}
We then have

\begin{multicols}{2}
\begin{enumerate}
    \item $\displaystyle \lim_{(x,y) \to (x_0,y_0)} (f(x,y) + g(x,y)) = L + M$,

    \item $\displaystyle  \lim_{(x,y) \to (x_0,y_0)} (k f(x,y)) = kL$,

    \item $\displaystyle \lim_{(x,y) \to (x_0,y_0)} (f(x,y) g(x,y)) = LM$,

    \item $\displaystyle \lim_{(x,y) \to (x_0,y_0)} \frac{f(x,y)}{g(x,y)} = \frac{L}{M}$ if $M \not= 0$,

    \item $\displaystyle \lim_{(x,y) \to (x_0,y_0)} {f(x,y)^p} = L^p$ for $p>0$,
\end{enumerate}
\end{multicols}
   \end{theorem}
\end{frame}


\begin{frame}
    \frametitle{Worksheet}
    Determine if the limit exists and if it is, find it.
    \begin{enumerate}
        \item $\displaystyle \lim_{(x,y) \to (0,1)} \frac{x - xy + 3}{x^2y + 5xy - y^3}$,
        \item $\displaystyle \lim_{(x,y) \to (3,-4)} \sqrt{x^2 + y^2}$,
        \item $\displaystyle \lim_{(x,y) \to (0,0)} \frac{x^2 - xy}{\sqrt{x} - \sqrt{y}}$,
        \item $ \displaystyle \lim_{(x,y) \to (0,0)} \frac{x}{y} $.
    \end{enumerate}
\end{frame}


\begin{frame}
    \frametitle{Two-path test}
If $\lim_{(x,y) \to (a,b)} f(x,y) = L_1$ as $(x,y) \to (a,b)$ along a path $C_1$, 
and $\lim_{(x,y) \to (a,b)} f(x,y) = L_2$ as $(x,y) \to (a,b)$ along a path $C_2$, 
where $L_1 \neq L_2$, then $\lim_{(x,y) \to (a,b)} f(x,y)$ does not exist.
\end{frame}

\begin{frame}
    \frametitle{Worksheet}
    Determine if the limit exists and if it is, find it.
    \begin{enumerate}
        \item $\displaystyle \lim_{(x,y)\to (0,0)} \frac{x^2}{3x^2 + 4y^2}$
        \item $\displaystyle \lim_{(x, y, z) \rightarrow (0, 0, 0)} \frac{4 xy+ 2 yz+ 2 xz
}{16 x^2+ 4 y^2+ 4 z^2} $
    \item $\displaystyle \lim_{(x,y)\to(0,0)} \frac{3x^2y}{x^2 + y^2}$
    \end{enumerate}
\end{frame}

\section{Continuity}

\begin{frame}
    \frametitle{Continuity}
    A function $f$ is continuous at $\vect{a}\in \R^n$ if
    \begin{equation*}
        \lim_{\vect{x}\to \vect{a}} f(\vect{x}) = f(\vect{a}) \,.
    \end{equation*}
    We say that $f$ is continuous on $D$ if $f$ is continuous 
    at every point $\vect{x} \in D$.
\end{frame}






\end{document}

