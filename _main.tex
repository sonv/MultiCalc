% Options for packages loaded elsewhere
\PassOptionsToPackage{unicode}{hyperref}
\PassOptionsToPackage{hyphens}{url}
%
\documentclass[
]{article}
\usepackage{lmodern}
\usepackage{amssymb,amsmath}
\usepackage{ifxetex,ifluatex}
\ifnum 0\ifxetex 1\fi\ifluatex 1\fi=0 % if pdftex
  \usepackage[T1]{fontenc}
  \usepackage[utf8]{inputenc}
  \usepackage{textcomp} % provide euro and other symbols
\else % if luatex or xetex
  \usepackage{unicode-math}
  \defaultfontfeatures{Scale=MatchLowercase}
  \defaultfontfeatures[\rmfamily]{Ligatures=TeX,Scale=1}
\fi
% Use upquote if available, for straight quotes in verbatim environments
\IfFileExists{upquote.sty}{\usepackage{upquote}}{}
\IfFileExists{microtype.sty}{% use microtype if available
  \usepackage[]{microtype}
  \UseMicrotypeSet[protrusion]{basicmath} % disable protrusion for tt fonts
}{}
\makeatletter
\@ifundefined{KOMAClassName}{% if non-KOMA class
  \IfFileExists{parskip.sty}{%
    \usepackage{parskip}
  }{% else
    \setlength{\parindent}{0pt}
    \setlength{\parskip}{6pt plus 2pt minus 1pt}}
}{% if KOMA class
  \KOMAoptions{parskip=half}}
\makeatother
\usepackage{xcolor}
\IfFileExists{xurl.sty}{\usepackage{xurl}}{} % add URL line breaks if available
\IfFileExists{bookmark.sty}{\usepackage{bookmark}}{\usepackage{hyperref}}
\hypersetup{
  pdftitle={MATH 104: Multivariable Calculus},
  pdfauthor={Truong-Son Van},
  hidelinks,
  pdfcreator={LaTeX via pandoc}}
\urlstyle{same} % disable monospaced font for URLs
\usepackage[margin=1in]{geometry}
\usepackage{longtable,booktabs}
% Correct order of tables after \paragraph or \subparagraph
\usepackage{etoolbox}
\makeatletter
\patchcmd\longtable{\par}{\if@noskipsec\mbox{}\fi\par}{}{}
\makeatother
% Allow footnotes in longtable head/foot
\IfFileExists{footnotehyper.sty}{\usepackage{footnotehyper}}{\usepackage{footnote}}
\makesavenoteenv{longtable}
\usepackage{graphicx}
\makeatletter
\def\maxwidth{\ifdim\Gin@nat@width>\linewidth\linewidth\else\Gin@nat@width\fi}
\def\maxheight{\ifdim\Gin@nat@height>\textheight\textheight\else\Gin@nat@height\fi}
\makeatother
% Scale images if necessary, so that they will not overflow the page
% margins by default, and it is still possible to overwrite the defaults
% using explicit options in \includegraphics[width, height, ...]{}
\setkeys{Gin}{width=\maxwidth,height=\maxheight,keepaspectratio}
% Set default figure placement to htbp
\makeatletter
\def\fps@figure{htbp}
\makeatother
\setlength{\emergencystretch}{3em} % prevent overfull lines
\providecommand{\tightlist}{%
  \setlength{\itemsep}{0pt}\setlength{\parskip}{0pt}}
\setcounter{secnumdepth}{5}

\title{MATH 104: Multivariable Calculus}
\author{Truong-Son Van}
\date{Spring 2023}

\begin{document}
\maketitle

{
\setcounter{tocdepth}{2}
\tableofcontents
}
\hypertarget{syllabus}{%
\section*{Syllabus}\label{syllabus}}
\addcontentsline{toc}{section}{Syllabus}

\hypertarget{key-information}{%
\subsubsection*{Key information}\label{key-information}}
\addcontentsline{toc}{subsubsection}{Key information}

\begin{itemize}
\tightlist
\item
  Instructor: Truong-Son Van
\item
  Email: \href{mailto:son.van+104@fulbright.edu.vn}{\nolinkurl{son.van+104@fulbright.edu.vn}}
\item
  Class time: T \& Th: 9:45a - 11:15a
\item
  Class Location: CR 502
\item
  Office hours: TBD
\item
  Prerequisites: Calculus (MATH 101)
\end{itemize}

\hypertarget{textbooks-and-references}{%
\subsection*{Textbooks and references}\label{textbooks-and-references}}
\addcontentsline{toc}{subsection}{Textbooks and references}

It is highly recommended that students read the textbooks.

\begin{enumerate}
\def\labelenumi{\arabic{enumi}.}
\item
  In-class worksheets.
\item
  Active Calculus: Multivariable by Schlicker et al.~2018 edition.
  (\url{https://activecalculus.org/multi/preface-2.html})
\item
  Thomas' Calculus: Early Transcendentals by Hass, Heil, et al.~\(14^{th}\) edition.
\item
  Calculus Early Transcendental by Stewart. \(8^{th}\) edition.
\item
  Anything you can find on Google would work.
  Calculus is a subject that people have written about
  so much. So, there's no excuse for not having access
  to the knowledge.
\end{enumerate}

\hypertarget{course-description}{%
\subsubsection*{Course description}\label{course-description}}
\addcontentsline{toc}{subsubsection}{Course description}

How do we describe the trajectory of a space shuttle? How is the human body affected by
scuba diving to different depths for different lengths of time? The mathematics required to
describe most real life systems involves functions of more than one variable. The concepts
of the derivative and integral from a first course in calculus must therefore be extended to
higher dimensional settings. In this course students will be guided through the essential ideas
of multivariable calculus, including partial derivatives, multiple integrals and vector calculus,
and their applications. These mathematical tools are used extensively in the physical sciences
and engineering, and in other areas including economics and computer graphics.

\hypertarget{learning-objectives}{%
\subsubsection*{Learning objectives}\label{learning-objectives}}
\addcontentsline{toc}{subsubsection}{Learning objectives}

After the course, students are expected to:

\begin{itemize}
\item
  Be confident in handling functions of two or more variables and familiar with how
  they can be represented graphically
\item
  Understand the key concepts of multivariable calculus, including partial derivatives,
  the gradient vector, multiple integrals, line and surface integrals, the divergence and
  curl of a vector function
\item
  Know how such derivatives and integrals are calculated and some of their uses
\item
  Be able to apply these ideas to real world problems
\item
  Have improved analytic, computational and problem solving skills
\end{itemize}

\hypertarget{assessment}{%
\subsection*{Assessment}\label{assessment}}
\addcontentsline{toc}{subsection}{Assessment}

During the course, students are expected to compute their own percentage
points based on the following scheme.
The instructor is not responsible for providing the running percentage.

\begin{longtable}[]{@{}cc@{}}
\toprule
\textbf{Form of assessment} & \textbf{Weight}\tabularnewline
\midrule
\endhead
Weekly homeworks & 25\%\tabularnewline
Class worksheets & 15\%\tabularnewline
Mini-project & 15\%\tabularnewline
Midterm & 15\%\tabularnewline
Final & 30\%\tabularnewline
\bottomrule
\end{longtable}

The following is the non-negotiable letter grade breakdown. It is based on
common practice in the United States for standard courses such as Calculus.

\begin{longtable}[]{@{}cc@{}}
\toprule
\textbf{Letter Grade} & \textbf{Percentage}\tabularnewline
\midrule
\endhead
A & {[}93,100{]}\tabularnewline
A- & {[}90,93)\tabularnewline
B+ & {[}87,90)\tabularnewline
B & {[}83,87)\tabularnewline
B- & {[}80, 83)\tabularnewline
C+ & {[}77,80)\tabularnewline
C & {[}73,77)\tabularnewline
C- & {[}70,73)\tabularnewline
D+ & {[}67,70)\tabularnewline
D & {[}60, 66)\tabularnewline
F & {[}0,60)\tabularnewline
\bottomrule
\end{longtable}

\hypertarget{core-content}{%
\subsection*{Core content}\label{core-content}}
\addcontentsline{toc}{subsection}{Core content}

\begin{enumerate}
\def\labelenumi{\arabic{enumi}.}
\tightlist
\item
  Introduction
\end{enumerate}

\begin{itemize}
\tightlist
\item
  Functions of two variables
\item
  Graphs in three dimensions, surfaces and level curves
\item
  Functions of three or more variables
\item
  Limits and continuity
\item
  Vectors (review)
\end{itemize}

\begin{enumerate}
\def\labelenumi{\arabic{enumi}.}
\setcounter{enumi}{1}
\tightlist
\item
  Partial Derivatives
\end{enumerate}

\begin{itemize}
\tightlist
\item
  Partial derivatives
\item
  Tangent planes, linear approximations and differentials
\item
  Chain rule
\item
  Directional derivatives and gradient vectors
\item
  Extrema and optimization
\item
  Lagrange Multipliers
\end{itemize}

\begin{enumerate}
\def\labelenumi{\arabic{enumi}.}
\setcounter{enumi}{2}
\tightlist
\item
  Multiple Integrals
\end{enumerate}

\begin{itemize}
\tightlist
\item
  Double integrals
\item
  Double integrals in polar coordinates
\item
  Triple integrals
\item
  Triple integrals in cylindrical and spherical coordinates
\item
  Applications of multiple integrals
\end{itemize}

\begin{enumerate}
\def\labelenumi{\arabic{enumi}.}
\setcounter{enumi}{3}
\tightlist
\item
  Vector Calculus
\end{enumerate}

\begin{itemize}
\tightlist
\item
  Vector functions and their derivatives
\item
  Vector fields
\item
  Line integrals
\item
  The fundamental theorem of line integrals
\item
  Green's Theorem
\item
  Parametric surfaces and surface integrals
\item
  Curl and divergence
\item
  Divergence Theorem
\item
  Stokes Theorem
\end{itemize}

\hypertarget{late-assignments}{%
\subsection*{Late assignments}\label{late-assignments}}
\addcontentsline{toc}{subsection}{Late assignments}

\begin{itemize}
\tightlist
\item
  15\% of the possible total mark will be deducted for every 24 hrs (or part of 24 hrs) after the
  deadline. Work more than 2 days late will not be accepted.
\item
  Except for exceptional circumstances (see \href{https://onestop.fulbright.edu.vn/s/article/Exceptional-Circumstance}{definition}),
  I will not extend the deadlines.
\end{itemize}

\hypertarget{time-expectations}{%
\subsection*{Time expectations}\label{time-expectations}}
\addcontentsline{toc}{subsection}{Time expectations}

Some materials require time to be accustomed to. Some students are quicker than
others. However, on average, you should expect 10-15 hours per week (including class time)
on the materials in order to know the subject relatively well.

\hypertarget{collaboration-plagiarism}{%
\subsubsection*{Collaboration \& Plagiarism}\label{collaboration-plagiarism}}
\addcontentsline{toc}{subsubsection}{Collaboration \& Plagiarism}

Plagiarism is the act of submitting the intellectual property of another person as your own.
It is one of the most serious of academic offences. Acts of plagiarism include, but are not
limited to:
- Copying all or a part of another person's work and presenting it as your own, or not
giving proper credit.
- Purchasing a paper from someone (or a website) and presenting it as your own work.
- Re-submitting your work from one course to fulfill a requirement in another course.
Olin College's Rob Martello says: ``You must always take extreme care to acknowledge the
source of all quotes, content, and theories. Even if you paraphrase the information, you need
to cite all sources that informed your analysis. Citations must include page numbers, website
addresses, times of interviews, or other details as needed. When in doubt, cite.''
Further details can be found in the Code of Academic Integrity.

\hypertarget{learning-support}{%
\subsection*{Learning Support}\label{learning-support}}
\addcontentsline{toc}{subsection}{Learning Support}

In addition to your course instructors, there are other resources available to support your
academic work at Fulbright, including one-on-one consultations with learning support staff,
supplementary workshops, and both individual and group tutoring and mentoring in course
content, language learning, and academic skills. If you would like to request learning support,
please contact Fulbright Learning Support (\url{https://learning-support.notion.site}).

\hypertarget{wellbeing}{%
\subsection*{Wellbeing}\label{wellbeing}}
\addcontentsline{toc}{subsection}{Wellbeing}

Mental health and wellbeing are essential for the success of your academic journey. The
Fulbright Wellness Center provides various services including counseling, safer community,
and accessibility services. If you are experiencing undue personal or academic stress, are
feeling unsafe, or would like to know more about issues related to wellbeing, please contact
the Wellness Center via \href{mailto:wellness@fulbright.edu.vn}{\nolinkurl{wellness@fulbright.edu.vn}} or visit the Wellness Center office on
Level 5 of the Crescent campus.

For more information, pleaes check
\url{https://onestop.fulbright.edu.vn/s/article/Health-and-Wellness-Introduction}

\newpage

\hypertarget{tentative-course-schedule}{%
\subsection*{Tentative Course Schedule}\label{tentative-course-schedule}}
\addcontentsline{toc}{subsection}{Tentative Course Schedule}

The following schedule will be updated as we go so that students will know what to read
before/after class.

\begin{longtable}[]{@{}cclc@{}}
\toprule
\begin{minipage}[b]{0.12\columnwidth}\centering
\textbf{Date}\strut
\end{minipage} & \begin{minipage}[b]{0.10\columnwidth}\centering
\textbf{Session}\strut
\end{minipage} & \begin{minipage}[b]{0.55\columnwidth}\raggedright
\textbf{Content}\strut
\end{minipage} & \begin{minipage}[b]{0.12\columnwidth}\centering
\textbf{Deadlines}\strut
\end{minipage}\tabularnewline
\midrule
\endhead
\begin{minipage}[t]{0.12\columnwidth}\centering
Feb 7 \& 9\strut
\end{minipage} & \begin{minipage}[t]{0.10\columnwidth}\centering
1\&2\strut
\end{minipage} & \begin{minipage}[t]{0.55\columnwidth}\raggedright
Linear algebra crash course \& Vector-valued functions\strut
\end{minipage} & \begin{minipage}[t]{0.12\columnwidth}\centering
\strut
\end{minipage}\tabularnewline
\begin{minipage}[t]{0.12\columnwidth}\centering
Feb 14 \& 16\strut
\end{minipage} & \begin{minipage}[t]{0.10\columnwidth}\centering
3 \& 4\strut
\end{minipage} & \begin{minipage}[t]{0.55\columnwidth}\raggedright
Integration: arc length\strut
\end{minipage} & \begin{minipage}[t]{0.12\columnwidth}\centering
\strut
\end{minipage}\tabularnewline
\begin{minipage}[t]{0.12\columnwidth}\centering
Feb 21 \& 23\strut
\end{minipage} & \begin{minipage}[t]{0.10\columnwidth}\centering
5 \& 6\strut
\end{minipage} & \begin{minipage}[t]{0.55\columnwidth}\raggedright
Introduction and functions of several variables\strut
\end{minipage} & \begin{minipage}[t]{0.12\columnwidth}\centering
\strut
\end{minipage}\tabularnewline
\begin{minipage}[t]{0.12\columnwidth}\centering
Feb 28 \& Mar 2\strut
\end{minipage} & \begin{minipage}[t]{0.10\columnwidth}\centering
7 \& 8\strut
\end{minipage} & \begin{minipage}[t]{0.55\columnwidth}\raggedright
Limits \& continuity\strut
\end{minipage} & \begin{minipage}[t]{0.12\columnwidth}\centering
\strut
\end{minipage}\tabularnewline
\begin{minipage}[t]{0.12\columnwidth}\centering
Mar 7 \& 9\strut
\end{minipage} & \begin{minipage}[t]{0.10\columnwidth}\centering
9 \& 10\strut
\end{minipage} & \begin{minipage}[t]{0.55\columnwidth}\raggedright
Partial derivatives \& Tangent planes, chain rules \& higher derivatives\strut
\end{minipage} & \begin{minipage}[t]{0.12\columnwidth}\centering
\strut
\end{minipage}\tabularnewline
\begin{minipage}[t]{0.12\columnwidth}\centering
Mar 14 \& 16\strut
\end{minipage} & \begin{minipage}[t]{0.10\columnwidth}\centering
11 \& 12\strut
\end{minipage} & \begin{minipage}[t]{0.55\columnwidth}\raggedright
Directional derivatives, gradient vectors \& Extrema and Optimization\strut
\end{minipage} & \begin{minipage}[t]{0.12\columnwidth}\centering
\strut
\end{minipage}\tabularnewline
\begin{minipage}[t]{0.12\columnwidth}\centering
Mar 21 \& 23\strut
\end{minipage} & \begin{minipage}[t]{0.10\columnwidth}\centering
13 \& 14\strut
\end{minipage} & \begin{minipage}[t]{0.55\columnwidth}\raggedright
Extrema and Optimization (cont.)\strut
\end{minipage} & \begin{minipage}[t]{0.12\columnwidth}\centering
\strut
\end{minipage}\tabularnewline
\begin{minipage}[t]{0.12\columnwidth}\centering
Mar 28\strut
\end{minipage} & \begin{minipage}[t]{0.10\columnwidth}\centering
15\strut
\end{minipage} & \begin{minipage}[t]{0.55\columnwidth}\raggedright
Midterm exam\strut
\end{minipage} & \begin{minipage}[t]{0.12\columnwidth}\centering
\strut
\end{minipage}\tabularnewline
\begin{minipage}[t]{0.12\columnwidth}\centering
\strut
\end{minipage} & \begin{minipage}[t]{0.10\columnwidth}\centering
\strut
\end{minipage} & \begin{minipage}[t]{0.55\columnwidth}\raggedright
\strut
\end{minipage} & \begin{minipage}[t]{0.12\columnwidth}\centering
\strut
\end{minipage}\tabularnewline
\begin{minipage}[t]{0.12\columnwidth}\centering
\textbf{Mar 30}\strut
\end{minipage} & \begin{minipage}[t]{0.10\columnwidth}\centering
\strut
\end{minipage} & \begin{minipage}[t]{0.55\columnwidth}\raggedright
\textbf{Break}\strut
\end{minipage} & \begin{minipage}[t]{0.12\columnwidth}\centering
\strut
\end{minipage}\tabularnewline
\begin{minipage}[t]{0.12\columnwidth}\centering
\strut
\end{minipage} & \begin{minipage}[t]{0.10\columnwidth}\centering
\strut
\end{minipage} & \begin{minipage}[t]{0.55\columnwidth}\raggedright
\strut
\end{minipage} & \begin{minipage}[t]{0.12\columnwidth}\centering
\strut
\end{minipage}\tabularnewline
\begin{minipage}[t]{0.12\columnwidth}\centering
Apr 4 \& 6\strut
\end{minipage} & \begin{minipage}[t]{0.10\columnwidth}\centering
16 \& 17\strut
\end{minipage} & \begin{minipage}[t]{0.55\columnwidth}\raggedright
Double integrals\strut
\end{minipage} & \begin{minipage}[t]{0.12\columnwidth}\centering
\strut
\end{minipage}\tabularnewline
\begin{minipage}[t]{0.12\columnwidth}\centering
Apr 11\strut
\end{minipage} & \begin{minipage}[t]{0.10\columnwidth}\centering
18\strut
\end{minipage} & \begin{minipage}[t]{0.55\columnwidth}\raggedright
Applications of Double Integrals\strut
\end{minipage} & \begin{minipage}[t]{0.12\columnwidth}\centering
\strut
\end{minipage}\tabularnewline
\begin{minipage}[t]{0.12\columnwidth}\centering
Apr 13\strut
\end{minipage} & \begin{minipage}[t]{0.10\columnwidth}\centering
19\strut
\end{minipage} & \begin{minipage}[t]{0.55\columnwidth}\raggedright
Surface Area\strut
\end{minipage} & \begin{minipage}[t]{0.12\columnwidth}\centering
\strut
\end{minipage}\tabularnewline
\begin{minipage}[t]{0.12\columnwidth}\centering
Apr 18 \& 20\strut
\end{minipage} & \begin{minipage}[t]{0.10\columnwidth}\centering
20 \& 21\strut
\end{minipage} & \begin{minipage}[t]{0.55\columnwidth}\raggedright
Triple Integrals\strut
\end{minipage} & \begin{minipage}[t]{0.12\columnwidth}\centering
\strut
\end{minipage}\tabularnewline
\begin{minipage}[t]{0.12\columnwidth}\centering
Apr 25 \& 27\strut
\end{minipage} & \begin{minipage}[t]{0.10\columnwidth}\centering
22 \& 23\strut
\end{minipage} & \begin{minipage}[t]{0.55\columnwidth}\raggedright
Change of variables\strut
\end{minipage} & \begin{minipage}[t]{0.12\columnwidth}\centering
\strut
\end{minipage}\tabularnewline
\begin{minipage}[t]{0.12\columnwidth}\centering
\strut
\end{minipage} & \begin{minipage}[t]{0.10\columnwidth}\centering
\strut
\end{minipage} & \begin{minipage}[t]{0.55\columnwidth}\raggedright
\strut
\end{minipage} & \begin{minipage}[t]{0.12\columnwidth}\centering
\strut
\end{minipage}\tabularnewline
\begin{minipage}[t]{0.12\columnwidth}\centering
\textbf{May 2}\strut
\end{minipage} & \begin{minipage}[t]{0.10\columnwidth}\centering
\strut
\end{minipage} & \begin{minipage}[t]{0.55\columnwidth}\raggedright
\textbf{Break}\strut
\end{minipage} & \begin{minipage}[t]{0.12\columnwidth}\centering
\strut
\end{minipage}\tabularnewline
\begin{minipage}[t]{0.12\columnwidth}\centering
\strut
\end{minipage} & \begin{minipage}[t]{0.10\columnwidth}\centering
\strut
\end{minipage} & \begin{minipage}[t]{0.55\columnwidth}\raggedright
\strut
\end{minipage} & \begin{minipage}[t]{0.12\columnwidth}\centering
\strut
\end{minipage}\tabularnewline
\begin{minipage}[t]{0.12\columnwidth}\centering
May 4\strut
\end{minipage} & \begin{minipage}[t]{0.10\columnwidth}\centering
24\strut
\end{minipage} & \begin{minipage}[t]{0.55\columnwidth}\raggedright
\strut
\end{minipage} & \begin{minipage}[t]{0.12\columnwidth}\centering
\strut
\end{minipage}\tabularnewline
\begin{minipage}[t]{0.12\columnwidth}\centering
May 9\strut
\end{minipage} & \begin{minipage}[t]{0.10\columnwidth}\centering
25\strut
\end{minipage} & \begin{minipage}[t]{0.55\columnwidth}\raggedright
\strut
\end{minipage} & \begin{minipage}[t]{0.12\columnwidth}\centering
\strut
\end{minipage}\tabularnewline
\begin{minipage}[t]{0.12\columnwidth}\centering
May 11\strut
\end{minipage} & \begin{minipage}[t]{0.10\columnwidth}\centering
26\strut
\end{minipage} & \begin{minipage}[t]{0.55\columnwidth}\raggedright
\strut
\end{minipage} & \begin{minipage}[t]{0.12\columnwidth}\centering
\strut
\end{minipage}\tabularnewline
\begin{minipage}[t]{0.12\columnwidth}\centering
May 16\strut
\end{minipage} & \begin{minipage}[t]{0.10\columnwidth}\centering
27\strut
\end{minipage} & \begin{minipage}[t]{0.55\columnwidth}\raggedright
\strut
\end{minipage} & \begin{minipage}[t]{0.12\columnwidth}\centering
\strut
\end{minipage}\tabularnewline
\begin{minipage}[t]{0.12\columnwidth}\centering
May 18\strut
\end{minipage} & \begin{minipage}[t]{0.10\columnwidth}\centering
28\strut
\end{minipage} & \begin{minipage}[t]{0.55\columnwidth}\raggedright
\strut
\end{minipage} & \begin{minipage}[t]{0.12\columnwidth}\centering
\strut
\end{minipage}\tabularnewline
\begin{minipage}[t]{0.12\columnwidth}\centering
May 23\strut
\end{minipage} & \begin{minipage}[t]{0.10\columnwidth}\centering
29\strut
\end{minipage} & \begin{minipage}[t]{0.55\columnwidth}\raggedright
\strut
\end{minipage} & \begin{minipage}[t]{0.12\columnwidth}\centering
\strut
\end{minipage}\tabularnewline
\begin{minipage}[t]{0.12\columnwidth}\centering
May 25\strut
\end{minipage} & \begin{minipage}[t]{0.10\columnwidth}\centering
30\strut
\end{minipage} & \begin{minipage}[t]{0.55\columnwidth}\raggedright
\strut
\end{minipage} & \begin{minipage}[t]{0.12\columnwidth}\centering
\strut
\end{minipage}\tabularnewline
\bottomrule
\end{longtable}

\end{document}
