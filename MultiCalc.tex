% Options for packages loaded elsewhere
\PassOptionsToPackage{unicode}{hyperref}
\PassOptionsToPackage{hyphens}{url}
%
\documentclass[
]{book}
\usepackage{amsmath,amssymb}
\usepackage{iftex}
\ifPDFTeX
  \usepackage[T1]{fontenc}
  \usepackage[utf8]{inputenc}
  \usepackage{textcomp} % provide euro and other symbols
\else % if luatex or xetex
  \usepackage{unicode-math} % this also loads fontspec
  \defaultfontfeatures{Scale=MatchLowercase}
  \defaultfontfeatures[\rmfamily]{Ligatures=TeX,Scale=1}
\fi
\usepackage{lmodern}
\ifPDFTeX\else
  % xetex/luatex font selection
\fi
% Use upquote if available, for straight quotes in verbatim environments
\IfFileExists{upquote.sty}{\usepackage{upquote}}{}
\IfFileExists{microtype.sty}{% use microtype if available
  \usepackage[]{microtype}
  \UseMicrotypeSet[protrusion]{basicmath} % disable protrusion for tt fonts
}{}
\makeatletter
\@ifundefined{KOMAClassName}{% if non-KOMA class
  \IfFileExists{parskip.sty}{%
    \usepackage{parskip}
  }{% else
    \setlength{\parindent}{0pt}
    \setlength{\parskip}{6pt plus 2pt minus 1pt}}
}{% if KOMA class
  \KOMAoptions{parskip=half}}
\makeatother
\usepackage{xcolor}
\usepackage[margin=1in]{geometry}
\usepackage{longtable,booktabs,array}
\usepackage{calc} % for calculating minipage widths
% Correct order of tables after \paragraph or \subparagraph
\usepackage{etoolbox}
\makeatletter
\patchcmd\longtable{\par}{\if@noskipsec\mbox{}\fi\par}{}{}
\makeatother
% Allow footnotes in longtable head/foot
\IfFileExists{footnotehyper.sty}{\usepackage{footnotehyper}}{\usepackage{footnote}}
\makesavenoteenv{longtable}
\usepackage{graphicx}
\makeatletter
\def\maxwidth{\ifdim\Gin@nat@width>\linewidth\linewidth\else\Gin@nat@width\fi}
\def\maxheight{\ifdim\Gin@nat@height>\textheight\textheight\else\Gin@nat@height\fi}
\makeatother
% Scale images if necessary, so that they will not overflow the page
% margins by default, and it is still possible to overwrite the defaults
% using explicit options in \includegraphics[width, height, ...]{}
\setkeys{Gin}{width=\maxwidth,height=\maxheight,keepaspectratio}
% Set default figure placement to htbp
\makeatletter
\def\fps@figure{htbp}
\makeatother
\setlength{\emergencystretch}{3em} % prevent overfull lines
\providecommand{\tightlist}{%
  \setlength{\itemsep}{0pt}\setlength{\parskip}{0pt}}
\setcounter{secnumdepth}{5}
\ifLuaTeX
  \usepackage{selnolig}  % disable illegal ligatures
\fi
\usepackage[]{natbib}
\bibliographystyle{plainnat}
\IfFileExists{bookmark.sty}{\usepackage{bookmark}}{\usepackage{hyperref}}
\IfFileExists{xurl.sty}{\usepackage{xurl}}{} % add URL line breaks if available
\urlstyle{same}
\hypersetup{
  pdftitle={MATH 104: Multivariable Calculus (brief notes)},
  pdfauthor={Truong-Son Van},
  hidelinks,
  pdfcreator={LaTeX via pandoc}}

\title{MATH 104: Multivariable Calculus (brief notes)}
\author{Truong-Son Van}
\date{}

\usepackage{amsthm}
\newtheorem{theorem}{Theorem}[chapter]
\newtheorem{lemma}{Lemma}[chapter]
\newtheorem{corollary}{Corollary}[chapter]
\newtheorem{proposition}{Proposition}[chapter]
\newtheorem{conjecture}{Conjecture}[chapter]
\theoremstyle{definition}
\newtheorem{definition}{Definition}[chapter]
\theoremstyle{definition}
\newtheorem{example}{Example}[chapter]
\theoremstyle{definition}
\newtheorem{exercise}{Exercise}[chapter]
\theoremstyle{definition}
\newtheorem{hypothesis}{Hypothesis}[chapter]
\theoremstyle{remark}
\newtheorem*{remark}{Remark}
\newtheorem*{solution}{Solution}
\begin{document}
\maketitle

{
\setcounter{tocdepth}{2}
\tableofcontents
}
\chapter*{Spring 2024}\label{spring-2024}

\newpage

\newcommand{\vectorproj}[2][]{\mathrm{proj}_{\vect{#1}}\vect{#2}}
\newcommand{\vectorcomp}[2][]{\mathrm{comp}_{\vect{#1}}\vect{#2}}
\newcommand{\vect}{\mathbf}
\newcommand{\R}{\mathbb{R}}

\chapter{Vectors}\label{vectors}

\section{Basics}\label{basics}

\textbf{Reading: Stewart Chapter 12, Thomas Calculus Chapter 12,
Active Calculus Chapter 9}

You should be able to answer the following questions after reading this section:

\begin{itemize}
\item
  What is a vector?
\item
  What does it mean for two vectors to be equal?
\item
  How do we add two vectors together and multiply a vector by a scalar?
\item
  How do we determine the magnitude of a vector?
\item
  What is a unit vector
\item
  How do we find a unit vector in the direction of a given vector?
\end{itemize}

Typically, we talk about 3-dimensional vectors (as discussed in Stewart and Thomas).
However, since talking about \(n\)-dimensional vectors doesn't
require much more effort,
we will talk about \(n\)-dimensional vectors instead.

\begin{definition}
An \(n\)-dimensional Euclidean space \(\mathbb{R}^n\)
is the Cartesian product of \(n\) Euclidean spaces \(\mathbb{R}\).
\end{definition}

\begin{definition}
An \(n\)-dimensional vector \(\textbf{v}\in \mathbb{R}^n\) is a tuple
\begin{equation}
    \textbf{v} = \langle v_1,\dots, v_n \rangle \,,
\end{equation}
where \(v_i \in \mathbb{R}\).
\end{definition}

In dimensions less than or equal to 3, we represent a vector
geometrically by an arrow, whose length represents the magnitude.

\begin{remark}
A point in \(\mathbb{R}^n\) is also represented by an \(n\)-tuple
but with round brackets.
A vector connecting two points \(A= (a_1, \dots, a_n)\)
and \(B=(b_1, \dots, b_n)\) can be constructed as
\begin{equation*}
    \textbf{x} =  \langle b_1-a_1, \dots, b_n - a_n \rangle \,.
\end{equation*}

We denote the above vector as \(\vec{AB}\) where \(A\) is the tail (initial point)
and \(B\) is the tip/head (terminal point).
We denote \(\textbf{0}\) to be the zero vector, i.e.,
\begin{equation*}
    \textbf{0} = \langle 0, \dots, 0 \rangle \,.
\end{equation*}
\end{remark}

\begin{definition}
The length of a vector \(\textbf{v}\) (denoted by \(| \textbf{v}|\)) is defined to be
\begin{equation}
    |\textbf{v}| = \sqrt{ v_1^2 + \dots + v_n^2} \,.
\end{equation}
\end{definition}

\begin{definition}
A unit vector is a vector that has magnitude 1.
\end{definition}

\begin{exercise}
Turn a vector \(\textbf{v} \in \mathbb{R}^n\) into a unit vector with the same
direction.
\end{exercise}

\section*{Rules to manipulate vectors}\label{rules-to-manipulate-vectors}


Let \(\textbf{a}, \textbf{b} \in \mathbb{R}^n\) and \(c,d \in \mathbb{R}\). Then,

\begin{equation*}
    c( \textbf{a} + \textbf{b}) = \langle c a_1 + c b_1, \dots, c a_n + c b_n \rangle  
    = c\textbf{a} + c\textbf{b} \,,
\end{equation*}
and
\begin{equation*}
 (c+d) \textbf{a} = c\mathbf{a} + d\mathbf{a} \,.
\end{equation*}

These formulas are deceptively simple. Make sure you understand all the implications.

Because of this rule, sometimes it is good to write vectors in terms of elementary vectors:
\begin{equation*}
    \mathbf{u} = u_1 \mathbf{e_1} + \dots + u_n \mathbf{e_n} \,,
\end{equation*}
where
\(e_i = \langle 0,\dots, 1, \dots, 0\rangle\) is the vector which has zero at all entries
except that the \(i^{th}\) entry is 1.

In 3D,
\begin{equation*}
    \mathbf{e_1} = \mathbf{i} \,, \qquad 
    \mathbf{e_2} = \mathbf{j} \,, \qquad
    \mathbf{e_3} = \mathbf{k} \,.
\end{equation*}

\section*{Properties of vector operations}\label{properties-of-vector-operations}


Read the book

(Make sure you understand the geometric intepretation)

\section{Products}\label{products}

\subsection{Dot product}\label{dot-product}

\begin{itemize}
\item
  How is the dot product of two vectors defined and what geometric information does it tell us?
\item
  How can we tell if two vectors in \(\mathbb{R}^n\)
  are perpendicular?
\item
  How do we find the projection of one vector onto another?
\end{itemize}

\begin{definition}
The dot product of vectors \(\textbf{u} = \langle u_1, \dots, u_n \rangle\)
and \(\textbf{v} = \langle v_1, \dots, v_n \rangle\) in \(\mathbb{R}^n\) is the
scalar
\begin{equation*}
    \textbf{u} \cdot \textbf{v} = u_1 v_1 +\dots + u_n v_n \,.
\end{equation*}
\end{definition}

\subsection*{Properties of dot product}\label{properties-of-dot-product}


Let \(\textbf{u}, \textbf{v}, \textbf{w} \in \mathbb{R}^n\). Then,

\begin{enumerate}
\def\labelenumi{\arabic{enumi}.}
\item
  \(\textbf{u}\cdot \textbf{v} = \textbf{v}\cdot \textbf{u}\),
\item
  \((\textbf{u} + \textbf{v})\cdot \textbf{w} = (\textbf{u}\cdot \textbf{w}) + (\textbf{v}\cdot \textbf{w})\),
\item
  If \(c\) is a scalar, then \((c \textbf{u})\cdot \textbf{w} = c (\textbf{u}\cdot \textbf{w})\).
\end{enumerate}

\begin{theorem}[Law of cosine]
If \(\theta\) is the angle between the vectors \(\textbf{u}\) and \(\textbf{v}\), then
\begin{equation*}
        \textbf{u}\cdot \textbf{v} = |\textbf{u}|| \textbf{v}| \cos \theta \,.
   \end{equation*}
\end{theorem}

\begin{corollary}
Two vectors \(\textbf{u}\) and \(\textbf{v}\) are orthogonal to each other
if \(\textbf{u} \cdot \textbf{v} = 0\).
\end{corollary}

\subsection*{Projection}\label{projection}


Let \(\textbf{u}, \textbf{v}\in \mathbb{R}^n\). The component of \(\textbf{u}\)
in the direction of \(\textbf{v}\) is the scalar
\begin{equation*}
\mathrm{comp}_{\mathbf{v}}\mathbf{u} = \frac{\mathbf{u}\cdot \mathbf{v}}{|\mathbf{v}|} \,,
\end{equation*}
and the projection of \(\mathbf{u}\) onto \(\mathbf{v}\) is the vector
\begin{equation*}
    \mathrm{proj}_{\mathbf{v}}\mathbf{u} 
    =\left( \mathbf{u}\cdot \frac{\mathbf{v}}{|\mathbf{v}|}\right) \frac{\mathbf{v}}{|\mathbf{v}|} 
    = \frac{\mathbf{u}\cdot \mathbf{v}}{\mathbf{v} \cdot\mathbf{v}} \mathbf{v} \,.
\end{equation*}

Read the book for more details.
Make sure you understand the geometric meaning.

\subsection{3D special: Cross product}\label{d-special-cross-product}

This concept is very specific to \(\mathbb{R}^3\).
It will not make sense in other dimensions.

\begin{definition}
Let \(\mathbf{a}, \mathbf{b} \in \mathbb{R}^3\).
The cross product of \(\mathbf{a}\) and \(\mathbf{b}\) is defined to be
\begin{equation*}
    \mathbf{a} \times \mathbf{b} = \langle a_2 b_3 - a_3 b_2, a_3b_1 - a_1 b_3, a_1b_2 - a_2b_1 \rangle \,.
\end{equation*}
\end{definition}

\begin{theorem}
Let \(\theta\) be the angle between \(\mathbf{a}\) and \(\mathbf{b}\). Then,
\begin{equation*}
    | \mathbf{a} \times \mathbf{b} | = |\mathbf{a}||\mathbf{b}| \sin\theta \,.
\end{equation*}
\end{theorem}

\begin{theorem}
The vector \(\mathbf{a}\times \mathbf{b}\) is orthogonal to both \(\mathbf{a}\) and \(\mathbf{b}\).
\end{theorem}

\end{document}
