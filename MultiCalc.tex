% Options for packages loaded elsewhere
\PassOptionsToPackage{unicode}{hyperref}
\PassOptionsToPackage{hyphens}{url}
%
\documentclass[
]{article}
\usepackage{lmodern}
\usepackage{amssymb,amsmath}
\usepackage{ifxetex,ifluatex}
\ifnum 0\ifxetex 1\fi\ifluatex 1\fi=0 % if pdftex
  \usepackage[T1]{fontenc}
  \usepackage[utf8]{inputenc}
  \usepackage{textcomp} % provide euro and other symbols
\else % if luatex or xetex
  \usepackage{unicode-math}
  \defaultfontfeatures{Scale=MatchLowercase}
  \defaultfontfeatures[\rmfamily]{Ligatures=TeX,Scale=1}
\fi
% Use upquote if available, for straight quotes in verbatim environments
\IfFileExists{upquote.sty}{\usepackage{upquote}}{}
\IfFileExists{microtype.sty}{% use microtype if available
  \usepackage[]{microtype}
  \UseMicrotypeSet[protrusion]{basicmath} % disable protrusion for tt fonts
}{}
\makeatletter
\@ifundefined{KOMAClassName}{% if non-KOMA class
  \IfFileExists{parskip.sty}{%
    \usepackage{parskip}
  }{% else
    \setlength{\parindent}{0pt}
    \setlength{\parskip}{6pt plus 2pt minus 1pt}}
}{% if KOMA class
  \KOMAoptions{parskip=half}}
\makeatother
\usepackage{xcolor}
\IfFileExists{xurl.sty}{\usepackage{xurl}}{} % add URL line breaks if available
\IfFileExists{bookmark.sty}{\usepackage{bookmark}}{\usepackage{hyperref}}
\hypersetup{
  pdftitle={MATH 104: Multivariable Calculus (brief notes)},
  pdfauthor={Truong-Son Van},
  hidelinks,
  pdfcreator={LaTeX via pandoc}}
\urlstyle{same} % disable monospaced font for URLs
\usepackage[margin=1in]{geometry}
\usepackage{longtable,booktabs}
% Correct order of tables after \paragraph or \subparagraph
\usepackage{etoolbox}
\makeatletter
\patchcmd\longtable{\par}{\if@noskipsec\mbox{}\fi\par}{}{}
\makeatother
% Allow footnotes in longtable head/foot
\IfFileExists{footnotehyper.sty}{\usepackage{footnotehyper}}{\usepackage{footnote}}
\makesavenoteenv{longtable}
\usepackage{graphicx}
\makeatletter
\def\maxwidth{\ifdim\Gin@nat@width>\linewidth\linewidth\else\Gin@nat@width\fi}
\def\maxheight{\ifdim\Gin@nat@height>\textheight\textheight\else\Gin@nat@height\fi}
\makeatother
% Scale images if necessary, so that they will not overflow the page
% margins by default, and it is still possible to overwrite the defaults
% using explicit options in \includegraphics[width, height, ...]{}
\setkeys{Gin}{width=\maxwidth,height=\maxheight,keepaspectratio}
% Set default figure placement to htbp
\makeatletter
\def\fps@figure{htbp}
\makeatother
\setlength{\emergencystretch}{3em} % prevent overfull lines
\providecommand{\tightlist}{%
  \setlength{\itemsep}{0pt}\setlength{\parskip}{0pt}}
\setcounter{secnumdepth}{5}

\title{MATH 104: Multivariable Calculus (brief notes)}
\author{Truong-Son Van}
\date{}

\usepackage{amsthm}
\newtheorem{theorem}{Theorem}[section]
\newtheorem{lemma}{Lemma}[section]
\newtheorem{corollary}{Corollary}[section]
\newtheorem{proposition}{Proposition}[section]
\newtheorem{conjecture}{Conjecture}[section]
\theoremstyle{definition}
\newtheorem{definition}{Definition}[section]
\theoremstyle{definition}
\newtheorem{example}{Example}[section]
\theoremstyle{definition}
\newtheorem{exercise}{Exercise}[section]
\theoremstyle{definition}
\newtheorem{hypothesis}{Hypothesis}[section]
\theoremstyle{remark}
\newtheorem*{remark}{Remark}
\newtheorem*{solution}{Solution}
\begin{document}
\maketitle

{
\setcounter{tocdepth}{2}
\tableofcontents
}
\hypertarget{spring-2023}{%
\section*{Spring 2023}\label{spring-2023}}

\newpage

\newcommand{\vectorproj}[2][]{\mathrm{proj}_{\vect{#1}}\vect{#2}}
\newcommand{\vectorcomp}[2][]{\mathrm{comp}_{\vect{#1}}\vect{#2}}
\newcommand{\vect}{\mathbf}
\newcommand{\R}{\mathbb{R}}

\hypertarget{vectors}{%
\section{Vectors}\label{vectors}}

\hypertarget{basics}{%
\subsection{Basics}\label{basics}}

\textbf{Reading: Stewart Chapter 12, Thomas Calculus Chapter 12,
Active Calculus Chapter 9}

You should be able to answer the following questions after reading this section:

\begin{itemize}
\item
  What is a vector?
\item
  What does it mean for two vectors to be equal?
\item
  How do we add two vectors together and multiply a vector by a scalar?
\item
  How do we determine the magnitude of a vector?
\item
  What is a unit vector
\item
  How do we find a unit vector in the direction of a given vector?
\end{itemize}

Typically, we talk about 3-dimensional vectors (as discussed in Stewart and Thomas).
However, since talking about \(n\)-dimensional vectors doesn't
require much more effort,
we will talk about \(n\)-dimensional vectors instead.

\begin{definition}
An \(n\)-dimensional Euclidean space \(\mathbb{R}^n\)
is the Cartesian product of \(n\) Euclidean spaces \(\mathbb{R}\).
\end{definition}

\begin{definition}
An \(n\)-dimensional vector \(\textbf{v}\in \mathbb{R}^n\) is a tuple
\begin{equation}
    \textbf{v} = \langle v_1,\dots, v_n \rangle \,,
\end{equation}
where \(v_i \in \mathbb{R}\).
\end{definition}

In dimensions less than or equal to 3, we represent a vector
geometrically by an arrow, whose length represents the magnitude.

\begin{remark}
A point in \(\mathbb{R}^n\) is also represented by an \(n\)-tuple
but with round brackets.
A vector connecting two points \(A= (a_1, \dots, a_n)\)
and \(B=(b_1, \dots, b_n)\) can be constructed as
\begin{equation*}
    \textbf{x} =  \langle b_1-a_1, \dots, b_n - a_n \rangle \,.
\end{equation*}

We denote the above vector as \(\vec{AB}\) where \(A\) is the tail (initial point)
and \(B\) is the tip/head (terminal point).
We denote \(\textbf{0}\) to be the zero vector, i.e.,
\begin{equation*}
    \textbf{0} = \langle 0, \dots, 0 \rangle \,.
\end{equation*}
\end{remark}

\begin{definition}
The length of a vector \(\textbf{v}\) (denoted by \(| \textbf{v}|\)) is defined to be
\begin{equation}
    |\textbf{v}| = \sqrt{ v_1^2 + \dots + v_n^2} \,.
\end{equation}
\end{definition}

\begin{definition}
A unit vector is a vector that has magnitude 1.
\end{definition}

\begin{exercise}
Turn a vector \(\textbf{v} \in \mathbb{R}^n\) into a unit vector with the same
direction.
\end{exercise}

\hypertarget{rules-to-manipulate-vectors}{%
\subsection*{Rules to manipulate vectors}\label{rules-to-manipulate-vectors}}
\addcontentsline{toc}{subsection}{Rules to manipulate vectors}

Let \(\textbf{a}, \textbf{b} \in \mathbb{R}^n\) and \(c,d \in \mathbb{R}\). Then,

\begin{equation*}
    c( \textbf{a} + \textbf{b}) = \langle c a_1 + c b_1, \dots, c a_n + c b_n \rangle  
    = c\textbf{a} + c\textbf{b} \,,
\end{equation*}
and
\begin{equation*}
 (c+d) \textbf{a} = c\mathbf{a} + d\mathbf{a} \,.
\end{equation*}

These formulas are deceptively simple. Make sure you understand all the implications.

Because of this rule, sometimes it is good to write vectors in terms of elementary vectors:
\begin{equation*}
    \mathbf{u} = u_1 \mathbf{e_1} + \dots + u_n \mathbf{e_n} \,,
\end{equation*}
where
\(e_i = \langle 0,\dots, 1, \dots, 0\rangle\) is the vector which has zero at all entries
except that the \(i^{th}\) entry is 1.

In 3D,
\begin{equation*}
    \mathbf{e_1} = \mathbf{i} \,, \qquad 
    \mathbf{e_2} = \mathbf{j} \,, \qquad
    \mathbf{e_3} = \mathbf{k} \,.
\end{equation*}

\hypertarget{properties-of-vector-operations}{%
\subsection*{Properties of vector operations}\label{properties-of-vector-operations}}
\addcontentsline{toc}{subsection}{Properties of vector operations}

Read the book

(Make sure you understand the geometric intepretation)

\hypertarget{products}{%
\subsection{Products}\label{products}}

\hypertarget{dot-product}{%
\subsubsection{Dot product}\label{dot-product}}

\begin{itemize}
\item
  How is the dot product of two vectors defined and what geometric information does it tell us?
\item
  How can we tell if two vectors in \(\mathbb{R}^n\)
  are perpendicular?
\item
  How do we find the projection of one vector onto another?
\end{itemize}

\begin{definition}
The dot product of vectors \(\textbf{u} = \langle u_1, \dots, u_n \rangle\)
and \(\textbf{v} = \langle v_1, \dots, v_n \rangle\) in \(\mathbb{R}^n\) is the
scalar
\begin{equation*}
    \textbf{u} \cdot \textbf{v} = u_1 v_1 +\dots + u_n v_n \,.
\end{equation*}
\end{definition}

\hypertarget{properties-of-dot-product}{%
\subsubsection*{Properties of dot product}\label{properties-of-dot-product}}
\addcontentsline{toc}{subsubsection}{Properties of dot product}

Let \(\textbf{u}, \textbf{v}, \textbf{w} \in \mathbb{R}^n\). Then,

\begin{enumerate}
\def\labelenumi{\arabic{enumi}.}
\item
  \(\textbf{u}\cdot \textbf{v} = \textbf{v}\cdot \textbf{u}\),
\item
  \((\textbf{u} + \textbf{v})\cdot \textbf{w} = (\textbf{u}\cdot \textbf{w}) + (\textbf{v}\cdot \textbf{w})\),
\item
  If \(c\) is a scalar, then \((c \textbf{u})\cdot \textbf{w} = c (\textbf{u}\cdot \textbf{w})\).
\end{enumerate}

\begin{theorem}[Law of cosine]
If \(\theta\) is the angle between the vectors \(\textbf{u}\) and \(\textbf{v}\), then
\begin{equation*}
        \textbf{u}\cdot \textbf{v} = |\textbf{u}|| \textbf{v}| \cos \theta \,.
   \end{equation*}
\end{theorem}

\begin{corollary}
Two vectors \(\textbf{u}\) and \(\textbf{v}\) are orthogonal to each other
if \(\textbf{u} \cdot \textbf{v} = 0\).
\end{corollary}

\hypertarget{projection}{%
\subsubsection*{Projection}\label{projection}}
\addcontentsline{toc}{subsubsection}{Projection}

Let \(\textbf{u}, \textbf{v}\in \mathbb{R}^n\). The component of \(\textbf{u}\)
in the direction of \(\textbf{v}\) is the scalar
\begin{equation*}
\mathrm{comp}_{\mathbf{v}}\mathbf{u} = \frac{\mathbf{u}\cdot \mathbf{v}}{|\mathbf{v}|} \,,
\end{equation*}
and the projection of \(\mathbf{u}\) onto \(\mathbf{v}\) is the vector
\begin{equation*}
    \mathrm{proj}_{\mathbf{v}}\mathbf{u} 
    =\left( \mathbf{u}\cdot \frac{\mathbf{v}}{|\mathbf{v}|}\right) \frac{\mathbf{v}}{|\mathbf{v}|} 
    = \frac{\mathbf{u}\cdot \mathbf{v}}{\mathbf{v} \cdot\mathbf{v}} \mathbf{v} \,.
\end{equation*}

Read the book for more details.
Make sure you understand the geometric meaning.

\hypertarget{d-special-cross-product}{%
\subsubsection{3D special: Cross product}\label{d-special-cross-product}}

This concept is very specific to \(\mathbb{R}^3\).
It will not make sense in other dimensions.

\begin{definition}
Let \(\mathbf{a}, \mathbf{b} \in \mathbb{R}^3\).
The cross product of \(\mathbf{a}\) and \(\mathbf{b}\) is defined to be
\begin{equation*}
    \mathbf{a} \times \mathbf{b} = \langle a_2 b_3 - a_3 b_2, a_3b_1 - a_1 b_3, a_1b_2 - a_2b_1 \rangle \,.
\end{equation*}
\end{definition}

\begin{theorem}
Let \(\theta\) be the angle between \(\mathbf{a}\) and \(\mathbf{b}\). Then,
\begin{equation*}
    | \mathbf{a} \times \mathbf{b} | = |\mathbf{a}||\mathbf{b}| \sin\theta \,.
\end{equation*}
\end{theorem}

\begin{theorem}
The vector \(\mathbf{a}\times \mathbf{b}\) is orthogonal to both \(\mathbf{a}\) and \(\mathbf{b}\).
\end{theorem}

\newpage

\hypertarget{some-basic-equations-in-mathbbr3}{%
\section{\texorpdfstring{Some basic equations in \(\mathbb{R}^3\)}{Some basic equations in \textbackslash mathbb\{R\}\^{}3}}\label{some-basic-equations-in-mathbbr3}}

Just to build some toy examples for the future, we will play with some basic
equations in three dimensions.

\hypertarget{equations-for-lines}{%
\subsection{Equations for lines}\label{equations-for-lines}}

A line is a collection of points that is parallel to a vector and goes through a
specific point.
To capture this idea, we have the following representation for a line
\begin{equation*}
    L = \{\mathbf{r}(t) \,|  \mathbf{r}(t) = \mathbf{r}_0 + t \mathbf{v}, t\in \mathbb{R}\}  \,,
\end{equation*}
where \({r}_0\) is the initial position and \(\mathbf{v}\) is the direction.
The equation for \(\mathbf{r}(t)\) is called a \textbf{vector equation for a line \(L\)}.

Let \(\mathbf{v} = \langle v_1, v_2, v_3 \rangle\) and \(\mathbf{r}_0 = ( x_0, y_0, z_0 )\).
The \textbf{parametric equations} of \(L\) is the following system of equations

\begin{gather*}
    x = x_0 + v_1 t\,, \\
    y = y_0 + v_2 t\,, \\
    z = z_0 + v_3 t \,. 
\end{gather*}

This leads to the \textbf{symmetric equations} of \(L\)

\begin{equation*}
    \frac{x - x_0}{v_1} = \frac{y - y_0}{v_2} = \frac{z - z_0}{v_3} \,.
\end{equation*}

\begin{definition}
Two lines are parallel if their directional vectors are parallel (scalar multiple of each other).

Two lines that are not parallel and don't intersect each other are said to be skew.
\end{definition}

\hypertarget{equations-for-planes}{%
\subsection{Equations for planes}\label{equations-for-planes}}

A plane is a collection of points that is perpendicular to one specific direction
represented by a some vector called a \textbf{normal vector}.
Note that due to scaling, there are more than one normal vector.
To capture this idea, we have the following representation of a plane

\begin{equation*}
    P = \{ \mathbf{r} \, | \, \mathbf{n} \cdot (\mathbf{r}- \mathbf{r}_0 ) = 0 \} \,.
\end{equation*}

This is called a \textbf{vector equation for the plane \(P\)}.

Multiplying things out, we have the \textbf{scalar equation of the plane \(P\)} with
normal vector \(\mathbf{n} = \langle n_1, n_2, n_3 \rangle\) through a point \(P_0(x_0, y_0, z_0)\)
\begin{equation*}
    n_1(r_1- x_0) + n_2 (r_2 - y_0) + n_3(r_3 - z_0) = 0 \,.
\end{equation*}

The equation of the form
\begin{equation*}
    ax + by + cz + d = 0 
\end{equation*}
is called a \textbf{linear equation}.

\begin{definition}
Two planes are said to be parallel if their normal vectors are parallel.
If two planes are not parallel, they intersect in a straight line and
the angle between the two planes is defined to be the angle between the
two normal vectors.
\end{definition}

\hypertarget{cylinders}{%
\subsection{Cylinders}\label{cylinders}}

\begin{definition}
A cylinder is a surface that consists of all lines (called \textbf{rulings}) that
are parallel to a given line.
\end{definition}

\begin{example}
\leavevmode

\begin{enumerate}
\def\labelenumi{\arabic{enumi}.}
\tightlist
\item
  \(z = x^2\)
\item
  \(x^2 + y^2 = 1\)
\end{enumerate}

\end{example}

\hypertarget{quadric-surfaces}{%
\subsection{Quadric surfaces}\label{quadric-surfaces}}

\begin{definition}
A quadric surface is the graph of a second-degree equation in three variables
\(x,y\) and \(z\).
The equation that represents these surfaces is
\[Ax^2 + By^2 + Cz^2 + Dz = E\,.\]
\end{definition}

\begin{example}
\leavevmode

\begin{enumerate}
\def\labelenumi{\arabic{enumi}.}
\item
  Ellipsoid
  \[\frac{x^2}{a^2} + \frac{y^2}{b^2} + \frac{z^2}{c^2} = 1\,. \]
\item
  Hyperbolic paraboloid
  \[\frac{y^2}{b^2} - \frac{x^2}{a^2} = \frac{z}{c} \,.\]
\item
  Elliptical cone
  \[\frac{x^2}{a^2} + \frac{y^2}{b^2} = \frac{z^2}{c^2} \,.\]
\end{enumerate}

Read the books for more surfaces and pictures.

\end{example}

\newpage

\hypertarget{functions-in-higher-dimensions}{%
\section{Functions in higher dimensions}\label{functions-in-higher-dimensions}}

\textbf{Reading: Stewart Chapter 12, 13, Thomas Calculus Chapter 12, 13, Active Calculus Chapter 9}

\hypertarget{functions-of-several-variables}{%
\subsection{Functions of several variables}\label{functions-of-several-variables}}

\begin{definition}
A function of several variables is a function
\(f: D \to C\) where \(D \subseteq \mathbb{R}^m\) and \(C \subseteq \mathbb{R}^n\), where \(m\geq 2\) and \(n\geq 1\).
\[f({x}) = ( f_1(x_1,\dots, x_m),\dots, f_n(x_1,\dots, x_m)  ) \,.\]
\(D\) is called the domain of \(f\) and \(C\) is called the codomain of \(f\).
\end{definition}

The domain of \(f\) is where each of the component \(f_i\) of \(f\) is defined.

\begin{example}

The following are some examples of multivariable functions

\begin{enumerate}
\def\labelenumi{\arabic{enumi}.}
\item
  \(f(x,y) = x^2 - 2xy + y^2\)
\item
  \(f(x,y,z) = \frac{1}{1 - xy^2}\)
\end{enumerate}

\end{example}

\hypertarget{vector-functions}{%
\subsection{Vector functions}\label{vector-functions}}

\hypertarget{limit-continuity-and-differentiation}{%
\subsubsection{Limit, continuity and differentiation}\label{limit-continuity-and-differentiation}}

The expression in the vector equation for a line is an example of a function that maps from \(\mathbb{R}\) to \(\mathbb{R}^n\).
There's no one who would stop us from considering more general kinds of function.

\begin{definition}
A \textbf{vector function} (\textbf{vector-valued function}) is a function that has the codomain that belongs to \(\mathbb{R}^n\) where \(n\geq 2\). In other words, \(f: D \to \mathbb{R}^n\).
\end{definition}

\begin{example}

The following are some examples of vector funtions.

\begin{enumerate}
\def\labelenumi{\arabic{enumi}.}
\item
  \(\mathbf{r}(t) = \mathbf{r}_0 + t\mathbf{v}\)
\item
  \(\mathbf{f}(t) = \langle \cos(t),\sin(t), t \rangle\)
\end{enumerate}

\end{example}

Note that my definition is more general than that in the book.
However,
\textbf{In this course, whenever we talk about vector valued function, we will only refer to
that which has one dimensional domain (\(D \subseteq \mathbb{R}\)).}

By and large, there's nothing different between a vector function and
a one-variable scalar function.
All the concepts such as limit, continuity and differentiability are applied
to each coordinate the same way as in one dimensional case.

\begin{theorem}
Let \(\mathbf{r}: \mathbb{R}\to \mathbb{R}^n\), given by \(\mathbf{r}(t) = \langle r_1(t), \dots , r_n(t) \rangle\).
Then, \(\mathbf{r}\) is said to be continuous at \(t_0\) if
\begin{equation*}
    \mathbf{r}(t_0) = \lim_{t\to t_0} \mathbf{r}(t) \,,
\end{equation*}
where
\begin{equation*}
    \lim_{t\to t_0} \mathbf{r}(t) = \langle \lim_{t\to t_0}r_1(t) , \dots , \lim_{t\to t_0} r_n(t) \rangle \,. 
\end{equation*}
Furthermore, we can define the derivative of \(\mathbf{r}\)
\begin{equation*}
    \frac{d}{dt} \mathbf{r}(t) = \mathbf{r}'(t) = \lim_{h\to 0} \frac{\mathbf{r}(t+h) - \mathbf{r}(t)}{h} 
\end{equation*}
if this limit exists.
\end{theorem}

When \(\mathbf{r}:I \to \mathbb{R}^n\) (\(I\) is an interval in \(\mathbb{R}\)) is continuous,
we call it a \textbf{space curve} (to describe the intuitive picture of what
a curve should look like in our mind).

Geometrically, if \(\mathbf{r}'(t)\) exists and \(\mathbf{r}'(t) \not= \mathbf{0}\), it
represents the \textbf{tangent vector} of the curve \(\mathbf{r}\) at \(t\).

\begin{definition}
A \textbf{parametric equation} for a curve is an equation of the form
\[
x=x(t)\,, \quad  y = y(t)\,, \quad z = z(t) \,.
\]
\end{definition}

Typical differentiation rules apply.

\begin{theorem}[Differentiation rules]
\leavevmode

\begin{enumerate}
\def\labelenumi{\arabic{enumi}.}
\item
  \((\mathbf{u}(t) + \mathbf{v}(t))' = \mathbf{u}'(t) + \mathbf{v}'(t)\)
\item
  \((c \mathbf{u}(t))' = c \mathbf{u}'(t)\)
\item
  \((f(t) \mathbf{u}(t))' = f'(t) \mathbf{u}(t) + f(t) \mathbf{u}'(t)\)
\item
  \((\mathbf{u}(t) \cdot \mathbf{v}(t))' = \mathbf{u}'(t)\cdot \mathbf{v}(t) + \mathbf{u}(t)\cdot \mathbf{v}'(t)\)
\item
  \((\mathbf{u}(t) \times \mathbf{v}(t))' = \mathbf{u}'(t)\times \mathbf{v}(t) + \mathbf{u}(t)\times \mathbf{v}'(t)\)
\item
  \((\mathbf{u}(f(t)))' = \mathbf{u}'(f(t)) f'(t)\)
\end{enumerate}

\end{theorem}

\hypertarget{integrals}{%
\subsubsection{Integrals}\label{integrals}}

There are different ways to play with integrals for vector functions,
each has its own interpretation and physical applications.

\hypertarget{indefinite-integral}{%
\paragraph{Indefinite integral}\label{indefinite-integral}}

\begin{equation*}
    \int_a^b \mathbf{r}(t) \, dt = \left\langle \int_a^b r_1(t) \, dt, \int_a^b r_2(t) \, dt, \int_a^b r_3(t) \, dt \right\rangle
\end{equation*}

\hypertarget{arc-length-and-curvature}{%
\paragraph{Arc Length and curvature}\label{arc-length-and-curvature}}

\begin{definition}
The length the curve \(\mathbf{r}:[a,b] \to \mathbb{R}^n\) is defined to be
\begin{equation*}
L = \int_a^b \left| \mathbf{r}'(t) \right| \, dt \,.
\end{equation*}
\end{definition}

If one wants to keep track the length of the curve \(\mathbf{r}:[a,b] \to \mathbb{R}^n\) made by an airplane
at any time \(t\), one uses the \textbf{arc length function}

\begin{equation*}
    \ell(t) = \int_a^t \left| \mathbf{r}'(u) \right| \, du \,.
\end{equation*}

\hypertarget{re-parametrize-with-respect-to-arc-length}{%
\paragraph*{Re-parametrize with respect to arc length}\label{re-parametrize-with-respect-to-arc-length}}
\addcontentsline{toc}{paragraph}{Re-parametrize with respect to arc length}

The nice thing about \(s(t)\) is that it is a strictly increasing function with respect to \(t\),
given that \(\mathbf{r}'\) is non-zero for all \(t\).
Therefore, we can talk about the inverse of \(\ell\), \(\ell^{-1}:[0,L] \to [a,b]\)
\begin{equation*}
    t = \ell^{-1}(s) \,.
\end{equation*}
Therefore, we can re-write
\begin{equation*}
\mathbf{r}(t) = \mathbf{r}(\ell^{-1}(s)) \,.
\end{equation*}

\begin{theorem}
\[\left| \frac{d r(t)}{ds} \right| = 1 \,.\]
Thus,
\[l(s) = \int_0^s \left| \frac{d}{ds} \mathbf{r}(t) \right| \, dt = s \,.\]
\end{theorem}

\end{document}
