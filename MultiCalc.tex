% Options for packages loaded elsewhere
\PassOptionsToPackage{unicode}{hyperref}
\PassOptionsToPackage{hyphens}{url}
\documentclass[
]{book}
\usepackage{xcolor}
\usepackage[margin=1in]{geometry}
\usepackage{amsmath,amssymb}
\setcounter{secnumdepth}{5}
\usepackage{iftex}
\ifPDFTeX
  \usepackage[T1]{fontenc}
  \usepackage[utf8]{inputenc}
  \usepackage{textcomp} % provide euro and other symbols
\else % if luatex or xetex
  \usepackage{unicode-math} % this also loads fontspec
  \defaultfontfeatures{Scale=MatchLowercase}
  \defaultfontfeatures[\rmfamily]{Ligatures=TeX,Scale=1}
\fi
\usepackage{lmodern}
\ifPDFTeX\else
  % xetex/luatex font selection
\fi
% Use upquote if available, for straight quotes in verbatim environments
\IfFileExists{upquote.sty}{\usepackage{upquote}}{}
\IfFileExists{microtype.sty}{% use microtype if available
  \usepackage[]{microtype}
  \UseMicrotypeSet[protrusion]{basicmath} % disable protrusion for tt fonts
}{}
\makeatletter
\@ifundefined{KOMAClassName}{% if non-KOMA class
  \IfFileExists{parskip.sty}{%
    \usepackage{parskip}
  }{% else
    \setlength{\parindent}{0pt}
    \setlength{\parskip}{6pt plus 2pt minus 1pt}}
}{% if KOMA class
  \KOMAoptions{parskip=half}}
\makeatother
\usepackage{longtable,booktabs,array}
\usepackage{calc} % for calculating minipage widths
% Correct order of tables after \paragraph or \subparagraph
\usepackage{etoolbox}
\makeatletter
\patchcmd\longtable{\par}{\if@noskipsec\mbox{}\fi\par}{}{}
\makeatother
% Allow footnotes in longtable head/foot
\IfFileExists{footnotehyper.sty}{\usepackage{footnotehyper}}{\usepackage{footnote}}
\makesavenoteenv{longtable}
\usepackage{graphicx}
\makeatletter
\newsavebox\pandoc@box
\newcommand*\pandocbounded[1]{% scales image to fit in text height/width
  \sbox\pandoc@box{#1}%
  \Gscale@div\@tempa{\textheight}{\dimexpr\ht\pandoc@box+\dp\pandoc@box\relax}%
  \Gscale@div\@tempb{\linewidth}{\wd\pandoc@box}%
  \ifdim\@tempb\p@<\@tempa\p@\let\@tempa\@tempb\fi% select the smaller of both
  \ifdim\@tempa\p@<\p@\scalebox{\@tempa}{\usebox\pandoc@box}%
  \else\usebox{\pandoc@box}%
  \fi%
}
% Set default figure placement to htbp
\def\fps@figure{htbp}
\makeatother
\setlength{\emergencystretch}{3em} % prevent overfull lines
\providecommand{\tightlist}{%
  \setlength{\itemsep}{0pt}\setlength{\parskip}{0pt}}
\usepackage[]{natbib}
\bibliographystyle{plainnat}
\usepackage{bookmark}
\IfFileExists{xurl.sty}{\usepackage{xurl}}{} % add URL line breaks if available
\urlstyle{same}
\hypersetup{
  pdftitle={MATH 104: Multivariable Calculus (brief notes)},
  pdfauthor={Truong-Son Van},
  hidelinks,
  pdfcreator={LaTeX via pandoc}}

\title{MATH 104: Multivariable Calculus (brief notes)}
\author{Truong-Son Van}
\date{}

\usepackage{amsthm}
\newtheorem{theorem}{Theorem}[chapter]
\newtheorem{lemma}{Lemma}[chapter]
\newtheorem{corollary}{Corollary}[chapter]
\newtheorem{proposition}{Proposition}[chapter]
\newtheorem{conjecture}{Conjecture}[chapter]
\theoremstyle{definition}
\newtheorem{definition}{Definition}[chapter]
\theoremstyle{definition}
\newtheorem{example}{Example}[chapter]
\theoremstyle{definition}
\newtheorem{exercise}{Exercise}[chapter]
\theoremstyle{definition}
\newtheorem{hypothesis}{Hypothesis}[chapter]
\theoremstyle{remark}
\newtheorem*{remark}{Remark}
\newtheorem*{solution}{Solution}
\begin{document}
\maketitle

{
\setcounter{tocdepth}{2}
\tableofcontents
}
\chapter*{Spring 2025}\label{spring-2025}

\chapter*{Disclaimer}\label{disclaimer}


This is class notes for Multivariable Calculus at Fublbright University Vietnam.
I claim no originality in this work as it is mostly taken from the reference books.
However, all errors and typos are solely mine.

\newpage

\newcommand{\vectorproj}[2][]{\mathrm{proj}_{\vect{#1}}\vect{#2}}
\newcommand{\vectorcomp}[2][]{\mathrm{comp}_{\vect{#1}}\vect{#2}}
\newcommand{\vect}{\mathbf}
\newcommand{\R}{\mathbb{R}}

\chapter{Vectors \& Matrices}\label{vectors-matrices}

\section{Basics}\label{basics}

\textbf{Reading: Stewart Chapter 12, Thomas Calculus Chapter 12,
Active Calculus Chapter 9}

You should be able to answer the following questions after reading this section:

\begin{itemize}
\item
  What is a vector?
\item
  What does it mean for two vectors to be equal?
\item
  How do we add two vectors together and multiply a vector by a scalar?
\item
  How do we determine the magnitude of a vector?
\item
  What is a unit vector
\item
  How do we find a unit vector in the direction of a given vector?
\end{itemize}

Typically, we talk about 3-dimensional vectors (as discussed in Stewart and Thomas).
However, since talking about \(n\)-dimensional vectors doesn't
require much more effort,
we will talk about \(n\)-dimensional vectors instead.

\begin{definition}
An \(n\)-dimensional Euclidean space \(\mathbb{R}^n\)
is the Cartesian product of \(n\) Euclidean spaces \(\mathbb{R}\).
\end{definition}

\begin{definition}
An \(n\)-dimensional vector \(\textbf{v}\in \mathbb{R}^n\) is a tuple
\begin{equation}
    \textbf{v} = \langle v_1,\dots, v_n \rangle \,,
\end{equation}
where \(v_i \in \mathbb{R}\).
\end{definition}

In dimensions less than or equal to 3, we represent a vector
geometrically by an arrow, whose length represents the magnitude.

\begin{remark}
A point in \(\mathbb{R}^n\) is also represented by an \(n\)-tuple
but with round brackets.
A vector connecting two points \(A= (a_1, \dots, a_n)\)
and \(B=(b_1, \dots, b_n)\) can be constructed as
\begin{equation*}
    \textbf{x} =  \langle b_1-a_1, \dots, b_n - a_n \rangle \,.
\end{equation*}

We denote the above vector as \(\vec{AB}\) where \(A\) is the tail (initial point)
and \(B\) is the tip/head (terminal point).
We denote \(\textbf{0}\) to be the zero vector, i.e.,
\begin{equation*}
    \textbf{0} = \langle 0, \dots, 0 \rangle \,.
\end{equation*}
\end{remark}

\begin{definition}
The length of a vector \(\textbf{v}\) (denoted by \(| \textbf{v}|\)) is defined to be
\begin{equation}
    |\textbf{v}| = \sqrt{ v_1^2 + \dots + v_n^2} \,.
\end{equation}
\end{definition}

\begin{definition}
A unit vector is a vector that has magnitude 1.
\end{definition}

\begin{exercise}
Turn a vector \(\textbf{v} \in \mathbb{R}^n\) into a unit vector with the same
direction.
\end{exercise}

\section*{Rules to manipulate vectors}\label{rules-to-manipulate-vectors}


Let \(\textbf{a}, \textbf{b} \in \mathbb{R}^n\) and \(c,d \in \mathbb{R}\). Then,

\begin{equation*}
    c( \textbf{a} + \textbf{b}) = \langle c a_1 + c b_1, \dots, c a_n + c b_n \rangle  
    = c\textbf{a} + c\textbf{b} \,,
\end{equation*}
and
\begin{equation*}
 (c+d) \textbf{a} = c\mathbf{a} + d\mathbf{a} \,.
\end{equation*}

These formulas are deceptively simple. Make sure you understand all the implications.

Because of this rule, sometimes it is good to write vectors in terms of elementary vectors:
\begin{equation*}
    \mathbf{u} = u_1 \mathbf{e_1} + \dots + u_n \mathbf{e_n} \,,
\end{equation*}
where
\(e_i = \langle 0,\dots, 1, \dots, 0\rangle\) is the vector which has zero at all entries
except that the \(i^{th}\) entry is 1.

In 3D,
\begin{equation*}
    \mathbf{e_1} = \mathbf{i} \,, \qquad 
    \mathbf{e_2} = \mathbf{j} \,, \qquad
    \mathbf{e_3} = \mathbf{k} \,.
\end{equation*}

\section*{Properties of vector operations}\label{properties-of-vector-operations}


Read the book

(Make sure you understand the geometric intepretation)

\section{Products}\label{products}

\subsection{Dot product}\label{dot-product}

\begin{itemize}
\item
  How is the dot product of two vectors defined and what geometric information does it tell us?
\item
  How can we tell if two vectors in \(\mathbb{R}^n\)
  are perpendicular?
\item
  How do we find the projection of one vector onto another?
\end{itemize}

\begin{definition}
The dot product of vectors \(\textbf{u} = \langle u_1, \dots, u_n \rangle\)
and \(\textbf{v} = \langle v_1, \dots, v_n \rangle\) in \(\mathbb{R}^n\) is the
scalar
\begin{equation*}
    \textbf{u} \cdot \textbf{v} = u_1 v_1 +\dots + u_n v_n \,.
\end{equation*}
\end{definition}

\subsection*{Properties of dot product}\label{properties-of-dot-product}


Let \(\textbf{u}, \textbf{v}, \textbf{w} \in \mathbb{R}^n\). Then,

\begin{enumerate}
\def\labelenumi{\arabic{enumi}.}
\item
  \(\textbf{u}\cdot \textbf{v} = \textbf{v}\cdot \textbf{u}\),
\item
  \((\textbf{u} + \textbf{v})\cdot \textbf{w} = (\textbf{u}\cdot \textbf{w}) + (\textbf{v}\cdot \textbf{w})\),
\item
  If \(c\) is a scalar, then \((c \textbf{u})\cdot \textbf{w} = c (\textbf{u}\cdot \textbf{w})\).
\end{enumerate}

\begin{theorem}[Law of cosine]
If \(\theta\) is the angle between the vectors \(\textbf{u}\) and \(\textbf{v}\), then
\begin{equation*}
        \textbf{u}\cdot \textbf{v} = |\textbf{u}|| \textbf{v}| \cos \theta \,.
   \end{equation*}
\end{theorem}

\begin{corollary}
Two vectors \(\textbf{u}\) and \(\textbf{v}\) are orthogonal to each other
if \(\textbf{u} \cdot \textbf{v} = 0\).
\end{corollary}

\subsection*{Projection}\label{projection}


Let \(\textbf{u}, \textbf{v}\in \mathbb{R}^n\). The component of \(\textbf{u}\)
in the direction of \(\textbf{v}\) is the scalar
\begin{equation*}
\mathrm{comp}_{\mathbf{v}}\mathbf{u} = \frac{\mathbf{u}\cdot \mathbf{v}}{|\mathbf{v}|} \,,
\end{equation*}
and the projection of \(\mathbf{u}\) onto \(\mathbf{v}\) is the vector
\begin{equation*}
    \mathrm{proj}_{\mathbf{v}}\mathbf{u} 
    =\left( \mathbf{u}\cdot \frac{\mathbf{v}}{|\mathbf{v}|}\right) \frac{\mathbf{v}}{|\mathbf{v}|} 
    = \frac{\mathbf{u}\cdot \mathbf{v}}{\mathbf{v} \cdot\mathbf{v}} \mathbf{v} \,.
\end{equation*}

\subsection{3D special: Cross product}\label{d-special-cross-product}

This concept is very specific to \(\mathbb{R}^3\).
It will not make sense in other dimensions.

\begin{definition}
Let \(\mathbf{a}, \mathbf{b} \in \mathbb{R}^3\).
The cross product of \(\mathbf{a}\) and \(\mathbf{b}\) is defined to be
\begin{equation*}
    \mathbf{a} \times \mathbf{b} = \langle a_2 b_3 - a_3 b_2, a_3b_1 - a_1 b_3, a_1b_2 - a_2b_1 \rangle \,.
\end{equation*}
\end{definition}

\begin{theorem}
Let \(\theta\) be the angle between \(\mathbf{a}\) and \(\mathbf{b}\). Then,
\begin{equation*}
    | \mathbf{a} \times \mathbf{b} | = |\mathbf{a}||\mathbf{b}| \sin\theta \,.
\end{equation*}
\end{theorem}

\begin{theorem}
The vector \(\mathbf{a}\times \mathbf{b}\) is orthogonal to both \(\mathbf{a}\) and \(\mathbf{b}\).
\end{theorem}

\subsection{Distance from a point}\label{distance-from-a-point}

We can use the cross and dot products to measure the distance of one point to either a
plane or a line.

Let \(P \in \mathbb{R}^n\) and \(\vec{r}(t) = R_0 + t \vec{v}\) be a line.
Then the distance from \(P\) to \(\vec{r}(t)\) is
\[ Dist = \frac{| \vec{R_0 P} \times \vec{v}|}{| \vec{v} |}\]

\section{Matrices}\label{matrices}

A matrix is an 2 dimensional array with rows and columns.

\[ A = \begin{pmatrix} A_{11} & \dots & A_{1n}\\ \vdots & & \vdots \\ A_{n1} & \dots & A_{nn}  \end{pmatrix}\]

Another way to write out matrix \(A\) is
\[ A = (A_{ij})\]
where the first index \(i\) represents the row and the second index \(j\) represents the column.

\subsection{Operations on matrices}\label{operations-on-matrices}

\begin{enumerate}
\def\labelenumi{\arabic{enumi}.}
\item
  Addition: let \(A\) and \(B\) be two matrices with same dimension \(m\times n\).
  Then \(A + B\) is an \(m\times n\) matrix such that
  \[[A + B]_{ij} = A_{ij} + B_{ij}.\]
\item
  Scalar multiplication: let \(A\) be a \(m\times n\) matrix,
  \(c\) is a constant scalar.
  then \(cA\) is a \(m\times n\) matrix such that
  \[((cA)_{ij}) = (cA_{ij}).\]
\item
  Matrix multiplication: let \(A\) be \(m\times n\) matrix and \(B\) be \(n\times l\) matrix.
  Then the multiplication \(AB\) is a \(m\times l\) matrix such that
  \[ [AB]_{ij}  =  \sum_{k} A_{ik} B_{kj} .\]
\end{enumerate}

\subsection{Linear transformation}\label{linear-transformation}

A linear transformation is a function \(f: \mathbb{R}^n \to \mathbb{R}^m\) such that
\[ f(a \vec{u} + b \vec{v} ) = a f(\vec{u}) + b f(\vec{v}) \]
for all \(a,b \in \mathbb{R}\) and \(u,v \in \mathbb{R}^n\).

It turns out that every linear transformation \(f: \mathbb{R}^n \to \mathbb{R}^m\)
can be represented as a \(m\times n\) matrix.

\newpage

\chapter{\texorpdfstring{Some basic equations in \(\mathbb{R}^3\)}{Some basic equations in \textbackslash mathbb\{R\}\^{}3}}\label{some-basic-equations-in-mathbbr3}

Just to build some toy examples for the future, we will play with some basic
equations in three dimensions.

\section{Equations for lines}\label{equations-for-lines}

A line is a collection of points that is parallel to a vector and goes through a
specific point.
To capture this idea, we have the following representation for a line
\begin{equation*}
    L = \{\mathbf{r}(t) \,|  \mathbf{r}(t) = \mathbf{r}_0 + t \mathbf{v}, t\in \mathbb{R}\}  \,,
\end{equation*}
where \({r}_0\) is the initial position and \(\mathbf{v}\) is the direction.
The equation for \(\mathbf{r}(t)\) is called a \textbf{vector equation for a line \(L\)}.

Let \(\mathbf{v} = \langle v_1, v_2, v_3 \rangle\) and \(\mathbf{r}_0 = ( x_0, y_0, z_0 )\).
The \textbf{parametric equations} of \(L\) is the following system of equations

\begin{gather*}
    x = x_0 + v_1 t\,, \\
    y = y_0 + v_2 t\,, \\
    z = z_0 + v_3 t \,. 
\end{gather*}

This leads to the \textbf{symmetric equations} of \(L\)

\begin{equation*}
    \frac{x - x_0}{v_1} = \frac{y - y_0}{v_2} = \frac{z - z_0}{v_3} \,.
\end{equation*}

\begin{definition}
Two lines are parallel if their directional vectors are parallel (scalar multiple of each other).

Two lines that are not parallel and don't intersect each other are said to be skew.
\end{definition}

\section{Equations for planes}\label{equations-for-planes}

A plane is a collection of points that is perpendicular to one specific direction
represented by a some vector called a \textbf{normal vector}.
Note that due to scaling, there are more than one normal vector.
To capture this idea, we have the following representation of a plane

\begin{equation*}
    P = \{ \mathbf{r} \, | \, \mathbf{n} \cdot (\mathbf{r}- \mathbf{r}_0 ) = 0 \} \,.
\end{equation*}

This is called a \textbf{vector equation for the plane \(P\)}.

Multiplying things out, we have the \textbf{scalar equation of the plane \(P\)} with
normal vector \(\mathbf{n} = \langle n_1, n_2, n_3 \rangle\) through a point \(P_0(x_0, y_0, z_0)\)
\begin{equation*}
    n_1(r_1- x_0) + n_2 (r_2 - y_0) + n_3(r_3 - z_0) = 0 \,.
\end{equation*}

The equation of the form
\begin{equation*}
    ax + by + cz + d = 0 
\end{equation*}
is called a \textbf{linear equation}.

\begin{definition}
Two planes are said to be parallel if their normal vectors are parallel.
If two planes are not parallel, they intersect in a straight line and
the angle between the two planes is defined to be the angle between the
two normal vectors.
\end{definition}

\section{Cylinders}\label{cylinders}

\begin{definition}
A cylinder is a surface that consists of all lines (called \textbf{rulings}) that
are parallel to a given line.
\end{definition}

\begin{example}
\leavevmode

\begin{enumerate}
\def\labelenumi{\arabic{enumi}.}
\tightlist
\item
  \(z = x^2\)
\item
  \(x^2 + y^2 = 1\)
\end{enumerate}

\end{example}

\section{Quadric surfaces}\label{quadric-surfaces}

\begin{definition}
A quadric surface is the graph of a second-degree equation in three variables
\(x,y\) and \(z\).
The equation that represents these surfaces is
\[Ax^2 + By^2 + Cz^2 + Dz = E\,.\]
\end{definition}

\begin{example}
\leavevmode

\begin{enumerate}
\def\labelenumi{\arabic{enumi}.}
\item
  Ellipsoid
  \[\frac{x^2}{a^2} + \frac{y^2}{b^2} + \frac{z^2}{c^2} = 1\,. \]
\item
  Hyperbolic paraboloid
  \[\frac{y^2}{b^2} - \frac{x^2}{a^2} = \frac{z}{c} \,.\]
\item
  Elliptical cone
  \[\frac{x^2}{a^2} + \frac{y^2}{b^2} = \frac{z^2}{c^2} \,.\]
\end{enumerate}

Read the books for more surfaces and pictures.

\end{example}

\newpage

\chapter{Functions in higher dimensions}\label{functions-in-higher-dimensions}

\textbf{Reading: Stewart Chapter 12, 13, Thomas Calculus Chapter 12, 13, Active Calculus Chapter 9}

\section{Functions of several variables}\label{functions-of-several-variables}

\begin{definition}
A function of several variables is a function
\(f: D \to C\) where \(D \subseteq \mathbb{R}^m\) and \(C \subseteq \mathbb{R}^n\), where \(m\geq 2\) and \(n\geq 1\).
\[f({x}) = ( f_1(x_1,\dots, x_m),\dots, f_n(x_1,\dots, x_m)  ) \,.\]
\(D\) is called the domain of \(f\) and \(C\) is called the codomain of \(f\).
\end{definition}

The domain of \(f\) is where each of the component \(f_i\) of \(f\) is defined.

\begin{example}

The following are some examples of multivariable functions

\begin{enumerate}
\def\labelenumi{\arabic{enumi}.}
\item
  \(f(x,y) = x^2 - 2xy + y^2\)
\item
  \(f(x,y,z) = \frac{1}{1 - xy^2}\)
\end{enumerate}

\end{example}

\section{Vector functions}\label{vector-functions}

\subsection{Limit, continuity and differentiation}\label{limit-continuity-and-differentiation}

The expression in the vector equation for a line is an example of a function that maps from \(\mathbb{R}\) to \(\mathbb{R}^n\).
There's no one who would stop us from considering more general kinds of function.

\begin{definition}
A \textbf{vector function} (\textbf{vector-valued function}) is a function that has the codomain that belongs to \(\mathbb{R}^n\) where \(n\geq 2\). In other words, \(f: D  \to \mathbb{R}^n\).
\end{definition}

\begin{example}

The following are some examples of vector functions.

\begin{enumerate}
\def\labelenumi{\arabic{enumi}.}
\item
  \(\mathbf{r}(t) = \mathbf{r}_0 + t\mathbf{v}\)
\item
  \(\mathbf{f}(t) = \langle \cos(t),\sin(t), t \rangle\)
\end{enumerate}

\end{example}

Note that my definition is more general than that in the book.
However,
\textbf{In this course, whenever we talk about vector valued function, we will only refer to
that which has one dimensional domain (\(D \subseteq \mathbb{R}\)).}

By and large, there's nothing different between a vector function and
a one-variable scalar function.
All the concepts such as limit, continuity and differentiability are applied
to each coordinate the same way as in one dimensional case.

\begin{theorem}
Let \(\mathbf{r}: \mathbb{R}\to \mathbb{R}^n\), given by \(\mathbf{r}(t) = \langle r_1(t), \dots , r_n(t) \rangle\).
Then, \(\mathbf{r}\) is said to be continuous at \(t_0\) if
\begin{equation*}
    \mathbf{r}(t_0) = \lim_{t\to t_0} \mathbf{r}(t) \,,
\end{equation*}
where
\begin{equation*}
    \lim_{t\to t_0} \mathbf{r}(t) = \langle \lim_{t\to t_0}r_1(t) , \dots , \lim_{t\to t_0} r_n(t) \rangle \,. 
\end{equation*}
Furthermore, we can define the derivative of \(\mathbf{r}\)
\begin{equation*}
    \frac{d}{dt} \mathbf{r}(t) = \mathbf{r}'(t) = \lim_{h\to 0} \frac{\mathbf{r}(t+h) - \mathbf{r}(t)}{h} 
\end{equation*}
if this limit exists.
\end{theorem}

When \(\mathbf{r}:I \to \mathbb{R}^n\) (\(I\) is an interval in \(\mathbb{R}\)) is continuous,
we call it a \textbf{space curve} (to describe the intuitive picture of what
a curve should look like in our mind).

Geometrically, if \(\mathbf{r}'(t)\) exists and \(\mathbf{r}'(t) \not= \mathbf{0}\), it
represents the \textbf{tangent vector} of the curve \(\mathbf{r}\) at \(t\).

\begin{definition}
A \textbf{parametric equation} for a curve is an equation of the form
\[
x=x(t)\,, \quad  y = y(t)\,, \quad z = z(t) \,.
\]
\end{definition}

Typical differentiation rules apply.

\begin{theorem}[Differentiation rules]
\leavevmode

\begin{enumerate}
\def\labelenumi{\arabic{enumi}.}
\item
  \((\mathbf{u}(t) + \mathbf{v}(t))' = \mathbf{u}'(t) + \mathbf{v}'(t)\)
\item
  \((c \mathbf{u}(t))' = c \mathbf{u}'(t)\)
\item
  \((f(t) \mathbf{u}(t))' = f'(t) \mathbf{u}(t) + f(t) \mathbf{u}'(t)\)
\item
  \((\mathbf{u}(t) \cdot \mathbf{v}(t))' = \mathbf{u}'(t)\cdot \mathbf{v}(t) + \mathbf{u}(t)\cdot \mathbf{v}'(t)\)
\item
  \((\mathbf{u}(t) \times \mathbf{v}(t))' = \mathbf{u}'(t)\times \mathbf{v}(t) + \mathbf{u}(t)\times \mathbf{v}'(t)\)
\item
  \((\mathbf{u}(f(t)))' = \mathbf{u}'(f(t)) f'(t)\)
\end{enumerate}

\end{theorem}

\subsection{Integrals}\label{integrals}

There are different ways to play with integrals for vector functions,
each has its own interpretation and physical applications.

\subsubsection{Indefinite integral}\label{indefinite-integral}

\begin{equation*}
    \int_a^b \mathbf{r}(t) \, dt = \left\langle \int_a^b r_1(t) \, dt, \int_a^b r_2(t) \, dt, \int_a^b r_3(t) \, dt \right\rangle
\end{equation*}

\subsubsection{Arc Length and curvature}\label{arc-length-and-curvature}

\begin{definition}
The length the curve \(\mathbf{r}:[a,b] \to \mathbb{R}^n\) is defined to be
\begin{equation*}
L = \int_a^b \left| \mathbf{r}'(t) \right| \, dt \,.
\end{equation*}
\end{definition}

If one wants to keep track the length of the curve \(\mathbf{r}:[a,b] \to \mathbb{R}^n\) made by an airplane
at any time \(t\), one uses the \textbf{arc length function}

\begin{equation*}
    \ell(t) = \int_a^t \left| \mathbf{r}'(u) \right| \, du \,.
\end{equation*}

\subsubsection*{Re-parametrize with respect to arc length}\label{re-parametrize-with-respect-to-arc-length}


The nice thing about \(\ell(t)\) is that it is a strictly increasing function with respect to \(t\),
given that \(\mathbf{r}'\) is non-zero for all \(t\).
Therefore, letting \(s = \ell(t)\), we can talk about the inverse of \(\ell\), \(\ell^{-1}:[0,L] \to [a,b]\)
\begin{equation*}
    t = \ell^{-1}(s) \,.
\end{equation*}
Therefore, we can re-write
\begin{equation*}
\mathbf{r}(t) = \mathbf{r}(\ell^{-1}(s)) \,.
\end{equation*}

\begin{theorem}
\[\left| \frac{d r(t)}{ds} \right| = 1 \,.\]
Thus,
\[l(s) = \int_0^s \left| \frac{d}{ds} \mathbf{r}(t) \right| \, dt = s \,.\]
\end{theorem}

Because of the unchanging nature of the arc-length (with respect to the
parametrization),
it is used to define a geometric quantity of a space curve called \textbf{curvature}.

\begin{definition}[curvature]
Let \(\mathbf{T}(t)\) be the unit tangent vector of the curve \(\mathbf{r}:[a,b] \to \mathbb{R}^3\).
The curvature of \(\mathbf{r}(t(s))\) is defined to be
\begin{equation*}
    \kappa(s) = \left| \frac{d \mathbf{T}(t(s))}{ds}\right| \,.
\end{equation*}
\end{definition}

To convert this into the parameter \(t\), we write \(s= s(t)\) and use chain rule
to get.

\begin{theorem}
We have that
\begin{equation*}
    \kappa(s(t)) =  \frac{|\mathbf{T}'(t)|}{|\mathbf{r}'(t)|}   \,.
\end{equation*}
\end{theorem}

\subsection{\texorpdfstring{Space curve in \(\mathbb{R}^3\) and motion in space}{Space curve in \textbackslash mathbb\{R\}\^{}3 and motion in space}}\label{space-curve-in-mathbbr3-and-motion-in-space}

Read the book. This part is not required but it is so beautiful, you may want to read it
as an exercise at home (to test how much you understand what we've been discussing so far).

\section{Activity: on osculating circle and curvature}\label{activity-on-osculating-circle-and-curvature}

For those who are interested in the geometrical meaning of the curvature without having
to accept from the book that the curvature is the inverse of the radius of the osculating circle,
please take a look at \url{https://github.com/sonv/MultiCalc/blob/main/Writing/latexbuild/osculating.pdf}.

\newpage

\chapter{Partial derivatives}\label{partial-derivatives}

\section{Limits and continuity}\label{limits-and-continuity}

The following definition is from Stewart.

\begin{definition}
Let \(f: \mathbb{R}^n \to \mathbb{R}^m\) be a function. Then we say that the limit of \(f(x)\) as \(x\) approaches \(a\) is \(L\) and we write

\[\lim_{x \to a} f(x) = L\]

if for every number \(\epsilon > 0\) there is a corresponding number \(\delta > 0\) such that
\(|f(x,y) - L| < \epsilon\)
if \(| x - a| < \delta\).
\end{definition}

Finding if a function has limit as a point in higher dimension is not as simple as
the case for 1 dimension.

Determining whether a multivariable function has a limit sometimes is an art
and it requires a lot of experiences and practice.
However, there are certain rules that could help us.

\begin{theorem}

Let \(L,M\) and \(k\) be real numbers and that
\begin{equation*}
    \lim_{x \to a} f(x,y) = L \,, \qquad 
    \lim_{x \to a} g(x,y) = M \,.
\end{equation*}
We then have

\begin{enumerate}
\def\labelenumi{\arabic{enumi}.}
\item
  \(\displaystyle \lim_{x \to a} (f(x) + g(x)) = L + M\),
\item
  \(\displaystyle  \lim_{x \to a} (k f(x)) = kL\),
\item
  \(\displaystyle \lim_{x \to a} (f(x) g(x)) = LM\),
\item
  \(\displaystyle \lim_{x \to a} \frac{f(x)}{g(x)} = \frac{L}{M}\) if \(M \not= 0\),
\item
  \(\displaystyle \lim_{x \to a} {f(x)^p} = L^p\) for \(p>0\),
\end{enumerate}

\end{theorem}

\textbf{Strategy to find out that a two-variable function does NOT have a limit.}

If \(\lim_{(x,y) \to (a,b)} f(x,y) = L_1\) as \((x,y) \to (a,b)\) along a path \(C_1\),
and \(\lim_{(x,y) \to (a,b)} f(x,y) = L_2\) as \((x,y) \to (a,b)\) along a path \(C_2\),
where \(L_1 \neq L_2\), then \(\lim_{(x,y) \to (a,b)} f(x,y)\) does not exist.

\begin{example}
\(\lim_{(x,y)\to (0,0)} \frac{x^2 - y^2}{x^2 + y^2}\) does not exist.

\(\lim_{(x,y)\to (0,0)} \frac{xy}{x^2 + y^2}\) does not exist.

\(\lim_{(x,y)\to (0,0)} \frac{xy^2}{x^4 + y^4}\) does not exist.

\(\lim_{(x,y)\to (0,0)} \frac{3x^2y}{x^2 + y^2} = 0\).
\end{example}

\section{Partial derivatives}\label{partial-derivatives-1}

given a function \(f:\mathbb{R}^n \to \mathbb{R}^m\), the partial derivative with respect to the \(j\)th variable \(x_j\)
of the \(i\)th output at \(a \in \mathbb{R}^n\) is
\begin{equation*}
    \frac{ \partial }{\partial x_j} f_{i}(a) 
    = \lim_{h\to 0} \frac{ f(a_1, \dots, a_{i-1}, a_i + h , a_{i+1}, \dots, a_n) - f(a_1, \dots, a_{i-1}, a_i  , a_{i+1}, \dots, a_n)}{h} \,.
\end{equation*}

\textbf{Notations.}
The partial derivatives sometimes have different notations:
\begin{equation*}
   \partial_j f_i (a) =  
   \partial_{x_j} f_i (a) =  
    \frac{ \partial }{\partial x_j} f_{i}(a) .
\end{equation*}
From here, one can define higher partial derivatives such as the following

\begin{equation*}
    \partial^3_{1 2 3}  f_i (a) = \frac{\partial}{\partial x_1} \frac{\partial}{\partial x_2} \frac{\partial}{\partial x_3} f_i (a)\,.
\end{equation*}

Note that the power over the symbol \(\partial\) represents the order of derivatives.

\subsubsection*{Some important notations}\label{some-important-notations}


Let \(f:D \to \mathbb{R}\) be a function. We write the following, if exist,
\begin{equation*}
    \nabla f = \begin{bmatrix}
        \partial_{x_1} f\\
        \vdots \\
        \partial_{x_n} f\\
    \end{bmatrix}
\end{equation*}

\begin{equation*}
    \Delta f = \partial_{x_1}^2 f + \dots \partial_{x_n}^2 f \,.
\end{equation*}

\section{Differentiability}\label{differentiability}

\begin{definition}[Differentiability]
Let \(f:\mathbb{R}^n \to \mathbb{R}^m\).
\(f\) is said to be differentiable at \(a \in \mathbb{R}^n\) if
there exists a linear transformation \([Df]_a\) such that for every vector \(\mathbf{h} \in \mathbb{R}^n\)
\[\lim_{|\mathbf{h}| \to 0} \frac{ f(a + \mathbf{h}) - f(a) - [Df]_a \mathbf{h}}{| \mathbf{h} |} = 0 . \]
\end{definition}

For \(f:\mathbb{R}^n \to \mathbb{R}^m\), \([Df]_a\) is a \(m\times n\) matrix given by
\[[Df]_a = \left[ \frac{\partial }{\partial x_j} f_i (a)\right].\]

This is called the \emph{Jacobian matrix} of \(f\) at \(a\).

For some good intuition, please go to \url{https://mathinsight.org/differentiability_multivariable_definition}.

\begin{theorem}
Let \(f:\mathbb{R}^n \to \mathbb{R}^m\).
If the partial derivatives \(\partial_j f_i\) exist near \(a\in \mathbb{R}^n\) and are continuous
at \(a\), then \(f\) is differentiable at \(a\).
\end{theorem}

\begin{theorem}
Let \(f:\mathbb{R}^n \to \mathbb{R}^m\).
If \(f\) is differentiable at \(a\) then \(f\) is continuous at \(a\).
\end{theorem}

\section{Chain rule}\label{chain-rule}

\begin{theorem}
Let \(f: \mathbb{R}^n \to \mathbb{R}^l, g: \mathbb{R}^m \to \mathbb{R}^n\) be
differentiable functions.
Then,
\[ [D (f\circ g)]_a = [Df]_{g(a)} [Dg]_a. \]
\end{theorem}

Here's a special case

\begin{theorem}
Let \(f: \mathbb{R}^n \to \mathbb{R}, g: \mathbb{R}^m \to \mathbb{R}^n\) be
differentiable functions.
Then,
\[z(y_1, \dots, y_m) = (f\circ g)(y_1, \dots, y_m)\]
is differentiable and
\begin{equation*}
    \frac{\partial z}{\partial y_i} = \sum_{j=1}^n \frac{\partial f}{\partial x_j} \frac{\partial g_j}{\partial y_i} \,.
\end{equation*}
\end{theorem}

\section{Directional derivative}\label{directional-derivative}

\begin{definition}
Let \(\mathbf{u} \in \mathbb{R}^n\). The directional derivative of \(f:\mathbb{R}^n \to \mathbb{R}\) at \(a\in \mathbb{R}^n\)
in the direction of \(\mathbf{u}\) is the following limit (if exists)
\begin{equation*}
    D_{\mathbf{u}} f(a) = \lim_{h \to 0} \frac{ f( a + h \mathbf{u}) - f(a)}{h}\,.
\end{equation*}
\end{definition}

How can one compute directional derivative?

\begin{theorem}
If \(f:\mathbb{R}^n \to \mathbb{R}\) is differentiable then
\begin{equation*}
    D_{\mathbf{u}} f(a) = \nabla f(a) \cdot \mathbf{u} \,.
\end{equation*}
\end{theorem}

\section{Tangent planes}\label{tangent-planes}

Let's think about tangent planes in a more systematic way, based on the definition
of a plane learned in the first chapter.

Recall the \(c\)-level surface of a function \(f(x,y,z)\) is the collection
\begin{equation*}
    \{ (x,y,z) | f(x,y,z) = c \} \,.
\end{equation*}

\begin{definition}
The tangent plane at the point \(P(x_0, y_0, z_0)\) on the \(c\)-level surface of a differentiable \(f\)
is the plane through \(P_0\), normal to \(\nabla f (x_0, y_0, z_0)\).
\end{definition}

\newpage

\chapter{Optimization}\label{optimization}

\section{First and second derivative tests}\label{first-and-second-derivative-tests}

Read Stewart Chapter 14, Thomas Chapter 14,

We will study multivariable scalar functions
\[ f: D \to \mathbb{R}\,,\]
where \(D\subseteq \mathbb{R}^n\), \(n\geq 2\).

\begin{definition}
A function \(f:D \to \mathbb{R}\) has a \textbf{local maximum} at \(\mathbf{x_0}\) if
\(f(\mathbf{x_0}) \geq f(\mathbf{x})\) for \(\mathbf{x} \in B_\delta(\mathbf{x_0})\) for small enough \(\delta\).
\(f\) has a \textbf{global maximum} at \(\mathbf{x_0}\) if
\(f(\mathbf{x_0}) \geq f(\mathbf{x})\) for \(\mathbf{x} \in D\).
\(f\) has a \textbf{local (global) minimum} at \(\mathbf{x_0}\) if
\(-f\) has a local (global) maximum at \(\mathbf{x_0}\)
\end{definition}

\begin{theorem}[First derivative test]
Let \(f:D \to \mathbb{R}\) be a function.
If \(\mathbf{x_0}\) is a local minimum and \(f\) has partial derivatives at \(\mathbf{x_0}\).
Then
\begin{equation*}
    \partial_{x_i} f(\mathbf{x}_0) = 0 \,.
\end{equation*}
\end{theorem}

The converse is not true, as having \(\nabla f(\mathbf{x}_0) = \mathbf{0}\) does not mean
that \(f\) has a local minimum at \(\mathbf{x}_0\).

\begin{exercise}
Think of a function that the converse to the above theorem is not true.
\end{exercise}

This leads to the following notion.

\begin{definition}
\(\mathbf{x}_0\) is said to be a \textbf{critical point} of \(f:D\to \mathbb{R}\) if
\begin{equation*}
    \nabla f(\mathbf{x}_0) = 0
\end{equation*}
or one of the partial derivatives \(\partial_{x_i} f(\mathbf{x}_0)\) fails to exist.
\end{definition}

\textbf{Please pay attention about the ``fail to exist'' condition.}

\begin{theorem}[Second derivative test for functions of 2 variables]

Suppose the second partial derivatives of \(f\) are continuous near \((a,b)\)
and suppose that \((a,b)\) is a critical point of \(f\).
Let
\begin{equation*}
    D = f_{xx}(a,b) f_{yy}(a,b) - f_{xy}(a,b)^2\,.
\end{equation*}

\begin{enumerate}
\def\labelenumi{\arabic{enumi}.}
\item
  If \(D>0\) and \(f_{xx}(a,b) >0\), then \(f(a,b)\) is a local minimum.
\item
  If \(D>0\) and \(f_{xx}(a,b) <0\), then \(f(a,b)\) is a local maximum.
\item
  If \(D<0\), then \(f(a,b)\) is neither a local maximum nor local minimum.
\item
  If \(D=0\), then we cannot conclude.
\end{enumerate}

\end{theorem}

\begin{theorem}[Extreme value theorem]
If \(f\) is continuous on a \emph{closed} and \emph{bounded} set \(D\). Then,
\(f\) attains an absolute minimum and an absolute maximum in \(D\).
\end{theorem}

\subsection{Algorithm to find absolute maxima and minima on closed bounded regions}\label{algorithm-to-find-absolute-maxima-and-minima-on-closed-bounded-regions}

\begin{enumerate}
\def\labelenumi{\arabic{enumi}.}
\item
  Find the values of \(f\) at the critical points of \(f\) in \(D\).
\item
  Find the extreme values of \(f\) on the boundary of \(D\).
\item
  The largest of the values from steps 1 and 2 is the absolute maximum value;
  the smallest of these values is the absolute minimum value.
\end{enumerate}

\section{Constrained optimization}\label{constrained-optimization}

Constrained optimization takes various forms, depending on the assumptions.
We will deal with the most straight forward form.
The problem we will study is the following:

Maximize/minimize a function \(f:D\to \mathbb{R}\), subject to a constraint (side condition)
of the form
\(g(\mathbf{x}) = k\), for some constant \(k\in \mathbb{R}\).

\begin{theorem}[Method of Lagrange Multiplier]

Suppose the maximum/minimum values of \(f\) exist and \(\nabla g(\mathbf{x}) \not=0\) where \(g(\mathbf{x}) = k\).
To find the maximum and minimum values of \(f\) subject to constraint
\(g(\mathbf{x}) = k\), we do the following:

\begin{enumerate}
\def\labelenumi{\arabic{enumi}.}
\item
  Find all values of \(\mathbf{x}\) and \(\lambda \in \mathbb{R}\) such that
  \begin{equation*}
   \nabla f(\mathbf{x}) =\lambda \nabla g(\mathbf{x})\,,
  \end{equation*}
  and
  \begin{equation*}
   g(\mathbf{x}) = k \,.
  \end{equation*}
\item
  Evaluate \(f\) at all the points \(\mathbf{x}\) that result from step 1. The largest of
  these values is the maximum of \(f\); the smallest is the minimum value of \(f\).
\end{enumerate}

\end{theorem}

\newpage

\chapter{Multiple integrals}\label{multiple-integrals}

Read Stewart Chapter 15 and Thomas Chapter 15

\textbf{Notations:}

Rectangle \(R= [a_1,b_1]\times \dots \times [a_n, b_n] \subseteq \mathbb{R}^n\).

\section{Basic definition}\label{basic-definition}

\begin{definition}

Let \(f\) be a function on a rectangle \(R\).
An n-fold Riemann sum for \(f\) over \(R\) is a sum of the following form
\begin{equation*}
 \sum_{i_1=1}^{m_1}\dots \sum_{i_n=1}^{m_n} f(\xi_{i_1\dots i_n}) \Delta A \,,
\end{equation*}
where

\begin{itemize}
\item
  \(\Delta A = \Delta x_1\times\dots  \times \Delta x_n\),
\item
  \(\Delta x_i = (b_i-a_i)/m_i\),
\item
  \(\xi_{i_1\dots i_n}\in R_{i_1\dots i_n}\),
\item
  \(R_{i_1\dots i_n}= \prod [a_i + (i_1-1)\Delta x, a_i+ i_1\Delta x_i]\).
\end{itemize}

\end{definition}

\begin{definition}
The double integral of \(f\) over a rectangle \(R \subseteq \mathbb{R}^2\) is
\begin{equation*}
    \iint_{R} f(x,y) \, dA = \lim_{m,n\to \infty} \sum_{i=1}^m \sum_{j=1}^n f(\xi_{ij}) \Delta A 
\end{equation*}
if the limit exists.

The triple integral of \(f\) over a rectangle \(R \subseteq \mathbb{R}^3\) is
\begin{equation*}
    \iiint_{R} f(x,y) \, dA = \lim_{m,n,l\to \infty} \sum_{i=1}^m \sum_{j=1}^n \sum_{k=1}^l f(\xi_{ijk}) \Delta A 
\end{equation*}
if the limit exists.
\end{definition}

\subsection{Some properties of integrals}\label{some-properties-of-integrals}

\begin{enumerate}
\def\labelenumi{\arabic{enumi}.}
\item
  Let \(U, V\) be disjoint domains, then
  \begin{equation*}
  \iint_{U \cup V} f \, dA = \iint_U f \, dA + \iint_V f\, dA \,. 
  \end{equation*}
\item
  \begin{equation*}
   \iint_U (f + g) \, dA = \iint_U f \, dA + \iint_V f \, dA \,.
  \end{equation*}
\end{enumerate}

\section{Iterated integrals}\label{iterated-integrals}

Suppose that \(f\) is integrable on \(R= [a,b]\times [c,d]\).
An iterated integral of \(f\) is defined as
\begin{equation*}
    \int_a^b A(x) \, dx \,,
\end{equation*}
where
\begin{equation*}
    A(x) = \int_c^d f(x,y) \, dy \,.
\end{equation*}
Typically, we write the above as
\begin{equation*}
   \int_a^b \int_c^d f(x,y) \, dy dx \,. 
\end{equation*}
This means that we integrate in \(y\) before in \(x\)-- always integrate the inner part first.

Similarly, we can define an iterated integral in a different order
\begin{equation*}
    \int_c^d \int_a^b f(x,y) \, dx dy \,.
\end{equation*}

The biggest question:
Is it true that
\begin{equation*}
   \int_a^b \int_c^d f(x,y) \, dy dx  =
    \int_c^d \int_a^b f(x,y) \, dx dy \,?
\end{equation*}

\begin{theorem}[Special case of Fubini]
If \(f\) is continuous on the rectangle \(R\), then
\begin{equation*}
    \iint_R f(x,y) \, dA = \int_a^b \int_c^d f(x,y) \, dy dx = \int_c^d \int_a^b f(x,y) \, dx dy \,.
\end{equation*}
\end{theorem}

\begin{example}
Let
\begin{equation*}
    f(x,y) =  \frac{x^2 - y^2}{(x^2 + y^2)^2} \,.
\end{equation*}
\begin{equation*}
    \int_0^1\int_0^1 f(x,y) \,dy dx = \frac{\pi}{4} = - \int_0^1\int_0^1 f(x,y) \, dx dy \,.
\end{equation*}
\end{example}

Everything we discuss here is true for three-variable functions.

\section{Change of coordinates}\label{change-of-coordinates}

A coordinate transformation is a function \(\varphi\), which
is bijective and differentiable for which \(D\varphi\) is
invertible at all points in the domain.
Here,
\begin{equation*}
    D\varphi = 
    \begin{pmatrix}
        \partial_1 \varphi_1 & \partial_2 \varphi_1 \\
        \partial_1 \varphi_2 & \partial_2 \varphi_2 
    \end{pmatrix} \,.
\end{equation*}

We will need to re-call the notion of invertible matrix here.
For an \(n\times n\) matrix \(A\), it is invertible iff \(\det A \not= 0\),.

\begin{theorem}
Let \(f\) be a function of \((x,y)\) defined on the domain \(D\).
Let
\begin{equation*}
    \begin{pmatrix}
        x \\ y
    \end{pmatrix}
     = \varphi(u,v)
\end{equation*}
for some coordinate change function \(\varphi: D \to S\).
If \(f\) is continuous and \(\varphi\) is differentiable, then
\begin{equation*}
    \int_S f \, dA = \int_D f\circ \varphi |\det D \varphi| \, dA
\end{equation*}
\end{theorem}

\subsection{Applications of change of coordinates}\label{applications-of-change-of-coordinates}

\subsubsection{Polar coordinate}\label{polar-coordinate}

In \(\mathbb{R}^2\),
when the region of integration is a section of a disk centered at \(0\).
Let
\begin{equation*}
    \begin{pmatrix}
        x \\ y
    \end{pmatrix}
    =
    \varphi(r,\theta) = 
    \begin{pmatrix}
        r\cos\theta \\
        r\sin\theta 
    \end{pmatrix} \,,
\end{equation*}
where \(a \leq r \leq b\) and \(\alpha \leq \theta \leq \beta\).

\subsubsection{Cylindrical coordinate}\label{cylindrical-coordinate}

In \(\mathbb{R}^3\), when the region of integration is part of a cylinder.
Let
\begin{equation*}
    \begin{pmatrix}
        x \\ y \\z
    \end{pmatrix}
    =
    \varphi(r,\phi,\theta) = 
    \begin{pmatrix}
        r\cos\theta\\
        r\sin\theta\\
        z
    \end{pmatrix} \,,
\end{equation*}
where \(a \leq r \leq b\), \(\alpha \leq \theta \leq \beta\).

\subsubsection{Spherical coordinate}\label{spherical-coordinate}

In \(\mathbb{R}^3\), when the region of integration is a section of a ball centered at \(0\).
Let
\begin{equation*}
    \begin{pmatrix}
        x \\ y \\z
    \end{pmatrix}
    =
    \varphi(\rho,\phi,\theta) = 
    \begin{pmatrix}
        \rho\sin\phi\cos\theta\\
        \rho\sin\phi\sin\theta\\
        \rho \cos\phi
    \end{pmatrix} \,,
\end{equation*}
where \(a \leq \rho \leq b\), \(\alpha \leq \theta \leq \beta\), and
\(c \leq \phi \leq d\).

\newpage

\chapter{Vector Calculus}\label{vector-calculus}

Read Chapter 16 in Stewart.

\section{Vector fields}\label{vector-fields}

\begin{definition}
Let \(D\) be a domain on \(\mathbb{R}^n\).
A vector field on \(\mathbb{R}^n\) is a function \(\mathbf{F}: D \to \mathbb{R}^n\)
that assign each point \(\mathbf{x}\in D\) to a vector \(\mathbf{F}(\mathbf{x}) \in \mathbb{R}^n\).
\end{definition}

In \(\mathbb{R}^2\), one typically write the vector fields in terms of \textbf{component functions} \(P, Q\)
\[\mathbf{F}(x,y) = P(x,y) \mathbf{i} + Q(x,y) \mathbf{j}\,.\]

In \(\mathbb{R}^3\), one typically write the vector fields in terms of \textbf{component functions} \(P, Q, R\)
\[\mathbf{F}(x,y,z) = P(x,y) \mathbf{i} + Q(x,y) \mathbf{j} + R(x,y,z) \mathbf{k}\,.\]

\begin{example}
Newton's Law of Gravitation
\begin{equation*}
    \mathbf{F}(\mathbf{x}) = - \frac{m M G}{| \mathbf{x}|^3 } \mathbf{x} \,,
\end{equation*}
where \(\mathbf{x}\) is the position in \(\mathbb{R}^3\).
\end{example}

\begin{example}
Coulomb's Law for the electric force exerted by an electric charge \(Q\)
at the origin on another charge \(q\) at a point \(\mathbf{x}\in \mathbb{R}^3\).
\begin{equation*}
    \mathbf{F}(\mathbf{x}) = \frac{ \epsilon q Q}{|\mathbf{x}|^3} \mathbf{x} \,.
\end{equation*}
\end{example}

\section{Line integrals}\label{line-integrals}

Let's focus on \(\mathbb{R}^2\).
We now perform a Riemann-sum-like action.

\begin{definition}
Let \(C\) be a curve.
The \textbf{line integral of \(f\) along \(C\)} is defined as
\begin{equation*}
    \int_C f(x,y) \, ds = \lim_{n\to \infty} \sum_{i=1}^n f(x_i^*, y_i^*) \Delta s_i \,,
\end{equation*}
where \(\Delta s_i\) is the length of a subarc of \(C\).
\end{definition}

\begin{proposition}
Suppose \(C\) is smooth and is parametrized by \(\mathbf{r}(t), a\leq t \leq b\). Then
\begin{equation*}
    \int_C f(x,y) \, ds = \int_a^b f(\mathbf{r}(t)) |\mathbf{r}'(t)| \, dt \,.
\end{equation*}
\end{proposition}

Note: when integrating with respect to arc length like this, reverse the direction of traversing the curve \(C\) will NOT result
in a change of sign of the final solution.

\begin{equation*}
    \int_{-C} f(x,y) \, ds =  \int_C f(x,y) \, ds \,.
\end{equation*}

Now we define line integrals of vector fields.

\begin{definition}
Let \(\mathbf{F}\) be a continuous vector field defined
on a curve \(C\).
Then the \textbf{line integral of \(\mathbf{F}\) along \(C\)} is defined as
\begin{equation*}
    \int_C \mathbf{F} \cdot d \mathbf{r} 
    = \int_C \mathbf{F}\cdot \mathbf{T} \, ds \,,
\end{equation*}
where \(\mathbf{T}\) is the unit tangent vector.
\end{definition}

\begin{proposition}
Suppose \(C\) is smooth and
parametrized by \(\mathbf{r}(t), a \leq t \leq b\).
Then
\begin{equation*}
    \int_C \mathbf{F} \cdot d \mathbf{r} 
    = \int_a^b \mathbf{F}(\mathbf{r}(t)) \cdot \mathbf{r}'(t) \, dt
\end{equation*}
\end{proposition}

We also use the following notations

\begin{align*}
\int_C f(x,y) dx := \int_a^b f(x(t), y(t) ) \, x'(t) \, dt \,, \\
\int_C f(x,y) dy := \int_a^b f(x(t), y(t) ) \, y'(t) \, dt \,, \\
\end{align*}

We can abbreviate the above by
\begin{equation*}
\int_C P(x,y) dx +
\int_C Q(x,y) dy 
= \int_C P(x,y) \, dx + Q(x,y) \, dy \,.
\end{equation*}

So,
\begin{equation*}
    \int_C \mathbf{F} \cdot d \mathbf{r} 
    = \int_C P \, dx + Q \, dy \,.
\end{equation*}

Note: as oppose to integrating the arc length, reversing the order of the above integrals
will change the sign of the integral.
This is because the arc length is always positive, while \(\Delta x\) and \(\Delta y\) could be either positive
or negative.

\begin{equation*}
\int_{-C} P(x,y) \, dx + Q(x,y) \, dy 
= -\int_C P(x,y) \, dx + Q(x,y) \, dy \,.
\end{equation*}

\begin{theorem}[Fundamental Theorem for line integrals]
Let \(C\) be a smooth curve given by the parametrization \(\mathbf{r}(t)\),
\(a \leq t \leq b\).
Let \(f\) be a differentiable function of two or three
variables whose gradient vector \(\nabla f\) is continuous on \(C\).
Then,
\begin{equation*}
    \int_C \nabla f \cdot d\mathbf{r} = f(\mathbf{r}(b)) - f(\mathbf{r}(a)) \,.
\end{equation*}
\end{theorem}

\begin{definition}
A \textbf{closed curve} is a curve that starts and ends at the same point.
A \textbf{simple closed curve} is a closed curve that never crosses itself.
\end{definition}

Sometimes, if \(C\) is a closed curve, we signify it by the following notation
\begin{equation*}
    \oint_C \nabla f \cdot d\mathbf{r} \,.
\end{equation*}

\begin{corollary}
If \(C\) is a closed curve and \(f:\mathbb{R}^n \to \mathbb{R}\) is a smooth function,
then
\begin{equation*}
    \oint_C \nabla f \cdot d\mathbf{r}  = 0 \,.
\end{equation*}
\end{corollary}

\begin{definition}
A vector field \(\mathbf{F}\) is called a \textbf{conservative vector field}
if it is the gradient of some scalar function, that is there exists
a function \(f\) such that
\begin{equation*}
    \nabla f = \mathbf{F} \,.
\end{equation*}
\end{definition}

Therefore, if \(\mathbf{F}\) is a conservative vector field, then
\begin{equation*}
    \oint_C \mathbf{F} \cdot d\mathbf{r} = 0 \,.
\end{equation*}

\subsection{Independence of path}\label{independence-of-path}

Suppose \(C_1\) and \(C_2\) are two piecewise smooth curves that have the same initial point \(A\)
and end point \(B\).
Then,
\begin{equation*}
    \int_{C_1} \mathbf{F}\cdot d\mathbf{r} 
    =
    \int_{C_2} \mathbf{F} \cdot d\mathbf{r} 
\end{equation*}
whenever \(\mathbf{F}\) is conservative (Why?).
The question is when is the converse true?

The following example is an example when the converse is not always true.

\begin{example}
Evaluate
\begin{equation*}
    \int_{C_i}  x^2 \, dy \,, \qquad i = 1,2
\end{equation*}
where \(C_1\) is the line segments from \((-1,-1) \to (-1,1) \to (1,1)\) and
\(C_2\) is the line segments from \((-1,-1) \to (1,-1) \to (1,1)\).
\end{example}

To further the discussion, we need a few definitions.

\begin{definition}
Let \(\mathbf{F}\) be a continuous vector field with domain \(D\), we say that the
line integral
\begin{equation*}
    \int_C \mathbf{F} \cdot d\mathbf{r} 
\end{equation*}
is \textbf{independent of path} if
\begin{equation*}
    \int_{C_1} \mathbf{F}\cdot d\mathbf{r} 
    =
    \int_{C_2} \mathbf{F} \cdot d\mathbf{r} 
\end{equation*}
for all paths that have the same starting and ending points.
\end{definition}

\begin{theorem}
\(\int_C \mathbf{F}\cdot d\mathbf{r}\) is independent of path in \(D\) if and only if
\(\oint_\Gamma \mathbf{F} \cdot d\mathbf{r} = 0\) for every closed path \(\Gamma\) in \(D\).
\end{theorem}

\begin{definition}
A domain \(D\) is said to be \textbf{open} if around each point, we can draw an open ball around it.
A domain \(D\) is said to be \textbf{connected} if for any two points, there is a path that connect them
together.
A domain \(D\) is said to be \textbf{simply connected} if is connected and there's no hole in it.
\end{definition}

\begin{theorem}
Suppose \(\mathbf{F}\) is a vector field that is continuous on an open
connected region \(D\).
If \(\int_C \mathbf{F} \cdot d \mathbf{r}\) is independent of path in \(D\),
then \(\mathbf{F}\) is a conservative vector field on \(D\).
\end{theorem}

\begin{proof}
todo
\end{proof}

The above theorem gives a way to determine if a vector field is conservative or not, from
the point of view of path independence.
However, it is often difficult to check the path independence property as one has to
integrate over ALL possible curves, and there are a lot of them\ldots{}

Another way is to take inspiration from Clairaut's theorem.
The question is to determine whether \(\mathbf{F}\) is conservative, given the mixed partial
derivatives of \(P\) and \(Q\) are the same, i.e.,
\begin{equation*}
    \frac{\partial P}{\partial y} = \frac{\partial Q}{\partial x}\,.
\end{equation*}
(Compare this with Clairaut's)

\begin{theorem}
Let \(\mathbf{F} = P\mathbf{i} + Q\mathbf{j}\) be a vector field on an open simply connected
region \(D\). Suppose that
\(P\) and \(Q\) have continuous first-order partial derivatives and
\begin{equation*}
    \frac{\partial P}{\partial y} = \frac{\partial Q}{\partial x}
\end{equation*}
through out \(D\).
Then \(\mathbf{F}\) is conservative.
\end{theorem}

\begin{remark}
The connectedness of \(D\) is crucial (why?).
\end{remark}

\section{Green's Theorem}\label{greens-theorem}

\begin{theorem}[Green's Theorem]
Let \(D\) be an open bounded simply connected domain in \(\mathbb{R}^2\),
\(\Gamma\) be the boundary of \(D\),
and \(\mathbf{F} = P\mathbf{i} + Q \mathbf{j}\) be a vector field.
If \(P\) and \(Q\) have continuous partial derivatives on an open region
that contains \(D\), then
\begin{equation*}
    \int_\Gamma \mathbf{F} \cdot d \ell  = \iint_D \left( \frac{\partial Q}{\partial x} - \frac{\partial P}{\partial y} \right) \, dA \,.
\end{equation*}
\end{theorem}

\section{Curl and Divergence}\label{curl-and-divergence}

\begin{definition}
Let \(\mathbf{F}\) be a vector field in \(\mathbb{R}^3\).
If all partial derivatives of \(P,Q,R\) exist, then we define
\begin{equation*}
    \mathrm{curl}\,\mathbf{F} =  \left( \frac{\partial R}{\partial y} - \frac{\partial Q}{\partial z} \right) \mathbf{i}
                       + \left( \frac{\partial P}{\partial z} - \frac{\partial R}{\partial x}  \right)  \mathbf{j}
                         + \left( \frac{\partial Q}{\partial x} - \frac{\partial P}{\partial y}  \right) \mathbf{k} \,.
\end{equation*}
\end{definition}

A different notation for \(\mathrm{curl} \, \mathbf{F}\) is
\begin{equation*}
    \nabla \times \mathbf{F} \,.
\end{equation*}

\begin{theorem}
If \(f\) is a function of 3 variables that has continuous second partial derivatives, then
\begin{equation*}
    \nabla \times ( \nabla f) = 0 \,.
\end{equation*}
\end{theorem}

\begin{theorem}
Suppose \(\mathbf{F}\) is a vector field on and simply connected domain \(D\) so that \(P,Q,R\)
all have continuous partial derivatives.
\(F\) is a conservative vector field if and only if
\(\nabla \times \mathbf{F} = 0\).
\end{theorem}

\begin{definition}
Let \(\mathbf{F}\) be a vector field in \(\mathbb{R}^3\).
If all partial derivatives of \(P,Q,R\) exist, then we define
\begin{equation*}
    \mathrm{div}\,\mathbf{F} =  
    \frac{\partial P}{\partial x} + \frac{\partial Q}{\partial y} + \frac{\partial R}{\partial z} \,.
\end{equation*}
\end{definition}

A different notation for \(\mathrm{div} \, \mathbf{F}\) is
\begin{equation*}
    \nabla \cdot \mathbf{F} \,.
\end{equation*}

\begin{theorem}
Suppose \(\mathbf{F}\) is a vector field on a domain \(D\) and \(P,Q,R\) have continuous
second-order partial derivatives.
Then,
\begin{equation*}
    \nabla \cdot (\nabla \times \mathbf{F}) = 0 \,.
\end{equation*}
\end{theorem}

\section{Surface integrals}\label{surface-integrals}

\subsection{Parametric surfaces}\label{parametric-surfaces}

Similar to the way we parametrize a curve by a one-variable vector function
\(\mathbf{r}(t)\), we can parametrize a surface by a two-variable vector function
\(\mathbf{r}(u,v)\).

We will only deal with surfaces in \(\mathbb{R}^3\) in this section. So, the parametrization
of a surface \(S\)
should be
\begin{equation*}
\mathbf{r}: D\subseteq \mathbb{R}^2 \to \mathbb{R}^3 \,.
\end{equation*}
We often write
\begin{equation*}
    \mathbf{r}(u,v) = x(u,v) \mathbf{i} + y(u,v) \mathbf{j} + z(u,v) \mathbf{k} \,.
\end{equation*}

From this parametrization,
we get to talk about the tangent plane of \(S\) at the point \(\mathbf{r}(u,v)\),
which is the plane that contains two tangent vectors
\begin{equation*}
    \mathbf{r}_u (u,v) = \frac{\partial x}{\partial u} \mathbf{i} +  \frac{\partial y}{\partial u} \mathbf{j} 
                            + \frac{\partial z}{\partial u} \mathbf{k} \,,
\end{equation*}
and
\begin{equation*}
    \mathbf{r}_v (u,v) = \frac{\partial x}{\partial v} \mathbf{i} +  \frac{\partial y}{\partial v} \mathbf{j} 
                            + \frac{\partial z}{\partial v} \mathbf{k} \,.
\end{equation*}

\subsection{Surface integral}\label{surface-integral}

\begin{definition}
Let \(S\) be a surface with parametrization.
The surface integral of \(f\) over the surface \(S\) is
\begin{equation*}
    \iint_S f(x,y,z) \, dS = \lim_{m,n\to \infty} \sum_{i=1}^m \sum_{j=1}^n f(P_{ij}^*) \Delta S_{ij} \,.
\end{equation*}
\end{definition}

Similarly to the line integral,
one can show that
\begin{equation*}
   \iint_S f(x,y,z) \, dS  = \iint_D f(\mathbf{r}(u,v)) | \mathbf{r_u}\times \mathbf{r_v} | \, dA \,. 
\end{equation*}

\subsection{Orientation of the surface}\label{orientation-of-the-surface}

Given a surface \(S\), we define the orientation of it as following

\begin{enumerate}
\def\labelenumi{\arabic{enumi}.}
\item
  If \(S\) has a boundary, then the \textbf{positive orientation} of the surface is that
  when one walks along the boundary of the surface with the head points in that direction, the surface is on the left.
\item
  If \(S\) does not have a boundary, then the \textbf{positive orientation} is the direction of the outward normal vector.
\end{enumerate}

\subsection{Surface integral of vector fields}\label{surface-integral-of-vector-fields}

\begin{definition}
If \(\mathbf{F}\) is a continuous vector field on an oriented surface \(S\) (parametrized by \(\mathbf{r}(u,v)\))
with unit normal vector \(\mathbf{n}\), then the \textbf{surface integral of \(\mathbf{F}\) over
\(S\)} is
\begin{equation*}
    \iint_S \mathbf{F}\cdot \, d\mathbf{S} = \iint_S \mathbf{F}\cdot \mathbf{n} \, dS 
    = \iint_S \mathbf{F}\cdot (\mathbf{r}_u\times \mathbf{r}_v) \, dA \,.
\end{equation*}
The integral is called the \textbf{flux of \(\mathbf{F}\)} across \(S\).
\end{definition}

\section{Stokes' and Divergence Theorem}\label{stokes-and-divergence-theorem}

\begin{theorem}[Stokes' Theorem]
Let \(S\) be an oriented smooth surface that is bounded by a simple closed
smooth boundary curve \(\partial S\) with positive orientation.
Let \(\mathbf{F}\) be a vector field whose components have continuous partial
derivatives on an open region in \(\mathbb{R}^3\) that contains \(S\).
Then
\begin{equation*}
    \int_{\partial S} \mathbf{F} \cdot \, d\mathbf{r} = \iint_S \nabla \times \mathbf{F} \cdot d\mathbf{S} \,.
\end{equation*}
\end{theorem}

The boundary of an area is a curve.
Similarly, the boundary of a solid is a surface.

\begin{theorem}[Divergence Theorem]
Let \(E\) be a simple solid region and let surface \(\partial E\) be the boundary of \(E\),
given with positive (outward) orientation.
Let \(\mathbf{F}\) be a vector field whose components have continuous partial derivatives.
Then,
\begin{equation*}
    \iint_{\partial E} \mathbf{F} \cdot d\mathbf{S} = \iiint_E \mathrm{div} \mathbf{F} \, dV \,.
\end{equation*}
\end{theorem}

\end{document}
