% Options for packages loaded elsewhere
\PassOptionsToPackage{unicode}{hyperref}
\PassOptionsToPackage{hyphens}{url}
%
\documentclass[
]{article}
\usepackage{amsmath,amssymb}
\usepackage{iftex}
\ifPDFTeX
  \usepackage[T1]{fontenc}
  \usepackage[utf8]{inputenc}
  \usepackage{textcomp} % provide euro and other symbols
\else % if luatex or xetex
  \usepackage{unicode-math} % this also loads fontspec
  \defaultfontfeatures{Scale=MatchLowercase}
  \defaultfontfeatures[\rmfamily]{Ligatures=TeX,Scale=1}
\fi
\usepackage{lmodern}
\ifPDFTeX\else
  % xetex/luatex font selection
\fi
% Use upquote if available, for straight quotes in verbatim environments
\IfFileExists{upquote.sty}{\usepackage{upquote}}{}
\IfFileExists{microtype.sty}{% use microtype if available
  \usepackage[]{microtype}
  \UseMicrotypeSet[protrusion]{basicmath} % disable protrusion for tt fonts
}{}
\makeatletter
\@ifundefined{KOMAClassName}{% if non-KOMA class
  \IfFileExists{parskip.sty}{%
    \usepackage{parskip}
  }{% else
    \setlength{\parindent}{0pt}
    \setlength{\parskip}{6pt plus 2pt minus 1pt}}
}{% if KOMA class
  \KOMAoptions{parskip=half}}
\makeatother
\usepackage{xcolor}
\usepackage[margin=1in]{geometry}
\usepackage{longtable,booktabs,array}
\usepackage{calc} % for calculating minipage widths
% Correct order of tables after \paragraph or \subparagraph
\usepackage{etoolbox}
\makeatletter
\patchcmd\longtable{\par}{\if@noskipsec\mbox{}\fi\par}{}{}
\makeatother
% Allow footnotes in longtable head/foot
\IfFileExists{footnotehyper.sty}{\usepackage{footnotehyper}}{\usepackage{footnote}}
\makesavenoteenv{longtable}
\usepackage{graphicx}
\makeatletter
\def\maxwidth{\ifdim\Gin@nat@width>\linewidth\linewidth\else\Gin@nat@width\fi}
\def\maxheight{\ifdim\Gin@nat@height>\textheight\textheight\else\Gin@nat@height\fi}
\makeatother
% Scale images if necessary, so that they will not overflow the page
% margins by default, and it is still possible to overwrite the defaults
% using explicit options in \includegraphics[width, height, ...]{}
\setkeys{Gin}{width=\maxwidth,height=\maxheight,keepaspectratio}
% Set default figure placement to htbp
\makeatletter
\def\fps@figure{htbp}
\makeatother
\setlength{\emergencystretch}{3em} % prevent overfull lines
\providecommand{\tightlist}{%
  \setlength{\itemsep}{0pt}\setlength{\parskip}{0pt}}
\setcounter{secnumdepth}{5}
\ifLuaTeX
  \usepackage{selnolig}  % disable illegal ligatures
\fi
\IfFileExists{bookmark.sty}{\usepackage{bookmark}}{\usepackage{hyperref}}
\IfFileExists{xurl.sty}{\usepackage{xurl}}{} % add URL line breaks if available
\urlstyle{same}
\hypersetup{
  pdftitle={MATH 104: Multivariable Calculus (brief notes)},
  pdfauthor={Truong-Son Van},
  hidelinks,
  pdfcreator={LaTeX via pandoc}}

\title{MATH 104: Multivariable Calculus (brief notes)}
\author{Truong-Son Van}
\date{}

\usepackage{amsthm}
\newtheorem{theorem}{Theorem}[section]
\newtheorem{lemma}{Lemma}[section]
\newtheorem{corollary}{Corollary}[section]
\newtheorem{proposition}{Proposition}[section]
\newtheorem{conjecture}{Conjecture}[section]
\theoremstyle{definition}
\newtheorem{definition}{Definition}[section]
\theoremstyle{definition}
\newtheorem{example}{Example}[section]
\theoremstyle{definition}
\newtheorem{exercise}{Exercise}[section]
\theoremstyle{definition}
\newtheorem{hypothesis}{Hypothesis}[section]
\theoremstyle{remark}
\newtheorem*{remark}{Remark}
\newtheorem*{solution}{Solution}
\begin{document}
\maketitle

{
\setcounter{tocdepth}{2}
\tableofcontents
}
\hypertarget{spring-2023}{%
\section*{Spring 2023}\label{spring-2023}}

\newpage

\newcommand{\vectorproj}[2][]{\mathrm{proj}_{\vect{#1}}\vect{#2}}
\newcommand{\vectorcomp}[2][]{\mathrm{comp}_{\vect{#1}}\vect{#2}}
\newcommand{\vect}{\mathbf}
\newcommand{\R}{\mathbb{R}}

\hypertarget{vectors}{%
\section{Vectors}\label{vectors}}

\hypertarget{basics}{%
\subsection{Basics}\label{basics}}

\textbf{Reading: Stewart Chapter 12, Thomas Calculus Chapter 12,
Active Calculus Chapter 9}

You should be able to answer the following questions after reading this section:

\begin{itemize}
\item
  What is a vector?
\item
  What does it mean for two vectors to be equal?
\item
  How do we add two vectors together and multiply a vector by a scalar?
\item
  How do we determine the magnitude of a vector?
\item
  What is a unit vector
\item
  How do we find a unit vector in the direction of a given vector?
\end{itemize}

Typically, we talk about 3-dimensional vectors (as discussed in Stewart and Thomas).
However, since talking about \(n\)-dimensional vectors doesn't
require much more effort,
we will talk about \(n\)-dimensional vectors instead.

\begin{definition}
An \(n\)-dimensional Euclidean space \(\mathbb{R}^n\)
is the Cartesian product of \(n\) Euclidean spaces \(\mathbb{R}\).
\end{definition}

\begin{definition}
An \(n\)-dimensional vector \(\textbf{v}\in \mathbb{R}^n\) is a tuple
\begin{equation}
    \textbf{v} = \langle v_1,\dots, v_n \rangle \,,
\end{equation}
where \(v_i \in \mathbb{R}\).
\end{definition}

In dimensions less than or equal to 3, we represent a vector
geometrically by an arrow, whose length represents the magnitude.

\begin{remark}
A point in \(\mathbb{R}^n\) is also represented by an \(n\)-tuple
but with round brackets.
A vector connecting two points \(A= (a_1, \dots, a_n)\)
and \(B=(b_1, \dots, b_n)\) can be constructed as
\begin{equation*}
    \textbf{x} =  \langle b_1-a_1, \dots, b_n - a_n \rangle \,.
\end{equation*}

We denote the above vector as \(\vec{AB}\) where \(A\) is the tail (initial point)
and \(B\) is the tip/head (terminal point).
We denote \(\textbf{0}\) to be the zero vector, i.e.,
\begin{equation*}
    \textbf{0} = \langle 0, \dots, 0 \rangle \,.
\end{equation*}
\end{remark}

\begin{definition}
The length of a vector \(\textbf{v}\) (denoted by \(| \textbf{v}|\)) is defined to be
\begin{equation}
    |\textbf{v}| = \sqrt{ v_1^2 + \dots + v_n^2} \,.
\end{equation}
\end{definition}

\begin{definition}
A unit vector is a vector that has magnitude 1.
\end{definition}

\begin{exercise}
Turn a vector \(\textbf{v} \in \mathbb{R}^n\) into a unit vector with the same
direction.
\end{exercise}

\hypertarget{rules-to-manipulate-vectors}{%
\subsection*{Rules to manipulate vectors}\label{rules-to-manipulate-vectors}}
\addcontentsline{toc}{subsection}{Rules to manipulate vectors}

Let \(\textbf{a}, \textbf{b} \in \mathbb{R}^n\) and \(c,d \in \mathbb{R}\). Then,

\begin{equation*}
    c( \textbf{a} + \textbf{b}) = \langle c a_1 + c b_1, \dots, c a_n + c b_n \rangle  
    = c\textbf{a} + c\textbf{b} \,,
\end{equation*}
and
\begin{equation*}
 (c+d) \textbf{a} = c\mathbf{a} + d\mathbf{a} \,.
\end{equation*}

These formulas are deceptively simple. Make sure you understand all the implications.

Because of this rule, sometimes it is good to write vectors in terms of elementary vectors:
\begin{equation*}
    \mathbf{u} = u_1 \mathbf{e_1} + \dots + u_n \mathbf{e_n} \,,
\end{equation*}
where
\(e_i = \langle 0,\dots, 1, \dots, 0\rangle\) is the vector which has zero at all entries
except that the \(i^{th}\) entry is 1.

In 3D,
\begin{equation*}
    \mathbf{e_1} = \mathbf{i} \,, \qquad 
    \mathbf{e_2} = \mathbf{j} \,, \qquad
    \mathbf{e_3} = \mathbf{k} \,.
\end{equation*}

\hypertarget{properties-of-vector-operations}{%
\subsection*{Properties of vector operations}\label{properties-of-vector-operations}}
\addcontentsline{toc}{subsection}{Properties of vector operations}

Read the book

(Make sure you understand the geometric intepretation)

\hypertarget{products}{%
\subsection{Products}\label{products}}

\hypertarget{dot-product}{%
\subsubsection{Dot product}\label{dot-product}}

\begin{itemize}
\item
  How is the dot product of two vectors defined and what geometric information does it tell us?
\item
  How can we tell if two vectors in \(\mathbb{R}^n\)
  are perpendicular?
\item
  How do we find the projection of one vector onto another?
\end{itemize}

\begin{definition}
The dot product of vectors \(\textbf{u} = \langle u_1, \dots, u_n \rangle\)
and \(\textbf{v} = \langle v_1, \dots, v_n \rangle\) in \(\mathbb{R}^n\) is the
scalar
\begin{equation*}
    \textbf{u} \cdot \textbf{v} = u_1 v_1 +\dots + u_n v_n \,.
\end{equation*}
\end{definition}

\hypertarget{properties-of-dot-product}{%
\subsubsection*{Properties of dot product}\label{properties-of-dot-product}}
\addcontentsline{toc}{subsubsection}{Properties of dot product}

Let \(\textbf{u}, \textbf{v}, \textbf{w} \in \mathbb{R}^n\). Then,

\begin{enumerate}
\def\labelenumi{\arabic{enumi}.}
\item
  \(\textbf{u}\cdot \textbf{v} = \textbf{v}\cdot \textbf{u}\),
\item
  \((\textbf{u} + \textbf{v})\cdot \textbf{w} = (\textbf{u}\cdot \textbf{w}) + (\textbf{v}\cdot \textbf{w})\),
\item
  If \(c\) is a scalar, then \((c \textbf{u})\cdot \textbf{w} = c (\textbf{u}\cdot \textbf{w})\).
\end{enumerate}

\begin{theorem}[Law of cosine]
If \(\theta\) is the angle between the vectors \(\textbf{u}\) and \(\textbf{v}\), then
\begin{equation*}
        \textbf{u}\cdot \textbf{v} = |\textbf{u}|| \textbf{v}| \cos \theta \,.
   \end{equation*}
\end{theorem}

\begin{corollary}
Two vectors \(\textbf{u}\) and \(\textbf{v}\) are orthogonal to each other
if \(\textbf{u} \cdot \textbf{v} = 0\).
\end{corollary}

\hypertarget{projection}{%
\subsubsection*{Projection}\label{projection}}
\addcontentsline{toc}{subsubsection}{Projection}

Let \(\textbf{u}, \textbf{v}\in \mathbb{R}^n\). The component of \(\textbf{u}\)
in the direction of \(\textbf{v}\) is the scalar
\begin{equation*}
\mathrm{comp}_{\mathbf{v}}\mathbf{u} = \frac{\mathbf{u}\cdot \mathbf{v}}{|\mathbf{v}|} \,,
\end{equation*}
and the projection of \(\mathbf{u}\) onto \(\mathbf{v}\) is the vector
\begin{equation*}
    \mathrm{proj}_{\mathbf{v}}\mathbf{u} 
    =\left( \mathbf{u}\cdot \frac{\mathbf{v}}{|\mathbf{v}|}\right) \frac{\mathbf{v}}{|\mathbf{v}|} 
    = \frac{\mathbf{u}\cdot \mathbf{v}}{\mathbf{v} \cdot\mathbf{v}} \mathbf{v} \,.
\end{equation*}

Read the book for more details.
Make sure you understand the geometric meaning.

\hypertarget{d-special-cross-product}{%
\subsubsection{3D special: Cross product}\label{d-special-cross-product}}

This concept is very specific to \(\mathbb{R}^3\).
It will not make sense in other dimensions.

\begin{definition}
Let \(\mathbf{a}, \mathbf{b} \in \mathbb{R}^3\).
The cross product of \(\mathbf{a}\) and \(\mathbf{b}\) is defined to be
\begin{equation*}
    \mathbf{a} \times \mathbf{b} = \langle a_2 b_3 - a_3 b_2, a_3b_1 - a_1 b_3, a_1b_2 - a_2b_1 \rangle \,.
\end{equation*}
\end{definition}

\begin{theorem}
Let \(\theta\) be the angle between \(\mathbf{a}\) and \(\mathbf{b}\). Then,
\begin{equation*}
    | \mathbf{a} \times \mathbf{b} | = |\mathbf{a}||\mathbf{b}| \sin\theta \,.
\end{equation*}
\end{theorem}

\begin{theorem}
The vector \(\mathbf{a}\times \mathbf{b}\) is orthogonal to both \(\mathbf{a}\) and \(\mathbf{b}\).
\end{theorem}

\newpage

\hypertarget{some-basic-equations-in-mathbbr3}{%
\section{\texorpdfstring{Some basic equations in \(\mathbb{R}^3\)}{Some basic equations in \textbackslash mathbb\{R\}\^{}3}}\label{some-basic-equations-in-mathbbr3}}

Just to build some toy examples for the future, we will play with some basic
equations in three dimensions.

\hypertarget{equations-for-lines}{%
\subsection{Equations for lines}\label{equations-for-lines}}

A line is a collection of points that is parallel to a vector and goes through a
specific point.
To capture this idea, we have the following representation for a line
\begin{equation*}
    L = \{\mathbf{r}(t) \,|  \mathbf{r}(t) = \mathbf{r}_0 + t \mathbf{v}, t\in \mathbb{R}\}  \,,
\end{equation*}
where \({r}_0\) is the initial position and \(\mathbf{v}\) is the direction.
The equation for \(\mathbf{r}(t)\) is called a \textbf{vector equation for a line \(L\)}.

Let \(\mathbf{v} = \langle v_1, v_2, v_3 \rangle\) and \(\mathbf{r}_0 = ( x_0, y_0, z_0 )\).
The \textbf{parametric equations} of \(L\) is the following system of equations

\begin{gather*}
    x = x_0 + v_1 t\,, \\
    y = y_0 + v_2 t\,, \\
    z = z_0 + v_3 t \,. 
\end{gather*}

This leads to the \textbf{symmetric equations} of \(L\)

\begin{equation*}
    \frac{x - x_0}{v_1} = \frac{y - y_0}{v_2} = \frac{z - z_0}{v_3} \,.
\end{equation*}

\begin{definition}
Two lines are parallel if their directional vectors are parallel (scalar multiple of each other).

Two lines that are not parallel and don't intersect each other are said to be skew.
\end{definition}

\hypertarget{equations-for-planes}{%
\subsection{Equations for planes}\label{equations-for-planes}}

A plane is a collection of points that is perpendicular to one specific direction
represented by a some vector called a \textbf{normal vector}.
Note that due to scaling, there are more than one normal vector.
To capture this idea, we have the following representation of a plane

\begin{equation*}
    P = \{ \mathbf{r} \, | \, \mathbf{n} \cdot (\mathbf{r}- \mathbf{r}_0 ) = 0 \} \,.
\end{equation*}

This is called a \textbf{vector equation for the plane \(P\)}.

Multiplying things out, we have the \textbf{scalar equation of the plane \(P\)} with
normal vector \(\mathbf{n} = \langle n_1, n_2, n_3 \rangle\) through a point \(P_0(x_0, y_0, z_0)\)
\begin{equation*}
    n_1(r_1- x_0) + n_2 (r_2 - y_0) + n_3(r_3 - z_0) = 0 \,.
\end{equation*}

The equation of the form
\begin{equation*}
    ax + by + cz + d = 0 
\end{equation*}
is called a \textbf{linear equation}.

\begin{definition}
Two planes are said to be parallel if their normal vectors are parallel.
If two planes are not parallel, they intersect in a straight line and
the angle between the two planes is defined to be the angle between the
two normal vectors.
\end{definition}

\hypertarget{cylinders}{%
\subsection{Cylinders}\label{cylinders}}

\begin{definition}
A cylinder is a surface that consists of all lines (called \textbf{rulings}) that
are parallel to a given line.
\end{definition}

\begin{example}
\leavevmode

\begin{enumerate}
\def\labelenumi{\arabic{enumi}.}
\tightlist
\item
  \(z = x^2\)
\item
  \(x^2 + y^2 = 1\)
\end{enumerate}

\end{example}

\hypertarget{quadric-surfaces}{%
\subsection{Quadric surfaces}\label{quadric-surfaces}}

\begin{definition}
A quadric surface is the graph of a second-degree equation in three variables
\(x,y\) and \(z\).
The equation that represents these surfaces is
\[Ax^2 + By^2 + Cz^2 + Dz = E\,.\]
\end{definition}

\begin{example}
\leavevmode

\begin{enumerate}
\def\labelenumi{\arabic{enumi}.}
\item
  Ellipsoid
  \[\frac{x^2}{a^2} + \frac{y^2}{b^2} + \frac{z^2}{c^2} = 1\,. \]
\item
  Hyperbolic paraboloid
  \[\frac{y^2}{b^2} - \frac{x^2}{a^2} = \frac{z}{c} \,.\]
\item
  Elliptical cone
  \[\frac{x^2}{a^2} + \frac{y^2}{b^2} = \frac{z^2}{c^2} \,.\]
\end{enumerate}

Read the books for more surfaces and pictures.

\end{example}

\newpage

\hypertarget{functions-in-higher-dimensions}{%
\section{Functions in higher dimensions}\label{functions-in-higher-dimensions}}

\textbf{Reading: Stewart Chapter 12, 13, Thomas Calculus Chapter 12, 13, Active Calculus Chapter 9}

\hypertarget{functions-of-several-variables}{%
\subsection{Functions of several variables}\label{functions-of-several-variables}}

\begin{definition}
A function of several variables is a function
\(f: D \to C\) where \(D \subseteq \mathbb{R}^m\) and \(C \subseteq \mathbb{R}^n\), where \(m\geq 2\) and \(n\geq 1\).
\[f({x}) = ( f_1(x_1,\dots, x_m),\dots, f_n(x_1,\dots, x_m)  ) \,.\]
\(D\) is called the domain of \(f\) and \(C\) is called the codomain of \(f\).
\end{definition}

The domain of \(f\) is where each of the component \(f_i\) of \(f\) is defined.

\begin{example}

The following are some examples of multivariable functions

\begin{enumerate}
\def\labelenumi{\arabic{enumi}.}
\item
  \(f(x,y) = x^2 - 2xy + y^2\)
\item
  \(f(x,y,z) = \frac{1}{1 - xy^2}\)
\end{enumerate}

\end{example}

\hypertarget{vector-functions}{%
\subsection{Vector functions}\label{vector-functions}}

\hypertarget{limit-continuity-and-differentiation}{%
\subsubsection{Limit, continuity and differentiation}\label{limit-continuity-and-differentiation}}

The expression in the vector equation for a line is an example of a function that maps from \(\mathbb{R}\) to \(\mathbb{R}^n\).
There's no one who would stop us from considering more general kinds of function.

\begin{definition}
A \textbf{vector function} (\textbf{vector-valued function}) is a function that has the codomain that belongs to \(\mathbb{R}^n\) where \(n\geq 2\). In other words, \(f: D \to \mathbb{R}^n\).
\end{definition}

\begin{example}

The following are some examples of vector functions.

\begin{enumerate}
\def\labelenumi{\arabic{enumi}.}
\item
  \(\mathbf{r}(t) = \mathbf{r}_0 + t\mathbf{v}\)
\item
  \(\mathbf{f}(t) = \langle \cos(t),\sin(t), t \rangle\)
\end{enumerate}

\end{example}

Note that my definition is more general than that in the book.
However,
\textbf{In this course, whenever we talk about vector valued function, we will only refer to
that which has one dimensional domain (\(D \subseteq \mathbb{R}\)).}

By and large, there's nothing different between a vector function and
a one-variable scalar function.
All the concepts such as limit, continuity and differentiability are applied
to each coordinate the same way as in one dimensional case.

\begin{theorem}
Let \(\mathbf{r}: \mathbb{R}\to \mathbb{R}^n\), given by \(\mathbf{r}(t) = \langle r_1(t), \dots , r_n(t) \rangle\).
Then, \(\mathbf{r}\) is said to be continuous at \(t_0\) if
\begin{equation*}
    \mathbf{r}(t_0) = \lim_{t\to t_0} \mathbf{r}(t) \,,
\end{equation*}
where
\begin{equation*}
    \lim_{t\to t_0} \mathbf{r}(t) = \langle \lim_{t\to t_0}r_1(t) , \dots , \lim_{t\to t_0} r_n(t) \rangle \,. 
\end{equation*}
Furthermore, we can define the derivative of \(\mathbf{r}\)
\begin{equation*}
    \frac{d}{dt} \mathbf{r}(t) = \mathbf{r}'(t) = \lim_{h\to 0} \frac{\mathbf{r}(t+h) - \mathbf{r}(t)}{h} 
\end{equation*}
if this limit exists.
\end{theorem}

When \(\mathbf{r}:I \to \mathbb{R}^n\) (\(I\) is an interval in \(\mathbb{R}\)) is continuous,
we call it a \textbf{space curve} (to describe the intuitive picture of what
a curve should look like in our mind).

Geometrically, if \(\mathbf{r}'(t)\) exists and \(\mathbf{r}'(t) \not= \mathbf{0}\), it
represents the \textbf{tangent vector} of the curve \(\mathbf{r}\) at \(t\).

\begin{definition}
A \textbf{parametric equation} for a curve is an equation of the form
\[
x=x(t)\,, \quad  y = y(t)\,, \quad z = z(t) \,.
\]
\end{definition}

Typical differentiation rules apply.

\begin{theorem}[Differentiation rules]
\leavevmode

\begin{enumerate}
\def\labelenumi{\arabic{enumi}.}
\item
  \((\mathbf{u}(t) + \mathbf{v}(t))' = \mathbf{u}'(t) + \mathbf{v}'(t)\)
\item
  \((c \mathbf{u}(t))' = c \mathbf{u}'(t)\)
\item
  \((f(t) \mathbf{u}(t))' = f'(t) \mathbf{u}(t) + f(t) \mathbf{u}'(t)\)
\item
  \((\mathbf{u}(t) \cdot \mathbf{v}(t))' = \mathbf{u}'(t)\cdot \mathbf{v}(t) + \mathbf{u}(t)\cdot \mathbf{v}'(t)\)
\item
  \((\mathbf{u}(t) \times \mathbf{v}(t))' = \mathbf{u}'(t)\times \mathbf{v}(t) + \mathbf{u}(t)\times \mathbf{v}'(t)\)
\item
  \((\mathbf{u}(f(t)))' = \mathbf{u}'(f(t)) f'(t)\)
\end{enumerate}

\end{theorem}

\hypertarget{integrals}{%
\subsubsection{Integrals}\label{integrals}}

There are different ways to play with integrals for vector functions,
each has its own interpretation and physical applications.

\hypertarget{indefinite-integral}{%
\paragraph{Indefinite integral}\label{indefinite-integral}}

\begin{equation*}
    \int_a^b \mathbf{r}(t) \, dt = \left\langle \int_a^b r_1(t) \, dt, \int_a^b r_2(t) \, dt, \int_a^b r_3(t) \, dt \right\rangle
\end{equation*}

\hypertarget{arc-length-and-curvature}{%
\paragraph{Arc Length and curvature}\label{arc-length-and-curvature}}

\begin{definition}
The length the curve \(\mathbf{r}:[a,b] \to \mathbb{R}^n\) is defined to be
\begin{equation*}
L = \int_a^b \left| \mathbf{r}'(t) \right| \, dt \,.
\end{equation*}
\end{definition}

If one wants to keep track the length of the curve \(\mathbf{r}:[a,b] \to \mathbb{R}^n\) made by an airplane
at any time \(t\), one uses the \textbf{arc length function}

\begin{equation*}
    \ell(t) = \int_a^t \left| \mathbf{r}'(u) \right| \, du \,.
\end{equation*}

\hypertarget{re-parametrize-with-respect-to-arc-length}{%
\paragraph*{Re-parametrize with respect to arc length}\label{re-parametrize-with-respect-to-arc-length}}
\addcontentsline{toc}{paragraph}{Re-parametrize with respect to arc length}

The nice thing about \(\ell(t)\) is that it is a strictly increasing function with respect to \(t\),
given that \(\mathbf{r}'\) is non-zero for all \(t\).
Therefore, letting \(s = \ell(t)\), we can talk about the inverse of \(\ell\), \(\ell^{-1}:[0,L] \to [a,b]\)
\begin{equation*}
    t = \ell^{-1}(s) \,.
\end{equation*}
Therefore, we can re-write
\begin{equation*}
\mathbf{r}(t) = \mathbf{r}(\ell^{-1}(s)) \,.
\end{equation*}

\begin{theorem}
\[\left| \frac{d r(t)}{ds} \right| = 1 \,.\]
Thus,
\[l(s) = \int_0^s \left| \frac{d}{ds} \mathbf{r}(t) \right| \, dt = s \,.\]
\end{theorem}

Because of the unchanging nature of the arc-length (with respect to the
parametrization),
it is used to define a geometric quantity of a space curve called \textbf{curvature}.

\begin{definition}[curvature]
Let \(\mathbf{T}(t)\) be the unit tangent vector of the curve \(\mathbf{r}:[a,b] \to \mathbb{R}^3\).
The curvature of \(\mathbf{r}(t(s))\) is defined to be
\begin{equation*}
    \kappa(s) = \left| \frac{d \mathbf{T}(t(s))}{ds}\right| \,.
\end{equation*}
\end{definition}

To convert this into the parameter \(t\), we write \(s= s(t)\) and use chain rule
to get.

\begin{theorem}
We have that
\begin{equation*}
    \kappa(s(t)) =  \frac{|\mathbf{T}'(t)|}{|\mathbf{r}'(t)|}   \,.
\end{equation*}
\end{theorem}

\hypertarget{space-curve-in-mathbbr3-and-motion-in-space}{%
\subsubsection{\texorpdfstring{Space curve in \(\mathbb{R}^3\) and motion in space}{Space curve in \textbackslash mathbb\{R\}\^{}3 and motion in space}}\label{space-curve-in-mathbbr3-and-motion-in-space}}

Read the book. This part is not required but it is so beautiful, you may want to read it
as an exercise at home (to test how much you understand what we've been discussing so far).

\hypertarget{activity-on-osculating-circle-and-curvature}{%
\subsection{Activity: on osculating circle and curvature}\label{activity-on-osculating-circle-and-curvature}}

For those who are interested in the geometrical meaning of the curvature without having
to accept from the book that the curvature is the inverse of the radius of the osculating circle,
please take a look at \url{https://github.com/sonv/MultiCalc/blob/main/Writing/latexbuild/osculating.pdf}.

\newpage

\hypertarget{partial-derivatives}{%
\section{Partial derivatives}\label{partial-derivatives}}

Read Stewart Chapter 14, Thomas Chapter 14,

We will study multivariable scalar functions
\[ f: D \to \mathbb{R}\,,\]
where \(D\subseteq \mathbb{R}^n\), \(n\geq 2\).

\hypertarget{multivariable-scalar-function}{%
\subsection{Multivariable scalar function}\label{multivariable-scalar-function}}

The following definition is from Thomas's book.

\begin{definition}
Suppose \(D\) is a set of \(n\)-tuples of real numbers \((x_1, x_2, \ldots, x_n)\).
A real-valued/scalar function \(f\) on \(D\) is a rule that assigns a unique (single) real
number
\[w = f(x_1, x_2, \ldots, x_n)\]
to each element in \(D\).
The set \(D\) is the function's domain. The set of \(w\)-values taken on by \(f\) is
the function's range. The symbol \(w\) is the dependent variable of \(f\), and \(f\)
is said to be a function of the \(n\) independent variables \(x_1\) to \(x_n\).
We also call the \(x_j\)'s the function's input variables and call \(w\) the function's output variable.
\end{definition}

As prototypes, we only focus on \(n=2\) and \(n=3\).

\hypertarget{graphs}{%
\subsubsection{Graphs}\label{graphs}}

\begin{definition}
The graph of function \(f:D \to \mathbb{R}\) is the set of all points
\((\mathbf{x}, f(\mathbf{x}))\), where \(\mathbf{x}\in D\).
Here \(D\subseteq \mathbb{R}^n\).
\end{definition}

For 2D, the graph of \(f\) is also called the \textbf{surface} \(z = f(x_1,x_2)\).

We cannot visualize the graph of a 3D function since it will be a four dimensional object.

\hypertarget{level}{%
\subsubsection{Level}\label{level}}

\begin{definition}
In 2D, the \textbf{\(c\)-level curves} of a function \(f\) of two variables are curves with equations
\(f(x,y) = c\), where \(c\) is a constant.

In 3D, the \textbf{\(c\)-level surface} of a function \(f\) of three variables are surfaces
with equations \(f(x,y,z) = c\), where \(c\) is a constant.
\end{definition}

\hypertarget{limits-and-continuity}{%
\subsection{Limits and continuity}\label{limits-and-continuity}}

The following definition is from Stewart.

\begin{definition}
Let \(f\) be a function of two variables whose domain \(D\) includes points arbitrarily close to \((a,b)\). Then we say that the limit of \(f(x,y)\) as \((x,y)\) approaches \((a,b)\) is \(L\) and we write

\[\lim_{(x,y)\to(a,b)} f(x,y) = L\]

if for every number \(\epsilon > 0\) there is a corresponding number \(\delta > 0\) such that
\(|f(x,y) - L| < \epsilon\)
if \((x,y) \in D\) and \(0 < \sqrt{(x-a)^2 + (y-b)^2} < \delta\).
\end{definition}

Finding if a function has limit as a point in higher dimension is not as simple as
the case for 1 dimension.

Determining whether a multivariable function has a limit sometimes is an art
and it requires a lot of experiences and practice.
However, there are certain rules that could help us.

\begin{theorem}

Let \(L,M\) and \(k\) be real numbers and that
\begin{equation*}
    \lim_{(x,y) \to (x_0,y_0)} f(x,y) = L \,, \qquad 
    \lim_{(x,y) \to (x_0,y_0)} g(x,y) = M \,.
\end{equation*}
We then have

\begin{enumerate}
\def\labelenumi{\arabic{enumi}.}
\item
  \(\displaystyle \lim_{(x,y) \to (x_0,y_0)} (f(x,y) + g(x,y)) = L + M\),
\item
  \(\displaystyle \lim_{(x,y) \to (x_0,y_0)} (k f(x,y)) = kL\),
\item
  \(\displaystyle \lim_{(x,y) \to (x_0,y_0)} (f(x,y) g(x,y)) = LM\),
\item
  \(\displaystyle \lim_{(x,y) \to (x_0,y_0)} \frac{f(x,y)}{g(x,y)} = \frac{L}{M}\) if \(M \not= 0\),
\item
  \(\displaystyle \lim_{(x,y) \to (x_0,y_0)} {f(x,y)^p} = L^p\) for \(p>0\),
\end{enumerate}

\end{theorem}

\textbf{Strategy to find out that a two-variable function does NOT have a limit.}

If \(\lim_{(x,y) \to (a,b)} f(x,y) = L_1\) as \((x,y) \to (a,b)\) along a path \(C_1\),
and \(\lim_{(x,y) \to (a,b)} f(x,y) = L_2\) as \((x,y) \to (a,b)\) along a path \(C_2\),
where \(L_1 \neq L_2\), then \(\lim_{(x,y) \to (a,b)} f(x,y)\) does not exist.

\begin{example}
\(\lim_{(x,y)\to (0,0)} \frac{x^2 - y^2}{x^2 + y^2}\) does not exist.

\(\lim_{(x,y)\to (0,0)} \frac{xy}{x^2 + y^2}\) does not exist.

\(\lim_{(x,y)\to (0,0)} \frac{xy^2}{x^4 + y^4}\) does not exist.

\(\lim_{(x,y)\to (0,0)} \frac{3x^2y}{x^2 + y^2} = 0\).
\end{example}

\hypertarget{partial-derivatives-1}{%
\subsection{Partial derivatives}\label{partial-derivatives-1}}

Given a function \(f(x,y)\). The partial derivative of \(f\) with respect to \(x\) and \((a,b)\),
denoted by \(f_x(a,b)\), is defined to be
\begin{equation*}
    f_x(a,b) = \lim_{h\to 0} \frac{ f(a+h,b) - f(a,b)}{h} \,.
\end{equation*}
Likewise, the partial derivative of \(f\) with respect to \(y\) and \((a,b)\),
denoted by \(f_y(a,b)\), is defined to be
\begin{equation*}
    f_y(a,b) = \lim_{h\to 0} \frac{ f(a,b+h) - f(a,b)}{h} \,.
\end{equation*}

Instead of thinking about derivative with respect to \(x,y\), we could think
about derivative with respect to the first and second direction.
This way of thinking is a bit better when one thinks about higher dimension.

\textbf{Notations.} If \(z = f(x,y)\), we write
\begin{equation*}
    f_x(x,y) = f_x = \partial_x f =  \frac{\partial f}{\partial x} = \frac{\partial}{\partial x} f(x,y) = f_1 = D_1 f = D_x f \,.
\end{equation*}

When you take partial derivatives, just treat other variables as constants and proceed as
in the case of one dimension.

More generally,
given a function \(f(x_1, \dots, x_n)\), its partial derivative with respect
to the \(i\)th variable \(x_i\) is
\begin{equation*}
    f_{x_i}(x_1, \dots, x_n) 
    = \lim_{h\to 0} \frac{ f(x_1, \dots, x_{i-1}, x_i + h , x_{i+1}, \dots, x_n) - f(x_1, \dots, x_{i-1}, x_i  , x_{i+1}, \dots, x_n)}{h} \,.
\end{equation*}

From here, one can define higher partial derivatives such as the following

\begin{equation*}
    \partial^3_{x_1 x_2 x_2}  f\,.
\end{equation*}

Note that the power over the symbol \(\partial\) represents the order of derivatives.

\begin{theorem}[Clairaut's Theorem]
Suppose \(f\) is defined on a disk \(D\) that contains the point \((a,b)\).
If the functions \(f_{xy}\) and \(f_{yx}\) are both continuous on \(D\), then
\begin{equation*}
    f_{xy}(a,b) = f_{yx}(a,b) \,.
\end{equation*}
\end{theorem}

\hypertarget{some-important-notations}{%
\paragraph*{Some important notations}\label{some-important-notations}}
\addcontentsline{toc}{paragraph}{Some important notations}

Let \(f:D \to \mathbb{R}\) be a function. We write the following, if exist,
\begin{equation*}
    \nabla f = \begin{bmatrix}
        \partial_{x_1} f\\
        \vdots \\
        \partial_{x_n} f\\
    \end{bmatrix}
\end{equation*}

\begin{equation*}
    \Delta f = \partial_{x_1}^2 f + \dots \partial_{x_n}^2 f \,.
\end{equation*}

\hypertarget{differentiability}{%
\subsection{Differentiability}\label{differentiability}}

Let \(S\) be a surface that has equation \(z = f(x,y)\).
Let \(C_1\) and \(C_2\) be two different curves on a surface \(S\)
intersect at a point \(P(x_0, y_0, z_0)\).

Heuristically, ``the'' \textbf{tangent plane} to the surface \(S\) at point \(P\) is defined the be the plane that contains
both of the tangent lines of both curves at \(P\).

How do you know if there is only one tangent plane at a point?
You actually don't know.
That's why the following definition exists.

\begin{definition}[Differentiability]
Let \(f:D \to \mathbb{R}\) and \(a\in \mathbb{R}^n\).
Let \(z = f(x)\) and \(\Delta z = f(a + \Delta x ) - f(a)\).
Then \(f\) is \textbf{differentiable at \(a\)} if \(\Delta z\) can be
expressed in the form
\begin{equation*}
    \Delta z = \sum_{i=1}^n \partial_i f(a) \Delta x_i +  \epsilon_i \Delta x_i  \,,
\end{equation*}
where \(\epsilon_i \to 0\) as \(\Delta x_i \to (0,0)\) and
\(\Delta x_i = x_i - a_i\).

\(f\) is said to be \textbf{differentiable} if it is differentiable at every point on the domain.
\end{definition}

The graph of a differentiable function \(f\) whose
domain is two dimensional is called a \textbf{smooth surface}.

Let \(\Delta x = (\Delta x_1, \dots, \Delta x_n)\).
Another reformulation of the definition of differentiability is that \(f\) is
differentiable if
\begin{equation*}
\lim_{|\Delta x| \to 0}
    \frac{\Delta z - \nabla f \cdot \Delta x }{| \Delta x|}  = 0 \,.
\end{equation*}
It would be a good exercise to see why this is equivalent with the definition above.

For some good intuition, please go to \url{https://mathinsight.org/differentiability_multivariable_definition}.

\begin{theorem}
If the partial derivatives \(\partial_i f\) (\(i= 1, \dots, n\)) exist near \(a\in \mathbb{R}^n\) and are continuous
at \(a\), then \(f\) is differentiable at \(a\).
\end{theorem}

\begin{theorem}
If a function \(f(x)\) is differentiable at \(a\) then \(f\) is continuous at \(a\).
\end{theorem}

\hypertarget{chain-rule}{%
\subsection{Chain rule}\label{chain-rule}}

\begin{theorem}
Let \(f(x_1,\dots, x_n), g_i(y_1,\dots, y_m)\) (\(i = 1,\dots, n\)) be
differentiable
functions.
Then,
\[z(y_1, \dots, y_m) = f(g_1(y_1, \dots, y_m), \dots, g_n(y_1, \dots, y_m))\]
is differentiable and
\begin{equation*}
    \frac{\partial z}{\partial y_i} = \sum_{j=1}^n \frac{\partial f}{\partial x_j} \frac{\partial g_j}{\partial y_i} \,.
\end{equation*}
\end{theorem}

\hypertarget{directional-derivative}{%
\subsection{Directional derivative}\label{directional-derivative}}

\begin{definition}
Let \(\mathbf{u} \in \mathbb{R}^n\). The directional derivative of \(f:\mathbb{R}^n \to \mathbb{R}\) at \(a\in \mathbb{R}^n\)
in the direction of \(\mathbf{u}\) is the following limit (if exists)
\begin{equation*}
    D_{\mathbf{u}} f(a) = \lim_{h \to 0} \frac{ f( a + h \mathbf{u}) - f(a)}{h}\,.
\end{equation*}
\end{definition}

How can one compute directional derivative?

\begin{theorem}
If \(f:\mathbb{R}^n \to \mathbb{R}\) is differentiable then
\begin{equation*}
    D_{\mathbf{u}} f(a) = \nabla f(a) \cdot \mathbf{u} \,.
\end{equation*}
\end{theorem}

\hypertarget{tangent-planes}{%
\subsection{Tangent planes}\label{tangent-planes}}

Let's think about tangent planes in a more systematic way, based on the definition
of a plane learned in the first chapter.

Recall the \(c\)-level surface of a function \(f(x,y,z)\) is the collection
\begin{equation*}
    \{ (x,y,z) | f(x,y,z) = c \} \,.
\end{equation*}

\begin{definition}
The tangent plane at the point \(P(x_0, y_0, z_0)\) on the \(c\)-level surface of a differentiable \(f\)
is the plane through \(P_0\), normal to \(\nabla f (x_0, y_0, z_0)\).
\end{definition}

\newpage

\hypertarget{optimization}{%
\section{Optimization}\label{optimization}}

\hypertarget{first-and-second-derivative-tests}{%
\subsection{First and second derivative tests}\label{first-and-second-derivative-tests}}

Read Stewart Chapter 14, Thomas Chapter 14,

We will study multivariable scalar functions
\[ f: D \to \mathbb{R}\,,\]
where \(D\subseteq \mathbb{R}^n\), \(n\geq 2\).

\begin{definition}
A function \(f:D \to \mathbb{R}\) has a \textbf{local maximum} at \(\mathbf{x_0}\) if
\(f(\mathbf{x_0}) \geq f(\mathbf{x})\) for \(\mathbf{x} \in B_\delta(\mathbf{x_0})\) for small enough \(\delta\).
\(f\) has a \textbf{global maximum} at \(\mathbf{x_0}\) if
\(f(\mathbf{x_0}) \geq f(\mathbf{x})\) for \(\mathbf{x} \in D\).
\(f\) has a \textbf{local (global) minimum} at \(\mathbf{x_0}\) if
\(-f\) has a local (global) maximum at \(\mathbf{x_0}\)
\end{definition}

\begin{theorem}[First derivative test]
Let \(f:D \to \mathbb{R}\) be a function.
If \(\mathbf{x_0}\) is a local minimum and \(f\) has partial derivatives at \(\mathbf{x_0}\).
Then
\begin{equation*}
    \partial_{x_i} f(\mathbf{x}_0) = 0 \,.
\end{equation*}
\end{theorem}

The converse is not true, as having \(\nabla f(\mathbf{x}_0) = \mathbf{0}\) does not mean
that \(f\) has a local minimum at \(\mathbf{x}_0\).

\begin{exercise}
Think of a function that the converse to the above theorem is not true.
\end{exercise}

This leads to the following notion.

\begin{definition}
\(\mathbf{x}_0\) is said to be a \textbf{critical point} of \(f:D\to \mathbb{R}\) if
\begin{equation*}
    \nabla f(\mathbf{x}_0) = 0
\end{equation*}
or one of the partial derivatives \(\partial_{x_i} f(\mathbf{x}_0)\) fails to exist.
\end{definition}

\textbf{Please pay attention about the ``fail to exist'' condition.}

\begin{theorem}[Second derivative test for functions of 2 variables]

Suppose the second partial derivatives of \(f\) are continuous near \((a,b)\)
and suppose that \((a,b)\) is a critical point of \(f\).
Let
\begin{equation*}
    D = f_{xx}(a,b) f_{yy}(a,b) - f_{xy}(a,b)^2\,.
\end{equation*}

\begin{enumerate}
\def\labelenumi{\arabic{enumi}.}
\item
  If \(D>0\) and \(f_{xx}(a,b) >0\), then \(f(a,b)\) is a local minimum.
\item
  If \(D>0\) and \(f_{xx}(a,b) <0\), then \(f(a,b)\) is a local maximum.
\item
  If \(D<0\), then \(f(a,b)\) is neither a local maximum nor local minimum.
\item
  If \(D=0\), then we cannot conclude.
\end{enumerate}

\end{theorem}

\begin{theorem}[Extreme value theorem]
If \(f\) is continuous on a \emph{closed} and \emph{bounded} set \(D\). Then,
\(f\) attains an absolute minimum and an absolute maximum in \(D\).
\end{theorem}

\hypertarget{algorithm-to-find-absolute-maxima-and-minima-on-closed-bounded-regions}{%
\subsubsection{Algorithm to find absolute maxima and minima on closed bounded regions}\label{algorithm-to-find-absolute-maxima-and-minima-on-closed-bounded-regions}}

\begin{enumerate}
\def\labelenumi{\arabic{enumi}.}
\item
  Find the values of \(f\) at the critical points of \(f\) in \(D\).
\item
  Find the extreme values of \(f\) on the boundary of \(D\).
\item
  The largest of the values from steps 1 and 2 is the absolute maximum value;
  the smallest of these values is the absolute minimum value.
\end{enumerate}

\hypertarget{constrained-optimization}{%
\subsection{Constrained optimization}\label{constrained-optimization}}

Constrained optimization takes various forms, depending on the assumptions.
We will deal with the most straight forward form.
The problem we will study is the following:

Maximize/minimize a function \(f:D\to \mathbb{R}\), subject to a constraint (side condition)
of the form
\(g(\mathbf{x}) = k\), for some constant \(k\in \mathbb{R}\).

\begin{theorem}[Method of Lagrange Multiplier]

Suppose the maximum/minimum values of \(f\) exist and \(\nabla g(\mathbf{x}) \not=0\) where \(g(\mathbf{x}) = k\).
To find the maximum and minimum values of \(f\) subject to constraint
\(g(\mathbf{x}) = k\), we do the following:

\begin{enumerate}
\def\labelenumi{\arabic{enumi}.}
\item
  Find all values of \(\mathbf{x}\) and \(\lambda \in \mathbb{R}\) such that
  \begin{equation*}
   \nabla f(\mathbf{x}) =\lambda \nabla g(\mathbf{x})\,,
  \end{equation*}
  and
  \begin{equation*}
   g(\mathbf{x}) = k \,.
  \end{equation*}
\item
  Evaluate \(f\) at all the points \(\mathbf{x}\) that result from step 1. The largest of
  these values is the maximum of \(f\); the smallest is the minimum value of \(f\).
\end{enumerate}

\end{theorem}

\hypertarget{multiple-integrals}{%
\section{Multiple integrals}\label{multiple-integrals}}

\textbf{Notations:}

Rectangle \(R= [a,b]\times [c,d]\).

\begin{definition}

Let \(f\) be a function on a rectangle \(R\).
A double Riemann sum for \(f\) over \(R\) is a sum of the following form
\begin{equation*}
 \sum_{i=1}^m \sum_{j=1}^n f(x_{ij}^*, y_{ij}^*) \Delta A \,,
\end{equation*}
where

\begin{itemize}
\item
  \(\Delta A = \Delta x\times \Delta y\),
\item
  \(\Delta x = (b-a)/m\), \(\Delta y = (d-c)/n\),
\item
  \((x_{ij}^*,y_{ij}^*) \in R_{ij}\),
\item
  \(R_{ij} = [a + (i-1)\Delta x, a+ i\Delta x]\times [b + (j-1)\Delta y, b+ j\Delta y]\).
\end{itemize}

\end{definition}

\begin{definition}
The double integral of \(f\) over a rectangle \(R\) is
\begin{equation*}
    \iint_{R} f(x,y) \, dA = \lim_{m,n\to \infty} \sum_{i=1}^m \sum_{j=1}^n f(x_{ij}^*, y_{ij}^*) \Delta A \,.
\end{equation*}
\end{definition}

Similar process can be generalized to \(n\)-fold integral over a \(n\)-dimensional cube.

\end{document}
